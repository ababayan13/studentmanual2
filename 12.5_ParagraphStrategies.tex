\section{\sloppy Develop Your Own Answer Before Looking at the Answer Choices}

Making this a habit will not only promote clear thinking but also insulate you from being swayed by tempting but incorrect answer choices.

\subsection{Practice This Strategy}

\textit{Directions: Read the following passage. Then, write what you think could be a plausible correct answer for the questions that follow.}

\bigskip
\begin{inparaenum}[\bfseries 1]
\indent \item Animals experience stress if they are not in an environment where they can express their evolved preferences. \item Therefore, Brian Hare, head of the Canine Cognition Center at Duke University argues that scientists should take into account the laboratory animal's stressors when designing protocols so that they can minimize these events and therefore collect data that is more reflective of the animal's natural responses. \item The foundation of the preference based approach is that animals should be treated based on the preferences of their species rather than more sweeping regulations. \item Since wild minks spend the majority of time in and around water, when a paper written by Dr. Mason in 2001 gave farm-raised minks a choice of a spacious room with preferred foods and toys or a smaller one with a water pool, the minks overwhelmingly chose the latter. \item As a ``proof of concept'', the cortisol levels (used as a measure of stress) of the minks with access to the pool were significantly lower after swimming.

\indent \item The Canine Cognition Center uses the preference based approach by recruiting human volunteers with their pet dogs to participate in research studies. \item Motivated by love and interest in dogs, many are interested in bringing their dog to the Biological Sciences building to participate. \item Because domestic dogs like to be near humans, Hare contends these results are more accurate than if the dogs were to be raised in animal storage facilities. 

\indent \item There are, however, sources of bias in these experiments that have been acknowledged by other scientists. \item In addition to the limitations of developing investigator-subject relationships, it can be tempting to think that because there are lots of different dogs being tested, the results can be generalized to all dogs. \item There is a sample bias because the people that bring their dogs to be tested at the Center are more likely to take good care of their dogs and be at least somewhat interested in dog cognition. \item The investigators can not control for the past experiences of the dogs being tested, which is a potential confounding variable in their experiments. 
\end{inparaenum}


\begin{enumerate}
\item Where is the best place to insert the following sentence?

\textit{For example, animal cages in laboratory facilities generally do not contain pools.}

My answer: \hrulefill

\vfill

\item Which of the following is the best revision of sentence 5 (reproduced below)?

\textit{As a ``proof of concept'', the cortisol levels (used as a measure of stress) of the minks with access to the pool were significantly lower after swimming.}

My answer: \hrulefill

\vfill

\item Of the following, which is the best way to phrase sentence 8 (reproduced below)?

\textit{Because domestic dogs like to be near humans, Hare contends these results are more accurate than if the dogs were to be raised in animal storage facilities.}

My answer: \hrulefill

\vfill

\item Which revision appropriately shortens sentence 9 (reproduced below)?

\textit{There are, however, sources of bias in these experiments that have been acknowledged by other scientists.}

My answer: \hrulefill

\vfill

\item Which of the following, if placed after sentence 11, would be the most effective concluding sentence for the essay?

My answer: \hrulefill
\end{enumerate}