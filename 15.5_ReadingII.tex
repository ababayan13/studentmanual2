\section{\sloppy SAT Worksheet: Practice with Passage-Based Reading Questions}

Directions: For each passage, underline the main point in each paragraph and put an arrow next
to the main idea of the passage. Then, answer the questions that follow. \textbf{After you are finished
answering the question, put the phrase and line number from the passage that served
as your evidence for your answer.}

\textbf{Questions 1-2 are based on the following passage.}

\begin{linenumbers*}
\modulolinenumbers[5] 
\indent I was born in the town of Kingston, in the island of Jamaica, some time in the present century. As a female, and a widow, I may be well excused giving the precise date of this important event. But I do not mind confessing that the century and myself were both young together, and that we have grown side by side into age and consequence. I am a Creole, and have good Scotch blood coursing in my veins. My father was a soldier, of an old Scotch family; and to him I often trace my affection for a camp-life, and my sympathy with what I have heard my friends call ``the pomp, pride, and circumstance of glorious war.'' All my life long I have followed the impulse which led me to be up and doing; and so far from resting idle anywhere, I have never wanted inclination to rove, nor will powerful enough to find a way to carry out my wishes. That these qualities have led me into many countries, and brought me into some strange and amusing adventures, the reader, if he or she has the patience to get through this book, will see. Some people, indeed, have called me quite a female Ulysses. I believe that they intended it as a compliment; but from my experience of the Greeks, I do not consider it a very flattering one.
\end{linenumbers*}

\textit{Adapted from \url{https://www.gutenberg.org/files/23031/23031-h/23031-h.htm}}

\begin{enumerate}
\item The passage is primarily concerned with the author's 

\begin{enumerate}[label=(\Alph*)]
\item how her friends view her
\item beginnings in Jamaica
\item her pride in her own strength
\item biracial background
\item influences on her personality
\end{enumerate}

\bigskip
\textbf{Evidence:} \hrulefill

\item In the passage, what does the author imply about Ulysses?

\begin{enumerate}[label=(\Alph*)]
\item he had formidable power
\item he was heavily influenced by the Greek society
\item he prided himself on his wartime abilities
\item he may have unflattering characteristics
\item he was well-admired
\end{enumerate}

\bigskip
\textbf{Evidence:} \hrulefill

\end{enumerate}

\textbf{Questions 1-4 are based on the following passage.}

\begin{linenumbers*}
\modulolinenumbers[5] 
\indent Has no writer ever dealt with the dramatic aspect of the unopened envelope? I cannot recall such a passage in any of my authors, and yet to my mind there is much matter for philosophy in what is always the expressionless shell of a boundless possibility. Your friend may run after you in the street, and you know at a glance whether his news is to be good, bad, or indifferent; but in his handwriting on the breakfast-table there is never a hint as to the nature of his communication. Whether he has sustained a loss or an addition to his family, whether he wants you to dine with him at the club or to lend him ten pounds, his handwriting at least will be the same, unless, indeed, he be offended, when he will generally write your name with a studious precision and a distant grace quite foreign to his ordinary calligraphy.

\indent These reflections, trite enough as I know, are nevertheless inevitable if one is to begin one's unheroic story in the modern manner, at the latest possible point. That is clearly the point at which a waiter brought me the fatal letter from Catherine Evers. Apart even from its immediate consequences, the letter had a \textbf{prima facie} interest, of no ordinary kind, as the first for years from a once constant correspondent. And so I sat studying the envelope with a curiosity too piquant not to be enjoyed. What in the world could so obsolete a friend find to say to one now? Six months earlier there had been a certain opportunity for an advance, which at that time could not possibly have been misconstrued; when they landed me, a few later, there was another and perhaps a better one. But this was the last summer of the late century, and already I was beginning to get about like a lamplighter on my two sticks. Now, young men about town, on two walking-sticks, in the year of grace 1900, meant only one thing. Quite a stimulating thing in the beginning, but even as I write, in this the next winter but one, a national irritation of which the name alone might prevent you from reading another word.
\end{linenumbers*}

\textit{From: \url{https://www.gutenberg.org/files/11153/11153-h/11153-h.htm}}

\begin{enumerate}

\item What is the role of the question in the opening paragraph (line 1)?

\begin{enumerate}[label=(\Alph*)]
\item to mirror the suspense felt when given a letter
\item to gauge audience interest in the topic of opening a letter
\item to foreshadow his reading of a letter
\item to highlight his interest in the contents of Catherine Ever's letter
\item to introduce his surprise at the quantity of writings about receiving a letter
\end{enumerate} 

\bigskip
\textbf{Evidence:} \hrulefill

\bigskip
\item Why is the author excited by the letter from Catherine Evers?

\begin{enumerate}[label=(\Alph*)]
\item he wants to read if there has been loss or an addition to the family
\item he wants to know why she is writing to him after they had fallen out of touch
\item he was interested in what she has written to him in the past
\item he relishes in the drama of opening a letter
\item he enjoys looking at the handwriting on the envelope
\end{enumerate} 

\bigskip
\textbf{Evidence:} \hrulefill

\bigskip
\item The tone of this passage is

\begin{enumerate}[label=(\Alph*)]
\item intense excitement
\item banal irritability
\item hopeful yearning
\item dramatic nostalgia 
\item bittersweet diffidence 
\end{enumerate}

\bigskip
\textbf{Evidence:} \hrulefill

\bigskip
\item Prima facie (in bold) most closely means

\begin{enumerate}[label=(\Alph*)]
\item obnoxious
\item clear
\item half-hearted
\item mundane
\item positive
\end{enumerate}

\bigskip
\textbf{Evidence:} \hrulefill

\end{enumerate}

\textbf{Questions 1-3 are based on the following passage.}

\begin{linenumbers*}
\modulolinenumbers[5]
\indent Of the ultimate nature of electricity, as of that of heat and light, we are at present ignorant. But it has been clearly established that all three phenomena are but manifestations of the energy pervading the universe. By means of suitable apparatus one form can be converted into another form. The heat of fuel burnt in a boiler furnace develops mechanical energy in the engine which the boiler feeds with steam. The engine revolves a dynamo, and the electric current thereby generated can be passed through wires to produce mechanical motion, heat, or light. We must remain content, therefore, with assuming that electricity is energy or motion transmitted through the ether from molecule to molecule, or from atom to atom, of matter. Scientific investigation has taught us how to produce it at will, how to \textbf{harness} it to our uses, and how to measure it; but not what it is. That question may, perhaps, remain unanswered till the end of human history. A great difficulty attending the explanation of electrical action is this—that, except in one or two cases, no comparison can be established between it and the operation of gases and fluids. When dealing with the steam-engine, any ordinary intelligence soon grasps the principles which govern the use of steam in cylinders or turbines. The diagrams show, it is hoped, quite plainly ``how it works.'' But electricity is elusive, invisible; and the greatest authorities cannot say what goes on at the poles of a magnet or on the surface of an electrified body. Even the existence of ``negative'' and ``positive'' electricity is problematical. However, we see the effects, and we know that if one thing is done another thing happens; so that we are at least able to use terms which, while convenient, are not at present controverted by scientific progress.

\indent Rub a vulcanite rod and hold one end near some tiny pieces of paper. They fly to it, stick to it for a time, and then fall off. The rod was electrified—that is, its surface was affected in such a way as to be in a state of molecular strain which the contact of the paper fragments alleviated. By rubbing large surfaces and collecting the electricity in suitable receivers the strain can be made to relieve itself in the form of a violent discharge accompanied by a bright flash. This form of electricity is known as static.
Next, place a copper plate and a zinc plate into a jar full of diluted sulphuric acid. If a wire be attached to them a current of electricity is said to flow along the wire. We must not, however, imagine that anything actually moves along inside the wire, as water, steam, or air, passes through a pipe. Professor Trowbridge says, ``No other agency for transmitting power can be stopped by such slight obstacles as electricity. A thin sheet of paper placed across a tube conveying compressed air would be instantly ruptured. It would take a wall of steel at least an inch thick to stand the pressure of steam which is driving a 10,000 horse-power engine. A thin layer of dirt beneath the wheels of an electric car can prevent the current which propels the car from passing to the rail, and then back to the power-house.'' There would, indeed, be a puncture of the paper if the current had a sufficient voltage, or pressure; yet the fact remains that current electricity can be very easily confined to its conductor by means of some insulating or nonconducting envelope.

\indent The most familiar form of electricity is that known as magnetism. When a bar of steel or iron is magnetized, it is supposed that the molecules in it turn and arrange themselves with all their north-seeking poles towards the one end of the bar, and their south-seeking poles towards the other. If the bar is balanced freely on a pivot, it comes to rest pointing north and south; for, the earth being a huge magnet, its north pole attracts all the north-seeking poles of the molecules, and its south poles the south-seeking poles. (The north-seeking pole of a magnet is marked N., though it is in reality the south pole; for unlike poles are mutually attractive, and like poles repellent.)

\indent There are two forms of magnet -- permanent and temporary. If steel is magnetized, it remains so; but soft iron loses practically all its magnetism as soon as the cause of magnetization is withdrawn. This is what we should expect; for steel is more closely compacted than iron, and the molecules therefore would be able to turn about more easily. It is fortunate for us that this is so, since on the rapid magnetization and demagnetization of soft iron depends the action of many of our electrical mechanisms.
\end{linenumbers*}

\textit{From: \url{https://www.gutenberg.org/files/28553/28553-h/28553-h.htm\#Chapter\_V}}

\begin{enumerate}

\item The primary purpose of this passage is 

\bigskip
\begin{enumerate}[label=(\Alph*)]
\item to describe magnetism, a common form of electricity
\item to narrate the history of electricity and widespread opinions on its use
\item to contrast electricity with a more concrete forces
\item to explain why electricity is important
\item to illustrate the concept of electricity
\end{enumerate}

\bigskip
\textbf{Evidence:} \hrulefill

\item Harness most nearly means

\bigskip
\begin{enumerate}[label=(\Alph*)]
\item to connect or tie
\item to coerce
\item to deliver
\item to relish
\item to empower
\end{enumerate}

\bigskip
\textbf{Evidence:} \hrulefill

\item The narrator most likely describes several experiments in paragraph two in order to

\bigskip
\begin{enumerate}[label=(\Alph*)]
\item to illustrate a concept that is difficult to describe
\item to depict the experiment that led to the discovery of static electricity
\item to contrast static electricity and magnetism 
\item to demonstrate the omnipresence of electricity in our daily lives
\item to explain why static electricity is more efficient than magnetism 
\end{enumerate}

\bigskip
\textbf{Evidence:} \hrulefill

\item According to the passage, the difference between the two types of magnetism is due to

\bigskip
\begin{enumerate}[label=(\Alph*)]
\item differences in the number of molecules in the materials
\item susceptibility of specific materials moving towards the north or south pole
\item properties of different materials
\item the differences between experimenters
\item the differences in experimental set up
\end{enumerate}

\bigskip
\textbf{Evidence:} \hrulefill

\end{enumerate}