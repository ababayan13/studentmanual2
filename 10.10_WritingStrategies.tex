\section{Faulty Comparisons}

Items being compared must have the same identity. For example, a dog can not be compared to another dog's toys. While this might sound easy, it isn't always easy because our brain is used to making the correct comparison even if it is written incorrectly on the page.

For example,

\begin{enumerate}
\item Spot is better than Ziggy's toys.

\begin{itemize}
\item The sentence is trying to compare Spot and Ziggy or Spot's toys and Ziggy's toys.
\item The sentence is actually comparing Spot and Ziggy's toys.
\item This sentence can be corrected at least two different ways. Spot's toys are better
than Ziggy's toys. OR Spot's toys are better than Ziggy's.
\end{itemize}

\item Paul's pet rock is larger than Jim.

\begin{itemize}
\item What two things is the sentence trying to compare? \hrulefill
\item What two things is the sentence actually comparing? \hrulefill
\item Re-write the sentence so that it is grammatically correct (Note: there are at least two
ways to do this.): \hrulefill
\end{itemize}
\end{enumerate}

\subsection{SAT Practice Questions}

\begin{enumerate}

\item \begin{inparaenum}[A]
J.K. Rowling’s books \tfrac{have inspired}{\item} millions with stories \tfrac{of}{\item} good triumphing over evil and the power of friendship, \tfrac{whereas}{\item} the new \tfrac{author}{\item} has not. \tfrac{No Error}{\item}
\end{inparaenum}

\item \begin{inparaenum}[A]
The cameraman  \tfrac{told}{\item} the celebrity that  \tfrac{he}{\item} should position  \tfrac{himself}{\item} \tfrac{closer to}{\item} the camera. \tfrac{No Error}{\item}
\end{inparaenum}

\end{enumerate} 
