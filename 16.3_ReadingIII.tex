\section{Practice}

\textit{Directions: Read the following passages and answer the questions that follow.}

\bigskip
\textbf{Questions 1-5 refer to the following passage.}

\bigskip

\textit{The following is a speech made to the Conference of Regional Chairmen of the Highways Transport Committee in 1918.}

\begin{linenumbers*}
\modulolinenumbers[5]
\indent I did not come today with the idea of bringing you anything new. On the contrary, I have come here to get the inspiration which association with those from the outside gives. There is no hope for this place unless we can keep in contact with the remainder of the United States. In isolation we think in a vacuum, and it is only when we know what you are thinking of on the outside that we get the impulse which leads to construction. I think I can say out of my knowledge of 12 years of administrative work in this city, that we have to look abroad, go up on the tops of the hills and see the great valleys of our country, before we know really what our policies should be. When we live alone or live in isolation and try to deal with things abstractly or theoretically we make mistakes.
The problem that you deal with is one that I have never had any contact with, but I know this from my knowledge of history; that you can judge the civilization of a nation, of a people, of a continent, or of any part of a nation, by the character of its highways. If you will think over that proposition you will realize that what I have said is true, that those parts of this Nation are most backward, where people live most alone, where they develop those diseases of the mind which come from living alone, where they develop supreme discontent with what is done at Washington or what is done in their own State legislatures, where they are unhappy and discontented, and movements that make against the welfare of our country arise, are those parts where there are poor highways and consequently a lack of communication between the people.

\indent Our eyes are all turned at this time to the other side of the water. I suppose that there has never been a month in the history of the United States when so many people were so anxious to see the morning paper or the evening paper as during the past month. There never has been a time when we have been so thrilled to the very core of our beings. Achievements that those boys over there have made are things that will live in our memories.

\indent And why has it been possible for France to carry on for four years a successful war against the greatest military power that the world has ever seen? Because France had the benefit of the engineering skill and of the foresight of two men who are 1,800 years apart—Napoleon and Caesar. Those men built the roads of France. Without those roads, conceived and built originally by Caesar for the conquest of the Gauls and for the conquest of the Teutons, without the roads built by Napoleon to stand off the enemies of France and to make aggressions to the eastward, Paris would have fallen at least two years ago. So that you gentlemen who are engaged in the business of developing the highways of the country and putting them to greater use may properly conceive of yourselves as engaged in a very farsighted, important bit of statemanship, work that does not have its only concern as to the farmer of this country or the helping of freight movement during this winter alone, but may have consequences that will extend throughout the centuries. Take the instance of Verdun. Verdun would have fallen unquestionably if it had not been for the roads that Napoleon constructed and that France has maintained; for all the credit is not to go to the man who conceived and the man who constructed. This is one thing where we have been short always. One thing that the people of the United States do not realize. It is not sufficient to pay 25,000 dollars a mile for a concrete foundation, but you must put aside 10 cents out of every dollar for the maintenance of these roads or your money has gone to waste and your conception is idle. And you gentlemen know, if you continue, as I hope you will, after the war, you will have not merely a function in the securing of the building of good roads, but will have a very great function in the maintaining of these roads as actual arteries in the system of transportation of the country. You remember that at Verdun the railroad was cut off, and Verdun was supported by the fact that she had trucks which could go 40 feet apart all night long over the great highway that had been built from Paris to the east.

\indent Now I saw my first national service in connection with the Interstate Commerce Commission and I was much impressed by the theory that the railroad men had, which was a very natural theory, arising out of their own experience and out of the fact that there was a new force in the world with which they were playing. Their conception was that the highway was a mere means of getting from the farm to the railroad; that the waterway was a mere means of carrying off the surplus waters from the hills to the oceans. The statement has often been made to me that there would never be an occasion when it would be necessary or possible to put into competition with the railroads the waterways of this country; that it would cost more to use those waterways or to use highways than it would to do the same transportation work by railroad. And they had obtained figures to show that under conditions of unlimited competition the Illinois Central, for instance, paralleling the Mississippi River, could do business at a cheaper rate than it could be transported by water, considering the cost of bringing it to the water station and unloading it at the other end. Now, as Mr. Chapin has said, a larger conception has come into the American mind—the conception of the utilization of all our resources. While the railroad has a great burden cast upon it; while it is the strong right arm in this work, still we must remember that the strong right arm must have fingers, and that there should be in a complete physical system a good left arm.

\indent The highways that you are interested in are more than interesting to me for another reason. I have thought of the men who will come back after the war. Every nation has had a problem to deal with the returning soldier. If you read Ferraro's history of Rome, you will find that one of the chief reasons why the republic of Rome went out of existence and the empire of Rome came into existence was because of the returned soldiers. They looked to their general to take care of them on their return, and their general found that the way to take care of them was to give them, as they said in those days, "bread and circuses," and so they reached over into Egypt, got the great wheat supply of that country, and provided the great circuses that are historical for the amusement of those people.

\indent The Emperor of Germany 10 years ago was asked why he was unwilling to agree to a demobilization of his forces or to a reduction of his army and he said because it would demoralize the industries of Germany. They could not reabsorb so many men without reducing wages and throwing upon the country so many unemployed that it would make against the welfare of the land. We will have that problem to deal with.

\indent The firm, strong position taken by the President in his note published yesterday indicates that he is ready to fight this thing out to a finish and that he will show to those on the other side that America has a determination to win, and that it is not a determination that fades quickly. If the Emperor of Germany has ever had a good look at a photograph of Woodrow Wilson, he has seen a prolongation of a chin that must have confirmed him in the belief that America does not take up a fight unless it puts it through; and we are to reach a military determination by whipping them until they say they have had enough.

\indent Now, when this thing is over, our men will begin to come back into the United States. But not all at once. We won't have three or four million men to deal with in a single month. We will have them slowly returning to us through a year or a year and a half. As those men come filtering in through our ports we ought to be able to meet every man at every port with the statement that he does not have to lie idle one single day. We ought to be able to say to the man,``Here is something that you can do at once. If your old position is not vacant, if you can not go home to the old place and take up the work that you were in, then the Government of the United States, in its wisdom, has provided something which you can do at wages upon which you can live well.''

\indent And what should that be? The greatest problem that any country has, to my mind, is its own self-support. We have come to be independent in our resources, to be strong, and be respected. So long as we are industrially dependent, agriculturally dependent, somebody has a lever that he can use in a time of crisis, as against this nation. Long years ago we were the greatest of all agricultural people, and Thomas Jefferson wanted us to remain in that position. He thought that the safety and security of the United States lay in the fact that we would live on farms. When De Toquevile came over here in 1830 he said the reason democracy was a success in this country was because we were all practically living on farms, living on what we raised ourselves, and standing equally.

\indent Today the tendency is away from the farm toward the city, toward industrial life, toward aggregations of people, away from the small town to the larger town, and from the larger town to the metropolis. People are being drawn from the farms, so that one-half of the \textbf{arable} land this side of the Mississippi is unused today; so that between here and New Orleans there are 40,000,000 acres of land privately owned and unused; so that in the great Northwest, Minnesota, Oregon, Washington, etc., there are 100,000,000 acres of cut-over lands that are practically unused; and we have a new nation practically in the undrained lands of our rivers and our bays and inlets, lands that are as rich as any that lie out of doors, as rich as the valley of the Nile or of the Euphrates. In the far western country, there are at least 15,000,000 acres of land that we can put under water. Under water, that land produces more than one crop a year, and that an exceptionally rich crop.

\indent We have been extending ourselves because of war in a great many different directions. The Government has taken to itself unprecedented and unthought-of powers because of the necessities of our condition. I say that to meet the problem of the returned soldier we ought to take advantage of this opportunity to do the work now that must eventually be done and reclaim these arid lands of the West. Turn the waters of the Colorado over the desert of Arizona, store those waters in the Grand River and in the Green River, and let them flow down at the right times on that desert so as to raise cotton and cantaloupes and alfalfa. Then come east and take the stumps from these cut-over lands. Do it not as a private enterprise, because that is a slow, slow process. Men are discouraged and disheartened when they look at the problem of pulling an Oregon fir stump out of the ground. It really requires large capital. Then come farther east and take these lands that are swamp, that need draining, and build ditches and dikes and put these lands into the service of America. This is what I call the making of the nation.
\end{linenumbers*}

From: https://www.gutenberg.org/files/19759/19759-h/19759-h.htm

\bigskip
\begin{enumerate}

\item What is the mood of the passage?

\bigskip
\begin{enumerate}[label=(\Alph*)]
\item unbridled enthusiasm
\item jovial to the point of flippant
\item enthusiastic yet professional
\item morose and determined
\item hopeful but condescending
\end{enumerate}

\bigskip
\textbf{Evidence:} \hrulefill

\bigskip
\item Which of the following is NOT listed as a reason to construct new highways?

\bigskip
\begin{enumerate}[label=(\Alph*)]
\item to allow for the transportation of goods during wartime
\item to encourage people to move to farming areas
\item to promote national unity
\item as a defense from foreign attack
\item to employ former soldiers
\end{enumerate}

\bigskip
\item Why did the author reference Thomas Jefferson's ideas for a farm-based nation (paragraph 9)?

\bigskip
\begin{enumerate}[label=(\Alph*)]
\item Jefferson supported the construction of farms and highways to connect the farms
\item it uses the power of the Founding Fathers to garner support for broad laws that will promote farming
\item it underscores the importance of self-sufficiency
\item because Jefferson supported the idea of an agricultural nation
\item because Jefferson agrees with the author that  farming states would survive better in times of war
\end{enumerate}

\bigskip
\textbf{Evidence:} \hrulefill

\bigskip
\item The word ``arable'' (in bold) most closely means

\bigskip
\begin{enumerate}[label=(\Alph*)]
\item full of air
\item able to be purchased
\item able to be changed
\item suitable for farming
\item large quantity
\end{enumerate}

\bigskip
\textbf{Evidence:} \hrulefill


\bigskip
\item What does the author assume with the use of the following quotation, ``The Government has taken to itself unprecedented and unthought-of powers because of the necessities of our condition'' in paragraph 3?

\bigskip
\begin{enumerate}[label=(\Alph*)]
\item some will oppose the federal government building highways
\item the Government policies have unprecedented levels of support
\item there is currently strong national unity and agreement on the building of highways
\item we should think critically about the roles of all branches of government
\item the Government should be involved in helping soldiers adjust to life at home
\end{enumerate}

\bigskip
\textbf{Evidence:} \hrulefill

\end{enumerate}

\bigskip
\textbf{Questions 1-5 refer to the following passage.}

\bigskip
\begin{linenumbers*}
\modulolinenumbers[5]
\indent There is, accordingly, a marked difference between the education which every one gets from living with others, as long as he really lives instead of just continuing to subsist, and the deliberate educating of the young. In the former case the education is incidental; it is natural and important, but it is not the express reason of the association. While it may be said, without exaggeration, that the measure of the worth of any social institution, economic, domestic, political, legal, religious, is its effect in enlarging and improving experience; yet this effect is not a part of its original motive, which is limited and more immediately practical. Religious associations began, for example, in the desire to secure the favor of overruling powers and to ward off evil influences; family life in the desire to gratify appetites and secure family perpetuity; systematic labor, for the most part, because of enslavement to others, etc. Only gradually was the by-product of the institution, its effect upon the quality and extent of conscious life, noted, and only more gradually still was this effect considered as a directive factor in the conduct of the institution. Even today, in our industrial life, apart from certain values of industriousness and thrift, the intellectual and emotional reaction of the forms of human association under which the world's work is carried on receives little attention as compared with physical output.

\indent But in dealing with the young, the fact of association itself as an immediate human fact, gains in importance. While it is easy to ignore in our contact with them the effect of our acts upon their disposition, or to subordinate that educative effect to some external and tangible result, it is not so easy as in dealing with adults. The need of training is too evident; the pressure to accomplish a change in their attitude and habits is too urgent to leave these consequences wholly out of account. Since our chief business with them is to enable them to share in a common life we cannot help considering whether or no we are forming the powers which will secure this ability. If humanity has made some headway in realizing that the ultimate value of every institution is its distinctively human effect--its effect upon conscious experience--we may well believe that this lesson has been learned largely through dealings with the young.

\indent We are thus led to distinguish, within the broad educational process which we have been so far considering, a more formal kind of education—that of direct tuition or schooling. In undeveloped social groups, we find very little formal teaching and training. Savage groups mainly rely for instilling needed dispositions into the young upon the same sort of association which keeps adults loyal to their group. They have no special devices, material, or institutions for teaching save in connection with initiation ceremonies by which the youth are inducted into full social membership. For the most part, they depend upon children learning the customs of the adults, acquiring their emotional set and stock of ideas, by sharing in what the elders are doing. In part, this sharing is direct, taking part in the occupations of adults and thus serving an apprenticeship; in part, it is indirect, through the dramatic plays in which children reproduce the actions of grown-ups and thus learn to know what they are like. To savages it would seem preposterous to seek out a place where nothing but learning was going on in order that one might learn.

\indent But as civilization advances, the gap between the capacities of the young and the concerns of adults widens. Learning by direct sharing in the pursuits of grown-ups becomes increasingly difficult except in the case of the less advanced occupations. Much of what adults do is so remote in space and in meaning that playful imitation is less and less adequate to reproduce its spirit. Ability to share effectively in adult activities thus depends upon a prior training given with this end in view. Intentional agencies—schools—and explicit material--studies--are devised. The task of teaching certain things is delegated to a special group of persons.
Without such formal education, it is not possible to transmit all the resources and achievements of a complex society. It also opens a way to a kind of experience which would not be accessible to the young, if they were left to pick up their training in informal association with others, since books and the symbols of knowledge are mastered.

\indent But there are conspicuous dangers attendant upon the transition from indirect to formal education. Sharing in actual pursuit, whether directly or vicariously in play, is at least personal and vital. These qualities compensate, in some measure, for the narrowness of available opportunities. Formal instruction, on the contrary, easily becomes remote and dead--abstract and bookish, to use the ordinary words of depreciation. What accumulated knowledge exists in low grade societies is at least put into practice; it is \textbf{transmuted} into character; it exists with the depth of meaning that attaches to its coming within urgent daily interests.
But in an advanced culture much which has to be learned is stored in symbols. It is far from translation into familiar acts and objects. Such material is relatively technical and superficial. Taking the ordinary standard of reality as a measure, it is artificial. For this measure is connection with practical concerns. Such material exists in a world by itself, unassimilated to ordinary customs of thought and expression. There is the standing danger that the material of formal instruction will be merely the subject matter of the schools, isolated from the subject matter of life-experience. The permanent social interests are likely to be lost from view. Those which have not been carried over into the structure of social life, but which remain largely matters of technical information expressed in symbols, are made conspicuous in schools. Thus we reach the ordinary notion of education: the notion which ignores its social necessity and its identity with all human association that affects conscious life, and which identifies it with imparting information about remote matters and the conveying of learning through verbal signs: the acquisition of literacy.

\indent Hence one of the weightiest problems with which the philosophy of education has to cope is the method of keeping a proper balance between the informal and the formal, the incidental and the intentional, modes of education. When the acquiring of information and of technical intellectual skill do not influence the formation of a social disposition, ordinary vital experience fails to gain in meaning, while schooling, in so far, creates only ``sharps'' in learning--that is, egoistic specialists. To avoid a split between what men consciously know because they are aware of having learned it by a specific job of learning, and what they unconsciously know because they have absorbed it in the formation of their characters by intercourse with others, becomes an increasingly delicate task with every development of special schooling.
\end{linenumbers*}

From: https://www.gutenberg.org/files/852/852-h/852-h.htm

\bigskip
\begin{enumerate}
\item The author of the passage views the beginnings of associations as
\begin{enumerate}[label=(\Alph*)]
\item hostile
\item utilitarian
\item unnecessary
\item difficult to maintain
\item progressive 
\end{enumerate}

\bigskip
\textbf{Evidence:} \hrulefill

\bigskip
\item By including the learning of ``savages'', what does the author attempt to do?
\begin{enumerate}[label=(\Alph*)]
\item to highlight that other societies have other opinions on schooling
\item to contrast it to formal schooling
\item to urge the integration of savages into the society and more specifically, formal schooling
\item to demonstrate that they are superior to the savage groups in many aspects of society
\item to discredit their way of learning and promote the formal school model
\end{enumerate}

\bigskip
\textbf{Evidence:} \hrulefill

\item The quotation at the beginning of the 5th paragraph ``But there are conspicuous dangers attendant upon the transition from indirect to formal education'' serves to do what?

\begin{enumerate}[label=(\Alph*)]
\item Give evidence that informal schooling should be emphasized over formal schooling
\item Make themselves more similar to the ``undeveloped societies''
\item Illustrate the potential pitfalls of formal schooling
\item Encourage homeschooling
\item Highlight the difficulties faced by all teachers
\end{enumerate}

\bigskip
\textbf{Evidence:} \hrulefill


\bigskip
\item ``Transmuted'' most nearly means

\begin{enumerate}[label=(\Alph*)]
\item extol
\item transience 
\item sorted
\item placed
\item depended upon
\end{enumerate}

\bigskip
\textbf{Evidence:} \hrulefill

\bigskip
\item What does the author view as the most important topic that students learn in formal schooling?

\bigskip
\begin{enumerate}[label=(\Alph*)]
\item reading
\item sharing
\item training for a specific profession
\item domestic duties
\item specialization in one field of study
\end{enumerate}

\bigskip
\textbf{Evidence:} \hrulefill
\end{enumerate}

\bigskip
\textbf{Questions 1-5 refer to the following passage on early human life written over 100 years ago.}

\bigskip
\begin{linenumbers*}
\modulolinenumbers[5]
\indent Although we have taken up the question of punishment and the manner of dealing with various childish iniquities before the question of character-building, it has only been done in order to clear the mind of some current misconceptions. In the statements of Froebel's simple and positive philosophy of child culture, misconception on the part of the reader must be guarded against, and these misconceptions generally arise from a feeling that, beautiful as his optimistic philosophy may be, there are some children too bad to profit by it--or at least that there are occasions when it will not work out in practice. In the preceding section we have endeavored to show in detail how this method applies to a representative list of faults and shortcomings, and having thus, we hope, proved that the method is applicable to a wide range of case--indeed to all possible cases--we will proceed to recount the fundamental principles which Froebel, and before him Pestalozzi, enunciated; which times who adhere to the new education are to-day working out into the detail of school-room practice.

\indent  As previously stated, the object of the moral training of the child is the \textbf{inculcation} of the love of righteousness. Froebel is not concerned with laying down a mass of observances which the child must follow, and which the parents must insist upon. He thinks rather that the child's nature once turned into the right direction and surrounded by right influences will grow straight without constant yankings and twistings. The child who loves to do right is safe. He may make mistakes as to what the right is, but he will learn by these mistakes, and will never go far astray.

\indent However, it is well to save him as far as possible from the pain of these mistakes. We need to preserve in him what has already been implanted there; the love of understanding the reasons for conduct. When the child asks ``Why?'' therefore, he should seldom be told ``Because mother says so.'' This is to deny a rightful activity of his young mind; to give him a monotonous and insufficient reason, temporary in its nature, instead of a lasting reason which will remain with him through life. Dante says all those who have lost what he calls ``the good of the intellect'' are in the Inferno. And when you refuse to give your child satisfactory reasons for the conduct you require of him, you refuse to cultivate in him that very good of the intellect which is necessary for his salvation.

\indent As soon, however, as your commands become positive instead of negative, the difficulty of meeting the situation begins to disappear. It is usually much easier to tell the child why he should do a thing than why he should not do its opposite. For example, it is much easier to make him see that he ought to be a helpful member of the family than to make him understand why he should stop making a loud noise, or refrain from waking up the baby. There is something in the child which in calm moments recognizes that love demands some sacrifice. To this something you must appeal and these calm moments, for the most part, you must choose for making the appeal. The effort is to prevent the appearance of evil by the active presence of good. The child who is busy trying to be good has little time to be naughty.

\indent Froebel's most characteristic utterance is perhaps this: ``A suppressed or perverted good quality--a good tendency, only repressed, misunderstood, or misguided--lies originally at the bottom of every shortcoming in man. Hence the only and infallible remedy for counteracting any shortcoming and even wickedness is to find the originally good source, the originally good side of the human being that has been repressed, disturbed, or misled into the shortcoming, and then to foster, build up, and properly guide this good side. Thus the shortcoming will at last disappear, although it may involve a hard struggle against habit, but not against original depravity in man, and this is accomplished so much the more rapidly and surely because man himself tends to abandon his shortcomings, for man prefers right to wrong.'' The natural deduction from this is that we should say ``do'' rather than ``don't; open up the natural way for rightful activity instead of uttering loud warning cries at the entrance to every wrong path.

\indent It is for this reason that the kindergarten tries by every means to make right doing delightful. This is one of the reasons for its songs, dances, plays, its bright colors, birds, and flowers. And in this respect it may well be imitated in every home. No one loves that which is disagreeable, ugly, and forbidding; yet many little children are expected to love right doing which is seldom attractively presented to them.

\indent The results of such treatment are apparent in the grown people of today. Most persons have an underlying conviction that sinners, or at any rate unconscientious persons, have a much easier and pleasanter time of it than those who try to do right. To the imagination of the majority of adults sin is dressed in glittering colors and virtue in gray, somber garments. There are few who do not take credit for right doing as if they had chosen a hard and disagreeable part instead of the more alluring ways of wrong. This is because they have been mistaught in childhood and have come to think of wrongdoing as pleasant and virtue as hard, whereas the real truth is exactly the opposite. It is wrongdoing that brings unpleasant consequences and virtue that brings happiness.

\indent There are those who object that by the kindergarten method right doing is made too easy. The children do not have to put forth enough effort, they say; they are not called upon to endure sufficient pain; they do not have the discipline which causes them to choose right no matter how painful right may be for the moment. Whether this dictum is ever true or not, it certainly is not true in early childhood. The love of righteousness needs to be firmly rooted in the character before it is strained and pulled upon. We do not start seedlings in the rocky soil or plant out saplings in time of frost. If tests and trials of virtue must come, let them come in later life when the love of virtue is so firmly established that it may be trusted to find a way to its own satisfaction through whatever difficulties may oppose.

\indent In the very beginning of any effort to live up to Froebel's requirements it is evident that children must not be measured by the way they appear to the neighbors. This is to reaffirm the power of that rigid tradition which has warped so many young lives. She who is trying to fix her child's heart upon true and holy things may well disregard her neighbor's comments on the child's manners or clothes or even upon momentary ebullitions of temper. She is working below the surface of things, is setting eternal forces to work, and she cannot afford to interrupt this work for the sake of shining the child up with any premature outside polish. If she is to have any peace of mind or to allow any to the child, if she is to live in any way a simple and serene life, she must establish a few fundamental principles by which she judges her child's conduct and regulates her own, and stand by these principles through thick and thin.

\indent Perhaps the most fundamental principle is that enunciated by Fichte. ``Each man,'' he says, ``is a free being in a world of other free beings.'' Therefore his freedom is limited only by the freedom of the other free beings. That is, they must ``divide the world amongst them.'' Stated in the form of a command he says again, ``Restrict your freedom through the freedom of all other persons with whom you come in contact.'' This is a rule that even a three-year-old child can be made to understand, and it is astonishing with what readiness he will admit its justice. He call do anything he wants to, you explain to him, except bother other people. And, of course, the corollary follows that every one else can do whatever he pleases except to bother the child.
\end{linenumbers*}

From: https://www.gutenberg.org/files/13467/13467-h/13467-h.htm

\bigskip
\begin{enumerate}

\item 

\bigskip The author's critics would most likely characterize her as 
\begin{enumerate}[label=(\Alph*)]
\item humble
\item overly diffident
\item conservative
\item pragmatic
\item idealistic
\end{enumerate}

\bigskip
\textbf{Evidence:} \hrulefill

\bigskip
\item ``Inculcation'' (in bold) is most similar to the word

\bigskip
\begin{enumerate}[label=(\Alph*)]
\item adoption 
\item discernment 
\item making friends with
\item impart wisdom
\item None of the above
\end{enumerate}

\bigskip
\textbf{Evidence:} \hrulefill

\bigskip
\item The author says ``We do not start seedlings in the rocky soil or plant out saplings in time of frost.'' in paragraph seven in order to

\bigskip
\begin{enumerate}[label=(\Alph*)]
\item refute opposition to her claim that early schooling is a time to develop character
\item highlight the farming knowledge of her community
\item tell her opponents that 
\item demonstrate that she has working knowledge of agriculture and education
\item to break up the overly technical jargon with a relatable example
\end{enumerate}

\bigskip
\textbf{Evidence:} \hrulefill


\bigskip
\item The author of the passage would most likely agree/disagree with which of the following statements:

\bigskip
\begin{enumerate}[label=(\Alph*)]
\item If satisfactory answers are given to the question ``why'', then the child will grow up to be more inquisitive
\item Parents should rely on formal schooling, such as kindergarten, to teach their children with this method and then use their preferred parenting style
\item Parents should constantly be looking to correct unfavorable behavior in their children
\item That spanking a child as punishment is not effective at changing behavior
\item The use of her method would rehabilitate adult criminals. 
\end{enumerate}

\bigskip
\textbf{Evidence:} \hrulefill


\bigskip
\item Which of the following is a literary device employed by the author?

\bigskip
\begin{enumerate}[label=(\Alph*)]
\item irony
\item literary reference
\item rhetorical question
\item quotation from an education researcher
\item paradox
\end{enumerate}

\bigskip
\textbf{Evidence:} \hrulefill
\end{enumerate}