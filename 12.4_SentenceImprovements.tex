\section{Picking the Most Clear and Concise Sentence}

In SAT sentence improvement problems, you should try to eliminate the 2-3 answer choices that are grammatically incorrect so that you are left with 2-3 other answer choices. The SAT sentence improvement section is looking for the "best" sentence, one that is concise and precise. In SAT world, this translates to the sentence that is not only grammatically correct AND concise. How does the SAT measure "conciseness"? By length. \\

Therefore, the answer choice that you are looking for is grammatically correct and short without changing the meaning of the original sentence. The latter part means that it can not be so short that it is missing a key part of the original sentence,
but this is not usually an issue on sentence improvement problems. \\

\textit{Directions: Go back to the previous exercise and look at the answer choices that you haven't yet eliminated. Find the shortest answer. Then, check if it preserves the original meaning of the sentence by reading this answer choice in place of the underlined part of the original sentence. If so, this answer choice is the correct answer, so you should mark it. The first sentence is done as an example.}

\begin{enumerate}
\item The Boston Common is \ul{ older than it but still just as well-maintained as Central Park}.
\begin{enumerate}[label=(\Alph*)]

\item older than it but still just as well-maintained as Central Park  \ul{Eliminate because of ambiguous pronoun}
\item older than Central Park but just as well-maintained. \textbf{\ul{Correct. It is grammatically correct and the most concise. }}
\item  older than Central Park; it is just as well-maintained.    \ul{Eliminate because of ambiguous pronoun}
\item older and it is just as well-maintained as Central Park. \textbf{\ul{Grammatically correct but not as concise as (B).}}
\item just as comfortable as Central Park and it is older than it. \ul{Eliminate because of ambiguous pronoun}

\end{enumerate}

\end{enumerate} 