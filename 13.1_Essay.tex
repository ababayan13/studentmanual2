\section{SAT Worksheet: Warm-Up}

\textit{Directions: Complete the following paragraph improvement problem set.}

\bigskip
\textbf{Questions 1-3 refer to the following passage.}

\begin{inparaenum}[\bfseries 1]
\item `'Wise men talk because they have something to say; fools, because they have to say something.''  -Plato

\item Plato's quote above hung on the wall in my high school psychology teacher's room. \textbf{2} Every Friday when he hosts philosophy club, I would look at that sign and before I raised my hand to give an opinion on that week's topic, I would decide if the comment was worth sharing. \item I would ask myself honestly if it was the comment of a sage or a nincompoop; this was the first time that I was challenged to pick out what I really wanted to share and what was a superficial, fleeting thought. \item But in sorting these comments, I started to define what was really important to me, what ideas I would share and defend.

\item The thing that always baffled me about music in general, but particularly prevalent in jazz music, is how people improvise. \item In all seriousness, my goal for this jazz class was to learn how people go up onstage and ``just play'', hoping that the topic of how to improvise could be written in some elegant formula with the designated qualities. \item But what I learned from this class is improvising is the part of the music, the part of the dialogue, where the soloist has the freedom to express themselves in the moment. \item In doing so, they are able to share what Plato might call their wisdom, with the audience.
\end{inparaenum}

\begin{enumerate}

\item Which of the following would be an appropriate title for the passage?

\begin{enumerate}[label=(\Alph*)]
\item How Jazz Musicians Learn to Improvise
\item Thoughts on Effective Discourse
\item Recollections from Philosophy Club
\item The Effects of Plato on Today's Society
\item How Jazz Music Affects Its Audience
\end{enumerate}

\item Which of the following revisions is most needed in sentence 2 (reproduced below)? 

\textit{Every Friday when he hosts philosophy club, I would look at that sign and before I raised my hand to give an opinion on that week's topic, I would decide if the comment was worth sharing.}
 
\begin{enumerate}[label=(\Alph*)]
\item Insert ``Seemingly'' at the beginning
\item Delete ``if the comment was worth sharing'' at the end of the sentence
\item Replace the pronoun ``he'' with its antecedent 
\item Change ``hosts'' to ``hosted''
\item Insert a phrase describing the sign
\end{enumerate}

\item Paragraph two might by improved by the addition of 
\begin{enumerate}[label=(\Alph*)]
\item a transition between paragraphs one and two at the beginning of the paragraph
\item a quote from a jazz musician about how he learned to improvise after sentence 6
\item a discussion of other topics that the author learned in the jazz class
\item an example of a jazz musician that was greatly interested in philosophy 
\item the author's thoughts about whether or not Plato would approve of jazz music at the end of the paragraph
\end{enumerate}
\end{enumerate}