\section{SAT Worksheet: Paragraph Improvement Practice}

\textit{Directions: Reading the following essay and complete the questions that follow:}

\bigskip
\indent (1) For Korean military brides during the Korean War, Korean food was part of their iden-
tities. (2) Korean food wasnt just food, the fact being that it was a symbol of home. (3) Even
when the military brides ate Korean food in America, it wasnt the same as eating food in Korea.
(4) The immigrants had to find ways of working Korean cooking into an American lifestyle.


\indent (5) Later, Korean food became more available. (6) Military brides were forced to make
American cooking their priority. (7) The struggle to maintain their identity was a daily one.
(8) Korean immigrant wives had to deal with the suppression of their ethnic identities from
their American counterparts. (9) Uninterested in preserving the cultural identity of their wives,
American food was insisted on by the husbands.


\indent (10) The secondary status of Korean food, according to what we know of the past, signified
the low social position of Korean culture in relation to American culture. (11) The search for food
represented a longing for home and also it served as a reminder of social inferiority.

\begin{enumerate}
\item{Which of the following is the best revision of the underlined portion of sentence 2 (reproduced below)?}

\textit{Korean food wasn’t just \underline{food, the fact being that it was a symbol of home.}}

\begin{enumerate}[label=(\Alph*)]
\item Korean food wasn’t just food, moreover, it was a symbol of home.
\item Being a symbol of home, Korean food wasn’t just food.
\item Korean food wasn’t just food; it was a symbol of home.
\item The fact that it was a symbol of home meant that Korean food wasn’t just food.
\item Being a symbol of home made Korean food more than just food.

\end{enumerate}

\item Where is the best place to insert the following sentence?

\textit{They used substitute ingredients to cook imitations and dined with other military brides.}

\begin{enumerate}[label=(\Alph*)]
\item After sentence 2
\item After sentence 3
\item After sentence 4
\item After sentence 5
\item After sentence 6
\end{enumerate}

\item Which of the following is the best way to revise and combine sentences 5 and 6 (reproduced below)?

\textit{Later, Korean food became more available. Military brides were forced to make American cooking their priority.}

\begin{enumerate}[label=(\Alph*)]
\item Korean food became more available, however, military brides were forced to make American cooking their priority.
\item Although it became more available, military brides were forced to make American cooking their priority.
\item Having Korean food become more available, military brides were forced to make American cooking their priority.
\item Even as Korean food became more available, military brides were forced to make American cooking their priority.
\item Forced to make American cooking their priority, Korean food became more available.
\end{enumerate}

\item Of the following, which is the best way to phrase sentence 9 (reproduced below)?

\textit{Uninterested in preserving the cultural identity of their wives, American food was insisted on by the husbands.}

\begin{enumerate}[label=(\Alph*)]
\item (as it is now)
\item Uninterested in preserving the cultural identity of their wives, the husbands insisted on American food.
\item Because they were uninterested in preserving the cultural identity of their wives, the husbands insisted on American food.
\item Insisting on American food, the husbands are uninterested in preserving the cultural identity of their wives.
\item American food was insisted on by the husbands because they were uninterested in preserving the cultural identity of their wives.
\end{enumerate}

\item Which revision appropriately shortens sentence 10 (reproduced below)?

\textit{The secondary status of Korean food, according to what we know of the past, signified the low social position of Korean culture in relation to American culture.}

\begin{enumerate}[label=(\Alph*)]
\item Delete “of Korean culture”.
\item Delete “in relation to American culture”.
\item Delete “Korean food”.
\item Delete “, according to what we know of the past,”.
\item Delete “the low social position of”.
\end{enumerate}

\item Which of the following is the best revision of sentence 11 (reproduced below)?

\textit{The search for food represented a longing for home and also it served as a reminder of social inferiority.}

\begin{enumerate}[label=(\Alph*)]
\item The search for food represented a longing for home and also a reminder of social inferiority.
\item The search for food not only represented a longing for home but also served as a reminder of social inferiority.
\item Representing a longing for home, it also served as a reminder of social inferiority.
\item The search for food represented a longing for home, it nevertheless served as a reminder of social inferiority.
\item The search for food represented a longing for home and serving as a reminder of social inferiority.

\end{enumerate}

\end{enumerate} 




