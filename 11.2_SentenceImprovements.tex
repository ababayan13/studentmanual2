\section{Identify the Error or Errors in the Original Sentence}
If you can identify the error or errors in the original sentence before looking at the answer choicaes. This can help you identify if there is an error and if so, to determine which changes need to be made in the correct answer choice. 

For example, try to identify the error in the following sentences:

It is extremely advantageous if you can identify the error or errors in the original sentence before looking at the answer choices and then think of possible corrections. These can help you to 1) Eliminate answer choice ``A" as the best sentence and 2) eliminate incorrect answer choices with the same error as the original sentence quickly. 

\bigskip
\textit{Directions: Determine if the sentences below have an error in the underlined region and, if so, circle it. Write the type of error on the first line and a sample correction on the second line. If you don't that there is an error, write ``No error" as the error type and move to the next sentence. The first question has been done for you.}

\begin{enumerate}
\item The Boston Common is \ul{ older than it but still just as well-maintained as Central Park}.

\bigskip
\textbf{Type of error(s):} \ul{unclear pronoun, not concise} \\
\textbf{Sample correction:} \ul{older than but still just as well-maintained as Central Park.}

\bigskip
\item The Central Intelligence Agency \ul{constant collects secrets from sources} located around the world. 

\bigskip
\textbf{Type of error(s):} \hrulefill  \\
\textbf{Sample correction:} \hrulefill

\item \ul{With determination, dilligence, and being willing to study for many hours}, anyone can achieve a high score on the SAT test. 

\bigskip
\textbf{Type of error(s):} \hrulefill  \\
\textbf{Sample correction:} \hrulefill

\bigskip
\item The movie \ul{that featured well-known actors were winning} many awards for acting, directing, producing, and writing. 

\bigskip
\textbf{Type of error(s):} \hrulefill  \\
\textbf{Sample correction:} \hrulefill

\bigskip
\item \ul{Many educators believe that technology of the sort that} helps monitor student progress and deliever feedback to parents could be helpful in increasing test performance. 

\bigskip
\textbf{Type of error(s):} \hrulefill  \\
\textbf{Sample correction:} \hrulefill

\item After waiting an hour for her friend, the woman finally \ul{arrived in the theater donning a red dress.}

\bigskip
\textbf{Type of error(s):} \hrulefill  \\
\textbf{Sample correction:} \hrulefill

\item Overjoyed that he was accepted his first choice college, \ul{Stephen is currently being slightly ridiculous.}

\bigskip
\textbf{Type of error(s):} \hrulefill  \\
\textbf{Sample correction:} \hrulefill

\item \ul{Many people think that Americans take the right to vote for granted, and I think that it is the right of Americans to not exercise their right to vote.}

\bigskip
\textbf{Type of error(s):} \hrulefill  \\
\textbf{Sample correction:} \hrulefill

\item The bank robbers threatened the tellers by waving their guns, one of the criminals held a teller hostage until the police arrived. 

\bigskip
\textbf{Type of error(s):} \hrulefill  \\
\textbf{Sample correction:} \hrulefill

\item After a major political event such as September 11th, the president will address the nation, with his purpose being to inform and comfort the public.

\bigskip
\textbf{Type of error(s):} \hrulefill  \\
\textbf{Sample correction:} \hrulefill

\item Mary's secret, the whereabouts of the items that had been missing for weeks, \ul{were more compelling than Jeff's.}

\bigskip
\textbf{Type of error(s):} \hrulefill  \\
\textbf{Sample correction:} \hrulefill

\end{enumerate}