\section{Strategies for Paired Passages}

Some SAT Reading Comprehension sections will include paired passages, in which there will be two passages on the same topic.There could be two short or two long passages. You will want to use all of the active reading strategies that we have talked about in previous sections. In addition, at the end of the second passage, you will want to ask yourself how the two passages are the same and how they are different. For example, one may be in favor of a topic and the other might be generally in favor of the same topic with some exceptions that they describe in the passage. 

\bigskip
Some relationships between passage 1 and 2 may be characterized by the following:

\begin{itemize}
\item Passage 2 provides evidence that proves the argument made in Passage 1.
\item Passage 2 elaborates on claims made in Passage 1.
\item Passage 2 exposes the flaws in the argument made in Passage 1.
\item Passage 2 provides an exception to the rule established in Passage 1.
\item Passage 2 contradicts the opinion presented in Passage 1.
\end{itemize}

\subsection{Practice with Paired Passages}

\textit{Directions: Read the following paired passages and answer the questions that follow.}

\textbf{Passage 1}

\begin{linenumbers}
\modulolinenumbers[5]
\indent A general impression prevails with the large picture-loving public that a special training is necessary to any proper appreciation of Rembrandt. He is the idol of the connoisseur because of his superb mastery of technique, his miracles of chiaroscuro, his blending of colors. Those who do not understand these matters must, it is supposed, stand quite without the \textbf{pale} of his admirers. Too many people, accepting this as a dictum, take no pains to make the acquaintance of the great Dutch master. It may be that they are repelled at the outset by Rembrandt's indifference to beauty. His pictures lack altogether those superficial qualities which to some are the first requisites of a picture. Weary of the familiar commonplaces of daily life, the popular imagination looks to art for happier scenes and fairer forms. This taste, so completely gratified by Raphael, is at first strangely disappointed by Rembrandt. While Raphael peoples his canvases with beautiful creatures of another realm, Rembrandt draws his material from the common world about us. In place of the fair women and charming children with whom Raphael delights us, he chooses his models from wrinkled old men and beggars. Rembrandt is nevertheless a poet and a visionary in his own way. ``For physical beauty he substitutes moral expression,'' says Fromentin. If in the first glance at his picture we see only a transcript of common life, a second look discovers something in this common life that we have never before seen there. We look again, and we see behind the commonplace exterior the poetry of the inner life. A vision of the ideal hovers just beyond the real. Thus we gain \textbf{refreshment,} not by being lifted out of the world, but by a revelation of the beauty which is in the world. Rembrandt becomes to us henceforth an interpreter of the secrets of humanity. As Raphael has been surnamed ``the divine,'' for the godlike beauty of his creations, so Rembrandt is ``the human,'' for his sympathetic insight into the lives of his fellow men.

\indent Even for those who are slow to catch the higher meaning of Rembrandt's work, there is still much to entertain and interest in his rare story-telling power--a gift which should in some measure compensate for his lack of superficial beauty. His story themes are almost exclusively Biblical, and his style is not less simple and direct than the narrative itself. Every detail counts for something in the development of the dramatic action. Probably no other artist has understood so well the pictorial qualities of patriarchal history. That singular union of poetry and prose, of mysticism and practical common sense, so striking in the Hebrew character, appealed powerfully to Rembrandt's imagination. It was peculiarly well represented in the scenes of angelic visitation. Jacob wrestling with the Angel affords a fine contrast between the strenuous realities of life and the pure white ideal rising majestically beyond. The homely group of Tobit's family is glorified by the light of the radiant angel soaring into heaven from the midst of them.
 
\bigskip
\textbf{Passage 2}

\indent While the world pays respectful tribute to Rembrandt the artist, it has been compelled to wait until comparatively recent years for some small measure of reliable information concerning Rembrandt the man. The sixteenth and seventeenth centuries seem to have been very little concerned with personalities. A man was judged by his work which appealed, if it were good enough, to an ever-increasing circle. There were no newspapers to record his doings and, if he chanced to be an artist, it was nobody's business to set down the details of his life. Sometimes a diarist chanced to pass by and to jot down a little gossip, quite unconscious of the fact that it would serve to stimulate generations yet unborn, but, for the most part, artists who did great work in a retiring fashion and were not honored by courts and princes as Rubens was, passed from the scene of their labors with all the details of their \textbf{sojourn} unrecorded.

\indent Rembrandt was fated to suffer more than mere neglect, for he seems to have been a light-hearted, headstrong, extravagant man, with no capacity for business. He had not even the supreme quality, associated in doggerel with Dutchmen, of giving too little and asking too much. Consequently, when he died poor and enfeebled, in years when his collection of works of fine art had been sold at public auction for a fraction of its value, when his pictures had been seized for debt, and wife, mistress, children, and many friends had passed, little was said about him. It was only when the superlative quality of his art was recognized beyond a small circle of admirers that people began to gather up such fragments of biography as they could find.

\indent Shakespeare has put into Mark Antony's mouth the statement that ``the evil that men do lives after them,'' and this was very much the case with Rembrandt van Ryn. His first biographers seem to have no memory save for his undoubted recklessness, his extravagance, and his debts. They remembered that his pictures fetched very good prices, that his studio was besieged for some years by more sitters than it could accommodate, that he was honored with commissions from the ruling house, and that in short, he had every chance that would have led a good business man to prosperity and an old age removed from stress and strain. These facts seem to have aroused their ire. They have assailed his memory with invective that does not stop short at false statement. They have found in the greatest of all Dutch artists a never-to-do-well who could not take advantage of his opportunities, who had the extravagance of a company promoter, an explosive temper and all the instincts that make for loose living.
\end{linenumbers}

\bigskip
\sloppy From: https://www.gutenberg.org/files/19602/19602-h/19602-h.htm and https://www.gutenberg.org/files/20607/20607-h/20607-h.htm

\begin{enumerate}

\item What is the main idea of passage 2? \hrulefill

\item What is the main purpose of passage 2? \hrulefill

\item What is your evidence for this? \hrulefill

\item What is the relationship between the paragraphs in passage 2? \hrulefill

\item What is the author's attitude towards Rembrandt? \hrulefill

\item What is your evidence for this? \hrulefill

\item ``Sojourn'' most closely means \hrulefill

\item What is your evidence for this? \hrulefill

\item List the similarities between Passage 1 and Passage 2. Be as specific as possible.
\begin{itemize}
\item
\item
\item
\end{itemize} 
 
\item List the differences between Passage 1 and Passage 2. Be as specific as possible.

\begin{itemize}
\item
\item
\item
\item
\item
\end{itemize} 

\item What is the relationship between passage 1 and passage 2? \hrulefill

\item What is your evidence for this? \hrulefill

\item Do you think that the author of passage 1 would agree with the statement in passage 2 that ``They have found in the greatest of all Dutch artists a never-to-do-well who could not take advantage of his opportunities''?

\hrulefill

\item What is your evidence for this? \hrulefill

\end{enumerate}

subsection{Some additional strategies for paired passages}

\begin{itemize}

\item Carefully read any \longline describing or giving information about the two passages.

\item Note that the first group of questions refers to the \longline; the second group of questions refers to the \longline; and the last group of questions refers to both passages as they relate to each other. Therefore, consider reading the first passage, then answering the \longline, and then reading the \longline and answering the questions about \longline and finally, answering the remaining questions.

\item Be aware that the first question can (and sometimes does) ask for the primary purpose of both passages.

\item In conclusion: Be aware of how the passages are alike and different. As you are reading each passage, looking for the main point, structure, and \longline of each passage. Then, determine the \longline between the two passages. 
\end{itemize}