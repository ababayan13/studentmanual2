\chapter{Answer Key}

\textbf{\ref{2.3}}

\begin{enumerate}[label=ex\arabic*)]
\item 11
\item E
\end{enumerate}

\textbf{\ref{2.4}}

\begin{enumerate}[label=\arabic*)]
\item C
\item D
\item B
\item E
\item 17
\item 15
\end{enumerate}

%%%%%%%%%% Chapter 10 %%%%%%%%%%%%%%%%

\section{Chapter 10: Strategies for Sentence Improvements and Sentence Errors}

\subsection{SAT Worksheet: Warm-Up}

\textbf{Note: Answers ordering the difficulty of the question types will differ. Most students find improving paragraphs the most difficult}

\bigskip
\begin{enumerate}

\item Answer: C But also advice of what and how to get to various destinations. Explanation: We must maintain parallel structure in the sentence. ``...not just (nouns) but also (we also want a noun here rather than the verb ``tells'').''

\item Answer: E No error

\end{enumerate}

\subsection{About the SAT Writing Section}

Fill in the following words (in bold) in the section. 

Your score on the SAT writing section is dependent upon your performance in three sections,
two multiple choice sections and one essay section. The essay section is the \textbf{first section} on
the SAT and consists of essay assignment. You will have \textbf{25} minutes
to complete the essay. We will focus on the essay in a later chapter.
In this section of the SAT manual, we will concentrate on the multiple choice sections.
There are two multiple choice sections, one with \textbf{35} questions to be completed in \textbf{25}
minutes and one with \textbf{14} questions to be completed in \textbf{10}
minutes.

\subsection{Types of Writing Multiple Choice Questions}

Fill in the following words (in bold) in the section.

\begin{enumerate}
\item The first type of writing multiple choice questions is \textbf{sentence improvements}
\begin{itemize}
\item In this type of question, one part of a sentence will be underlined and you will be asked to pick the version of the underlined part of the sentence.
\item This type of question is found in both the 25-minute and the 10-minute multiple choice sections.
\end{itemize}

\item The second type of writing multiple choice questions is \textbf{sentence errors}

\begin{itemize}
\item In this type of question, you will be asked to identify whether or not there is an error in the sentence given, and if so, circle the location of the error. You will not be asked to correct the error on the SAT.
\item This type of question is found in the 25-minute multiple choice section only.
\end{itemize} 

\item The last type of writing multiple choice questions is \textbf{essay}

\begin{itemize}
\item Approximately half of the questions are sentence improvement and sentence revision.
\item The other questions are paragraph or essay structure and logic questions.
\item There are a total of 6 paragraph improvement questions on the SAT, all in the 25-minute section.
\end{itemize}

\end{enumerate}

\subsection{Mastering Sentence Improvement and Sentence Error Question}

Review as a class.

\subsection{Verb Tense}

Verbs must agree with their subject in number. Many errors on the SAT writing section is related to subject-verb agreement and verb tense.

Subject-Verb Agreement

\begin{itemize}
\item This is manageable when sentences are straightforward.
\item For example, fill in the following blanks: He \textbf{is} smart. They \textbf{are} smart.
\item To make the questions more difficult, the SAT will separate the subject and the verb with prepositional phrases or descriptions with commas. An SAT question may also put the verb before the subject.
\item For example: Stephen for more than two weeks \textbf{is} happy because of his most recent grades. To solve
this type of question, identify the subject and cross out prepositional phrase. Then, identify the correct verb form.

\bigskip
Looking at it like this indicates that the subject is ``Stephen'' and so ``is'' is correct form of the verb.

\item{For example: The group, consisting of two adults and five children, \textbf{is} camping this weekend. To solve this type of question, identify the subject and cross out description (between the commas). Then, identify the correct verb form.}

\bigskip
The subject is ``the group'' and so ``is'' is correct form of the verb.

\item For example: Running \textbf{is} the girls' favorite sport. To solve this type of question, rearrange the sentence so that the subject comes before the verb. Then, identify the correct verb form.

\bigskip
The sentence is re-arranged as ``The girls' favorite sport \textbf{is} running.``Therefore, the correct verb form is ``is.''
\end{itemize}

\subsection{SAT Worksheet: SAT Writing Multiple Choice Practice with Subject-Verb Agreement}

\begin{enumerate}
\item have
\item is
\item sit
\item has
\item is
\item widen
\end{enumerate}

\bigskip
Using the Correct Verb Tense

\begin{itemize}
\item Note: The following refers to the sentences about Scott living in New York.

\item The first sentence \textbf{uses the past perfect. He lived in New York and then decided to move. It is implied that the decision to move also happened in the past rather than the present.} whereas the second \textbf{uses the present progressive tense. He is currently living in New York and is considering moving but hasn't done so yet.}

\item If I \textbf{were} to win the lottery, then I \textbf{would} travel around the world.
\end{itemize}

\bigskip Sample SAT Practice Questions

\bigskip
\begin{enumerate}
\item B requiring. Explanation: This is a violation of parallel structure and as a result verb tense. The correct answer should be ``required'' (participle) rather than requiring (progressive) because this is a daily condition rather than what is happening at this exact moment. 

\item C pledge to. Explanation: We need a verb tense that matches ``inspired''. Therefore, we should use ``have pledged to'' (the present perfect) rather than ``pledge'' (perfect tense). You may argue that there is not an error because you may hear the original phrasing spoken. This is an example of where spoken English is not always grammatically correct. 
\end{enumerate}

\subsection{Pronouns}

Agreement with the Antecedent

\begin{enumerate}
\item The noun that the pronoun replaces.
\item Kenna. Explanation: ``Her'' refers to ``Kenna the dog''.
\end{enumerate}

Unclear Pronouns

\begin{enumerate}
\item We are not sure who ``her'' refers to-- Emily or Kate.
\item Out of context, we are not sure who ``She'' or ``them'' is.
\item We are not sure if ``them'' refers to the rides or mice because both are plural nouns. If one was singular and the other was plural, the we would know the antecedent. 
\item The amusement park rides are full of mice, therefore we will avoid the rides by going to the movies instead. Here, we eliminate the pronoun ``them.''
\end{enumerate}

Consistent Point of View

\begin{itemize}
\item One. Explanation: later in the sentence, the pronoun ``one'' is also used and the pronouns must match. 
\end{itemize}

Sample SAT Practice Questions

\begin{enumerate}
\item No error. In this sentence, the pronouns clearly refer to the second grader. 

\item A were. From earlier, A note about SAT grammar: ``The conditional (would) is used for hypothetical situations. The basic formula is ``\textit{If \ldots were \ldots would''.}
\end{enumerate}

\subsection{Misplaced Modifiers}

\bigskip SAT Worksheet: SAT Writing Multiple Choice Practice with Modifiers

\begin{enumerate}
\item Correct

\item Sarah thought that she would be able to move her car through the snow, which had come down lightly throughout the morning.

\item We were impressed with the holiday tree that was full of lights.
\end{enumerate}

Practice SAT Questions

\begin{enumerate}
\item E No error. 
\item E The monkeys, dangling from the trees, frightened us when they tried to steal our sunglasses. Explanation: B and E are grammatically correct, but E is closest to the original sentence and more concise.
\end{enumerate}

\subsection{Parallelism}

\begin{enumerate}
\item ``going to parties'' is a clause rather than a noun. Correction: Emily likes soccer, hockey, and parties. 

\item all of the items in the list should be able to start with ``to go to'' and then a noun. Correction: When at college, Emily likes to go to soccer games, football games, and frisbee practice. 

\item The verb tense before ``but'' does not match the verb tense of the verb after ``but''.This is difficult to pick up on because we often hear the incorrect version in speech. Correction: Kelly likes going to the mall, but riding on the mall's elevators scares her.
\end{enumerate}

\bigskip
Practice SAT Questions

\begin{enumerate}
\item C The boy spoke French, Spanish, and Russian fluently. Explanation: A and C are grammatically correct but C is more concise. 

\item C decides. Explanation: This is a parallel structure error because of incorrect verb tense. The past is used in the rest of the sentence, so the present tense verb ``decides'' is incorrect. 

\end{enumerate}

\subsection{Faulty Comparisons}

\bigskip
2. Paul's pet rock is larger than Jim.

\bigskip
\begin{enumerate}
\item Paul's pet rock and Jim's pet rock
\item Paul's pet rock and Jim (the person)
\item Paul's pet rock is larger than Jim's. OR Paul's pet rock is larger than Jim's pet rock.
\end{enumerate}

\bigskip
SAT Practice Questions

\bigskip
\begin{enumerate}
\item D author. Explanation: The sentence as written is comparing J.K. Rowling's books and the new author. 

\item B he. Explanation: Because the cameraman is talking to the celebrity, we can infer that the pronoun ``he'' refers to the celebrity. 
\end{enumerate}

\subsection{Word Choice}

\begin{itemize}
\item idioms. \textbf{The SATs describes idioms as phrases (usually prepositional) that go together.}

\begin{enumerate}
\item to
\item with
\item to
\item to
\item for or to
\item against
\item upon
\item with
\item to
\end{enumerate}

\item commonly confused works. \textbf{Effect is the noun form and affect is the verb form. Watch out for easily confused words, like allusion and illusion or averse and adverse.  Allusion and illusion are called homophones, or words that sound the same but have different spellings and meanings.}

\item incorrect word choice

\bigskip 
\textit{Note that parallel structure is particularly important in these situations!}

\begin{enumerate}
\item\textbf{ and. She was deciding between going to the movies and going to the mall.}

\item \textbf{or. She could either go to the movies or go to the mall.}

\item \textbf{nor. Since her car ran out of gas, she could neither go to the movies nor go to the mall.}

\item \textbf{but also. Since she had the entire weekend free, she could not only go to the movies but also to the mall.}

\item \textbf{rather than. She went to the movies rather than the mall.}
\end{enumerate}
\end{itemize}

\bigskip
Practice SAT Questions
\begin{enumerate}
\item C or. Explanation: The phrase should read both to \ldots and.

\item A effected by. Explanation: The word in the blank is a verb, so it should be ``affected by.''
\end{enumerate}

\subsection{Other Common Errors}

\begin{itemize} 

\item Adjectives describe nouns. Adverbs describe verbs, adjectives, or other adverbs.  Adverbs often end in –ly. \textbf{Replace diligent with diligently because diligently describes a verb, worked.}

\item Comparatives, like better, worse, faster, smaller, compare two things or people.Superlatives, like best, worst, fastest, and smallest, describe one thing as the best of many or compare three or more things or people.\textbf{Replace more with most because it is assumed that there are more than 2 suits in the entire store.}

\item Fragments may contain a noun and a verb but they do not contain one or more independent clauses. Fragments must be corrected to form complete sentences. 

\bigskip
For example, a fragment may not be a sentence because\ldots

\begin{enumerate}
\item It describes something, but there is no subject-verb relationship
\item It may have most of the makings of a sentence but still be missing an important part of a verb string.
\item It may locate something in time and place with a prepositional phrase or a series of such phrases, but it's still lacking a proper subject-verb relationship within an independent clause.
\item It may even have a subject-verb relationship, but it has been subordinated to another idea by a dependent word and so cannot stand by itself.
\end{enumerate}

\textbf{This phrase given is not a sentence. There must be an independent clause added to the phrase or the phrase could be re-worded to make a sentence. For example, ``When Sam was headed to the store, his mother asked him to return home.'' OR Sam was headed to the store.''}

\item Watch out for wordiness and unnecessary repetition. \textbf{Delete the phrase ``rather than happiness'' because the sentence already states that the goal is to be happy.}

\item Try to break up long, wordy sentences that contain lots of clauses with a period or semi-colon. \textbf{After Luke wrote a list of all of the chores that he wanted to accomplish for the week, he set out to complete each one with a renewed sense of purpose and pride. He did not let himself get distracted by the fact that many of his friends thought that he could not accomplish everything he had written in such a short amount of time.}

\item The SAT want you to use active rather than passive voice. (Subject-verb-object rather than object-verb-subject) \textbf{The  author turned in the newspaper article an hour after the deadline.}

\end{itemize}

\subsection{Practice SAT Questions}

\begin{enumerate}
\item A quickly. Explanation: This is an adjective versus adverb error. ``Quick'' describes a verb (walk), so it should be in its adverb form, quickly. 

\item B me. Explanation: This is a pronoun error. The proper phrasing is between you and me. 

\item A as. Explanation: This is a word choice error. The correct phrasing is ``as \ldots adjective \ldots as'' so the word should be ``as'' instead of ``although''. 

\item A demonstrated. Explanation: This is a subject-verb agreement error and a verb tense error. The verb demonstrate must agree with its subject, each, in number. Also, the action took place in the past (``last year''), so the correct verb should be demonstrated. 

\item D a superhero. Explanation: This is a subject-verb agreement error. The object, superhero, must agree with its subject, five students, in number. Five students can not become one superhero, so instead the phrase should read ``become superheroes. 

\item C you.  Explanation: This is a pronoun consistency error. Earlier in the sentence, the pronoun ``one'' is used, so ``you'' should be replaced by ``one''. 

\item D was funny. Explanation: This is a parallel structure error. Previous items in the list, humor and acuity are nouns so the phrase ``was funny'' should be replaced with the noun ``the ability to be funny.'' One may argue that humor and the ability to be funny are the same, so the latter could be deleted. Fortunately, sentence error questions only ask you about the error, not how to fix it. 

\item C was. Explanation: This is a subject-verb agreement issue. The subject, Ray Charles, is singular, so the verb, were, must also be singular. Were is plural and should be replaced with the word ``was''. 
\end{enumerate}



%%%%%%%%% Chapter 11 %%%%%%%%%%%%

\section{Chapter 11: Sentence Improvements}

\subsection{SAT Worksheet: Warm-Up}

\textbf{Sample Answers}

\begin{enumerate}

\item The dog belonging to my neighbors have barked all night; I wish that it would be quiet so we could get some sleep. No error

\bigskip
This is E for agreement, as it has a subject-verb agreement error. The subject is ``dog’’ (singular), so the verb should be singular as well (has instead of had). This is A, awful verb tense. 

\item The party planner did not take into account all of the guest’s dietary restrictions; as a result, many guests were not able to eat during the party. 

\bigskip
No error

\item When the students entered the room, the lights went dim and a zombie, who claimed that he wanted to eat their brains, jumped out of the bureau. 

\bigskip
No error.  

\item After the committee’s decision was announced, Sherry decided that the best course of action would be to look inside to herself and decide if she should file an appeal. 

\bigskip
This is P for prepositions. The correct phrase is ``to look inside OF herself’’ rather than ``inside to’’ herself. 

\item The posters that were hung up by I around campus were promptly removed by campus officials, claiming that I had not received permission to place them around campus. 

\bigskip
This is Y for you, me, and other pronouns. ``I’’ is a subject pronoun but ``hung up by me’’ requires the object pronoun, I. 

\end{enumerate}

\subsection{Identify the Error or Errors in the Original Sentence}

\begin{enumerate}
\item Done in the student manual. 

\item \textbf{Type of error:} incorrect connector

\bigskip 
\textbf{Sample correction:} While most people detest high prices for food items, organic food sells well despite the increased cost. 

\bigskip
\item \textbf{Type of error:} No error

\bigskip 
\textbf{Sample correction:} No error

\bigskip
\item \textbf{Type of error:} parallel structure (incorrect verb tense)

\bigskip 
\textbf{Sample correction:} The movie featured many well-respected actors and won many awards for acting, directing, producing, and writing. 

\bigskip
\item \textbf{Type of error:} Awkward construction, not concise

\bigskip 
\textbf{Sample correction:} Many educators believe that technology that helps monitor student progress and deliver feedback to parents could be helpful in increasing test performance. 

\bigskip
\item \textbf{Type of error:} incorrect preposition

\bigskip 
\textbf{Sample correction:} After waiting an hour for her friend, the woman finally arrived at the theater donning a red dress. 

\bigskip
\item \textbf{Type of error:} informal (being)

\bigskip 
\textbf{Sample correction:} Overjoyed that he was accepted to his first choice college, Stephen is currently acting slightly ridiculous. 

\bigskip
\item \textbf{Type of error:} incorrect connector

\bigskip 
\textbf{Sample correction:} Many people think that Americans take the right to vote for granted, but I think that it is the right of Americans to not exercise their right to vote. 

\bigskip
\item \textbf{Type of error:} run-on sentence (One may also argue that the pronoun ``their'' is unclear. This will be addressed in a future section.)

\bigskip 
\textbf{Sample correction:} The bank robbers threatened the tellers by waving their guns. One of the criminals held a teller hostage until the police arrived. 

\bigskip
\item \textbf{Type of error:} informal and wordy (One may also argue that the use of the future verb tense is awkward in context. We currently do not have enough information to assess this with the given context clues. It will be addressed in a future section.)

\bigskip 
\textbf{Sample correction:} After a major political event such as September 11th, the president will address the nation to inform and comfort the public. 

\bigskip
\item \textbf{Type of error:} incorrect verb tense

\bigskip 
\textbf{Sample correction:} Mary's secret, the whereabouts of the items that had been missing for weeks, was more compelling than Jeff's. 

\end{enumerate}

\subsection{Strategy: Eliminate Incorrect Answers then Find the Most Concise Answer}

\textbf{Note: there is a leftover formatting issue in the instructions of 3. I'm sorry.}

\subsection{11.3: Sentence Improvements}

\begin{enumerate}
\item B older than Central Park but just as well-maintained. 
\item B for food items, organic food sells well despite the increased cost. (A) and (E) are incorrect because they use the incorrect connector ``but''. (C) is grammatically correct but not as concise as (B). (D) contains awkaward construction. 
\item A With determination and diligence, anyone can achieve a high score on the SAT test.(B) is incorrect because it uses the incorrect verb tense. (C) and (D) are incorrect because they change the meaning of the sentence. (E) is incorrect because the modifiers ``determination and diligence'' are far from what they are modifying, the word ``anyone''. 
\item E featured many well-respected actors and won many awards for. This is a parallel structure question. We are looking for the correct answer to contain two verbs with past tense, ``featured'' and ``won''. (A), (B), and (D) are incorrect because the use the incorrect verb tenses. (B) is also incorrect because there is a comma in the middle of the phrase. (C) is correct but not as concise as (E). 
\item B Many educators believe technology that. (A), (D), and (E) are grammatically correct but not as concise as (B). (C) uses ``which'', which is incorrect because which is used introduce clauses with commas or non-restrictive clauses. ``That'' should be used to introduce clauses with no commas or restrictive clauses.
\item E arrived at the theater donning a red dress. (A), (C), and (D) are incorrect because the correct idiom is ``arrived \textbf{at}''. (B) slightly changes the meaning of the original sentence by implying a change of order of events. 
\item C Stephen is acting slightly ridiculous. (A), (B), (D), and (E) are all informal constructions (e.g. using the word ``being'') and therefore not the best answers.
\item D Many people think that Americans take the right to vote for granted, but I think that it is the right of Americans to not exercise their right to vote. (A),  (B), and (C) do not use the best connector. ``But'' is a better connector than ``and'' because it demonstrates a contrast between the first and the second parts of the sentence. (E) is grammatically correct but not the best sentence because the perspective of the second part of the sentence has been changed. Now instead of stating it as my opinion, the sentence displays this as a fact or widespread belief. 
\item B threatened the tellers by waving their guns. One of the criminals held a teller hostage. (A) is incorrect because it contains two sentences without a connector. (C) is grammatically correct but not the best answer because the sentences are not intricately related and therefore a period is a better choice than a semi-colon in this case. \textbf{You may want to tell students on the real SATs they probably won't have to decide between (B) and (C), only one will be present. I meant to have (D) be the correct answer but it isn't due to a typo.} (D) and (E) are incorrect because the verb tense is ``threaten'' not ``threat''.
\item C the president will address the nation to inform and comfort the public. (A) and (D) are informal and therefore not the best choices, as they use the word ``being''. (B) and (E) are grammatically correct but not as concise as (C). 
\item E was more compelling than Jeff's. (A), (D), and (E) violate subject-verb agreement. The subject of the sentence, a secret, is singular, and therefore the verb should be singluar (was rather than were). (C) is grammatically correct but not as concise as (E). 
\end{enumerate}

\subsection{11.4: Sentence Improvements}

\begin{enumerate}
\item E the monkeys that tried to steal our sunglasses frightened us. Explanation: The original sentence contains an error with misplaced modifiers. 
\item A wanted to learn more about the world. No error
\item D mesmerizing. Explanation: The initial phrase is redundant. 
\item B The United Nations considers the decrease of water resources to be one of the major threats to global peace. Explanation: The original phrase contains an error with subject-verb agreement. The subject, ``decrease'' is singular so the verb phrase should be ``is considered''. 
\item D  if I were to have attended the meeting, then I would have a more descriptive answer for you. Explanation: The original sentence contains an error with the structure of a hypothetical sentence. 
\item E Robert was a microbiology graduate student at Pittsburgh University when he discovered that his true passion was science communication. Explanation: The original phrasing is not clear nor concise. 
\item E as easy as. Explanation: The original phrasing is not concise. The correct answer uses the correct phrasing (as \ldots as) and is concise. 
\item B despite the fact that no online schedules had been updated before the flight. Explanation: The correct sentence adds clarity about when the schedules were updated (a.k.a. before what?)
\item C are thought to contribute to complex diseases such as asthma; however, scientists have had trouble identifying. Explanation: The original sentence is a run-on sentence. The correct answer breaks up the sentence into more manageable parts while still maintaining the correct verb tenses. 
\item E Administration, so there are very few that remain in stores. Explanation: The phrase in the original sentence, ``since such is the case'', is awkward and not concise. A better cause and effect connector is ``so''. 
\item C The profits from textbooks and difficulties finding authors are the reasons why. Explanation: The original sentence violates parallel structure.
\item A is having. Explanation: This is an incorrect verb tense. Since it is happening on a daily basis it should be in the present tense, ``has''
\item C them. Explanation: ``Them'' is an unclear pronoun. We do not know if it refers to the archaeological sites or the items. 
\item E No error.
\item C less. Explanation: There are a number of health and sanitation issues. ``Fewer'', rather than ``less'' is used to describe a quantity. 
\item C and. Explanation: This contains an incorrect connector. The sentence is trying to say that you have two options and need to pick one, so it should use the connector ``or'' rather than ``and.''
\item A thanked. Explanation: This sentence violates parallel structure. The first part of the phrase uses the progressive (``was hugging'') therefore so should the next part (``was thanking''). 
\item C prepared at. Explanation: This is an incorrect idiom. The correct phrase is ``prepared for.''
\item D the most. Explanation: This sentence only compares two things, cats and dogs, so we should use the comparative (more) rather than the superlative (the most).
\item E No error. 
\item D provde. Explanation: The correct participle of the word ``prove'' is ``proven'' not ``proved''.
\item D marketing firm in Seattle. Explanation: This is a faulty comparison. The sentence compares the commercials with the firm in Seattle when it is trying to compare the commercials written by the agency in New York and the firm in Seattle.
\item D had arrived. Explanation: Sentences with a specific date should use the past tense rather than the past perfect (arrived instead of had arrived). 
\item C I. Explanation: This sentence contains a pronoun error. The phrase beginning with ``my boss'' is the object rather than the subject of the sentence. Therefore, the object pronoun me should be used rather than the subject pronoun I. In an example like this, eliminate ``my boss" and check if the sentence still makes sense. ``Given by I" is not correct but ``given by me'' is correct. 
\item A swam. Explanation: This is a verb tense error. The participle of the word ``swam'' is swum. 
\item C efficient to. Explanation: This is an adjectives vs. adverbs error. The word ``efficient'' describes the verb work, so it should be ``efficiently''. 
\item A that is. Explanation: This is a which vs. that error. ``Which'' is correct because which is used introduce clauses with commas or non-restrictive clauses. ``That'' is incorrect in this case because it is used to introduce clauses with no commas or restrictive clauses.
\item C you. Explanation: This is pronouns error. Earlier in the sentence, the pronoun ``one'' is used, so ``one'' should be used rather than ``you''.
\item B to be. Explanation: The correct phrase is ``is regarded as''. 
\end{enumerate}

\subsection{11.5: Correcting Sentence Error and Improvement Questions}

Answers will differ based on which problems the students got wrong.  

Note: Ignore page 132.



%%%%%%%%%%%%%% Chapter 12 %%%%%%%%%%%%%%%%


\section{Chapter 12: Four Strategies to Beat Paragraph Improvements}

\subsection{SAT Worksheet: Warm-Up}

\begin{enumerate}
\item 1
\item Approximately 11 sentence improvement questions, approximately 17 sentence error questions, and 6 paragraph improvement questions for a total of 35 questions
\item 25 minutes
\item 6
\item Sample answers: skimming the paragraphs, marking obvious errors in the paragraph, creating your own answer choice to the question before you look at the given answer choices. These and other strategies will be discussed in this chapter. 
\end{enumerate}

\subsection{Skim the Passage: Practice this strategy}

\begin{enumerate}
\item Implementation and limitations of the preference based approach, particularly how it is used at the the Canine Cognition Center.
\item Describing the preference based approach 
\item How the preference based approach is implemented at the Canine Cognition Center
\item Limitations of the preference based approach when used on dogs
\end{enumerate}

\subsection{Determine the Type of Question You're Being Asked: Practice this Strategy}

\begin{enumerate}
\item Combination/insertion
\item Grammar Revision
\item Grammar Revision
\item Grammar Revision
\item Paragraph Relationship
\end{enumerate}

\subsection{}

\subsection{Develop Your Own Answer Before Looking at the Answer Choices}

\begin{enumerate}
\item After sentence 3. Explanation: This adds the transition (``for example'') between the broad description of the preference based approach and specific example of the minks.
\item As a ``proof of concept'', the cortisol levels (used as a measure of stress) of the minks with access to the pool were significantly lower after swimming than beforehand. Explanation: This revision explicitly states the contrast of the cortisol levels before swimming and after swimming. 
\item Because domestic dogs like to be near humans, Hare contends the results obtained with this experimental setup are more accurate than if the dogs were to be raised in animal storage facilities. Explanation: Adds clarification to what ``results'' Hare is talking about. 
\item Other scientists have acknowledged sources of bias in these experiments. Explanation: The original sentence contains passive voice.
\item Despite limitations, the preference based approach is a useful tool when performing animal research. Explanation: We need a sentence that concludes the passage by tying the main points from each paragraph together. 
\end{enumerate}

\subsection{SAT Worksheet: Paragraph Improvement Practice}


\textbf{Passage 1 (about Duke Canine Cognition Center):}

\begin{enumerate}
\item Answer: B After sentence 3. Explanation: The original paragraph changes from a broad description of the preference based approach (before sentence 4) to the specific example of the cortisol levels of the mink (after sentence 4 in the same paragraph). Therefore, this part could be improved by adding a transition sentence. 
\item Answer: E As a ``proof of concept'', the mink's cortisol levels (used as a measure of stress) were significantly lower after swimming than before. Explanation: Unlike (A), (B), and (D), this revision explicitly states the contrast of the cortisol levels before swimming and after swimming. (E) is more concise than (C). 
\item Answer: D Because domestic dogs like to be near humans, Hare contends that the results collected with this in mind are more accurate than if the dogs were to be raised in animal storage facilities and then tested. Explanation: This answer choice adds clarification to what ``results'' Hare is talking about and is more concise (B).
\item Answer: E Delete ``by other scientists''. Explanation: The limitations have been acknowledged by other scientists implicitly. (A) is incorrect because the transition ``however'' is used to contrast the information in other paragraphs to this one. (B) and (D) are incorrect because the resultant phrase would not be a complete sentence. (C) is incorrect because it clarifies where the sources of bias are located.
\item Answer: A Despite some limitations of the preference based approach, it appears to be a good
alternative to animal storage and testing typically opposed to by animal advocates. Explanation: We need a sentence that concludes the passage by tying the main points from each paragraph together. (B) is incorrect because it is too extreme. (C) and (D) are unsupported by the information in the paragraph. (E) adds more information rather than act as a conclusion
\end{enumerate}

\textbf{Passage 2 (about familial searching):}

\begin{enumerate}
\item E For instance. Explanation: The original paragraph changes from a broad description of familial searching to the specific example of it being used in the Grim Sleeper case. Therefore, this part could be improved by adding a transition that signals that there is about to be an example, which (A)-(D) do not contain. 
\item C Change ``They'' at the beginning of the sentence to ``Law enforcement''. Explanation: This answer choice clarifies the pronoun ``they''. (A) uses an improper transition. (B) is grammatically correct but decreases the clarity of the sentence. (D) is not grammatically correct. (E) is an improper use of the past perfect. We do not want to use it because the sentence describes one event that happened in the past rather than something that happened continuously in the past. 
\item D As a result, some people support the use of familial searching in severe crimes including murder since they believe that the safety of the victim and other innocent civilians should be the highest priority. Explanation: (C) changes the original meaning of the sentence. (A), (B), and (E) are grammatically correct but not as concise as (D).
\item E In this group, there are divisions as to whether familial searching should be used in less dangerous crimes like misdemeanors. Some argue that familial searching should not be used in these crimes because the criminals are not threats to society.Explanation: (A), (C), and (D) are grammatically incorrect. (B) is a run-on sentence. 
\item A As of 2010, four states use familial searching. Explanation: This sentence is not directly related to the other sentences in the paragraph which discuss individual and societal views on familial searching but not state-wide views. Therefore, this is a poor topic sentence. 
\end{enumerate}

\textbf{Passage 3 (about selecting a university):}

\bigskip
\textbf{Note: Question 4 should read ``inserted before sentence 18''}

\begin{enumerate}
\item B People should select a college based on student's interests and preferences. Explanation: The other answer choices are incorrect mostly due to the fact that they only include information from one paragraph rather than focusing on the passage as a whole. (A) and (D) too extreme and not stated in the passage. (C) is only discussed in the last paragraph. (E) is only discussed in paragraph 1. 
\item D As a result, it can be tempting to apply to colleges with ``big names'', famous alumni, or ``good'' reputations according to friends or the media. Explanation: This sentence uses parallel structure correctly and is the most concise. 
\item D While a high school student may not know exactly what they want to major in during college, answering these questions can help student to nd schools that match their interests. For example, some schools are ``liberal arts'' whereas others are more focused in engineering, business, or research. Explanation: (A) and (C) are run-on sentences. (B) is grammatically correct but not as concise as the correct answer choice. (E)'s statement that ``These are liberal arts'' is not clear.
\item A Parents and students should also discuss if they will pay for college out-of-pocket or with merit-based scholarships, financial aid loans, or grants. Explanation: We can improve the paragraph by introducing and giving more information about ``these the financial facts'' stated in sentence 18. (B) and (C) provide more information about sentence 17 but does not describe ``these financial facts''. (D) is too broad and the information in (E) is not connected to either sentence 17 or 18.
\item A It is important to choose a college or college program that is aligned with a student's personal academic and social goals. Explanation:  We need a sentence that concludes the passage by tying the main points presented in the entire passage. (B) focuses on information found only in the last paragraph. While you may be tempted to use outside knowledge and pick (C) and (D) because they are true, these choices are incorrect because they introduce more information that is not well supported by another other paragraph. (E) contains information found only in the first paragraph.


%%%%%%%%%%%%%%%%%%%%%%%%%%%%%
% Chapter 13
%%%%%%%%%%%%%%%%%%%%%%%%%%%%%

\section{Six Strategies for a Perfect Six Essay and Practice}

\subsection{SAT Worksheet: Warm-Up}

\begin{enumerate}
\item (B) Thoughts on Effective Discourse. Explanation: (A) and (E) are incorrect because it only focuses on information from the last paragraph. (C) and (D) are incorrect because it only focuses on information from the 1st paragraph.
\item (D) Change ``hosts’’ to ``hosted.’’ Explanation: The first paragraph is describing the events in the past. Therefore, the verb should be in the past tense. 
\item (A) a transition between paragraphs one and two at the beginning of the paragraph. Explanation: Upon first reading of the passage, it is difficult to tell how the ideas of paragraphs 1 and 2 connect. As a result, it would be effective to state this explicitly. Of all the answer choices, this is the most important change because it will increase the cohesion of the essay.
\end{enumerate}

\subsection{Strategy 1: Know the Right Structure}

\begin{enumerate}
\item \textbf{Paragraph 1:} The introduction

\begin{itemize}
\item 1-2 sentences of hook (getting your audience interested in the topic) and general introduction to the topic
\item 1 sentence of transition to your thesis. You are setting up your argument here.
\item 1 sentence for the thesis
\end{itemize}

\item \textbf{Paragraph 2:} Body Paragraph 1
\begin{itemize}
\item The first sentence is a transition from the thesis to your first example
\item One sentence to describe your first example
\item 2-3 sentences of analysis. Here, you want to describe why your example supports your thesis. 
\end{itemize}

\item \textbf{Paragraph 3:} Body Paragraph 2

\begin{itemize}
\item Transition to the next paragraph
\item One sentence to describe your second example
\item 2-3 sentences of analysis. Here, you want to describe why your example supports your thesis. 
\end{itemize}

\item \textbf{Paragraph 4:} Body Paragraph 3

\begin{itemize}
\item Transition to the next paragraph
\item One sentence to describe your third example
\item 2-3 sentences of analysis. Here, you want to describe why your example supports your thesis. 
\end{itemize}

\item \textbf{Paragraph 5:} The conclusion 

\begin{itemize} 
\item transition to the next paragraph and re-state your thesis
\item one sentence to transition from your thesis to your analysis (there will be more on this type of analysis later in the chapter)
\item 2-3 sentences of analysis in which you discuss the widespread implications for the argument presented in the passage
\end{itemize}
\end{enumerate}

\subsection{Strategy 2: Present a Clear Thesis}

\begin{itemize}
\item Clearly state your position on the issue (e.g. do you agree or disagree with the statement presented in the assignment)
\item Lists your examples
\item Is grammatically correct
\end{itemize}

\subsection{SAT Worksheet: Practice Writing Thesis Statements}

\textbf{I used ``examples 1, 2, and 3'', rather than specific examples because students will brainstorm examples in the next section.}

\begin{enumerate}
\item Prompt 1

\begin{itemize}
\item According to my example 1, 2, and 3, it is better to accept and be happy with what you have rather than to seek additional sources of happiness. 

\item It is essential to always seek greater sources of happiness, according to example 1, 2, and 3. 
\end{itemize}

\item Prompt 2

\textbf{Note: the Assignment should actually read ``In extolling the virtues of our freedom in schools, do we solidify freedom as a value to be protected \textit{or} do we teach children to be complacent about freedoms lost?'''}

\begin{itemize}
\item In extolling the virtues of freedom in American schools, we solidify freedom as a value to be protected, as demonstrated by example 1, example 2, and example 3. 

\item In attempting to solidify freedom as a value to be protected, examples 1, 2, and 3 demonstrates that we actually teach children to be complacent about freedoms lost. 

\end{itemize}

\item Prompt 3
\begin{itemize} 
\item The examples of blank, blank, and blank illustrate that it is better to fill our days with novel experiences rather than to slow down to appreciate the present. 

\item It is better to take time to appreciate our present moments than to fill our days with novel adventures. The former viewpoint is supported by example 1, example 2, and example 3. 

\end{itemize}

\subsection{Use 1 Specific Example in Each Body Paragraph}

The SAT wants you to use examples from your reading, studies, experience, or observations. The SATs want students to focus on one example per body paragraph. It should be introduced in \textbf{the introduction paragraph} and explained in a general sense in \textbf{a body paragraph}. The rest of the paragraph should focus on explaining how this example supports your thesis.

\begin{itemize}
\item Examples of readings you’ve done: \textit{Romeo and Juliet, Lord of the Flies, The World is Flat}
\item Examples of topics you’ve studied: Historical events such as the Revolutionary War, the Civil War, the World Wars; Political events such as elections; Time periods such as the Middle Ages, the Renaissance, Transcendentalism, etc.
\end{itemize}

\end{enumerate}

\subsection{SAT Worksheet: Practice Writing Specific Examples}
\begin{enumerate}
\item Prompt 1

\begin{itemize}
\item Thesis: According to the novel ``Lean In''', Aesop’s fable ``The Dog and His Reflection'', and statistics on previous Massachusetts state lottery winners demonstrate that it is better to accept and be happy with what you have rather than to seek additional sources of happiness. 
\item Example 1: The book ``Lean In'' by Sheryl Sandberg describes the importance of finding happiness through a career and family life. 
\item Describe how your example supports your thesis: While Sandberg encourages her readers to ``lean in'' to what they are passionate about, she also describes the importance of gratitude and contentment. 

\item Example 2: Aesop’s fable, ``The Dog and His Reflection'', is about a dog carrying a bone that sees his reflection and, mistaking it for another dog with a bone, drops his bone to grab what he thinks is another bone. 
\item Describe how your example supports your thesis: In dropping his original bone into the water, the dog ends up with zero bones. Had he been content with what he had originally, he would still have a bone. 

\item Example 3: Statistics on lottery winners demonstrate that many lottery winners end up bankrupt and unhappy rather than financially stable and well off. 
\item Describe how your example supports your thesis: In seeking something that they thought would bring them happiness rather than being content with their current living situation, stories of lottery winners suggest that many people aren’t happy despite being put in a situation that they thought would make them happy. 
\end{itemize}

\item Prompt 2

\begin{itemize}
\item Thesis: In extolling the virtues of freedom in American schools, we solidify freedom as a value to be protected, as demonstrated by our teaching of World War II, the Suffrage Movement, and the story of Malala Yousafzai. 
\item Example 1: World War II, fought during the mid-20th century, involved axis and allied powers fighting mostly in Europe and Asia. 
Describe how your example supports your thesis: In describing atrocities of the axis powers, particularly in the invaded nations of Poland, Germany, and many others, and the World War that resulted, children learn that it is important to fight to protect the freedoms of others. 

\item Example 2: Women’s right to vote in America in the 1910s represented a dramatic change for the status of women in this country.

\item Describe how your example supports your thesis: In teaching students that certain groups like women did not always have civil rights, students learn that freedom is something to be cherished rather than taken for granted. 

\item Example 3: In the novel, ``I am Malala'', the author, Malala Yousafzai describes her advocacy work for girl’s education and how she was shot for the Taliban as a result of her advocacy. 
\item Describe how your example supports your thesis: Malala serves as a role model for many schoolchildren and empowers youth to not only value but also to advocate for their right to education. 
\subsection{SAT Worksheets: Practice Writing Conclusions}

\end{itemize}
\end{enumerate}

\begin{enumerate}

\item Prompt 1

\bigskip
Literary examples as well as the stories of lottery winners, suggests that we should be content with what we have, rather than constantly trying to find new sources of happiness. It is important that people consider this ``grass is always greener on the other side'' syndrome when weighing decisions. For example, while a new opportunity may seem like the solution to all problems, it oftentimes leaves something to be desired. This demonstrates the importance of gathering information and weighing the uncertain and unknown versus one’s current position when making a decision. 

\item Prompt 2

\bigskip
In describing events from the past and present that demonstrate the struggle for freedom, students in school lean that freedom is a value to be protected. This argument demonstrates a two-pronged approach for maintaining the freedoms that we enjoy today: an understanding of the violations of freedom in the past and today as well as participation in the process of monitoring government so that we can continue to enjoy these freedoms. While one may argue that this is the role of certain bodies or branches such as the Supreme Court, in learning about freedom, we each have an individual responsibility to protect it. 

\end{enumerate} 

\subsection{Strategy 5: Use Transitions Between Paragraphs}

\begin{enumerate}
\item Examples of transitions to be used between the introduction and first body paragraph: for example, for instance, namely, specifically, to illustrate

\item Examples of transitions to be used between body paragraphs: additionally, again, also, and, as well, besides, equally important, further, furthermore, in addition, moreover, then

\item Examples of transitions to be used between the last body paragraph and the conclusion: finally, in a word, in brief, briefly, in conclusion, in the end, in the final analysis, on the whole, thus, to conclude, to summarize, in sum, to sum up, in summary

\item Examples of transitions to be used between sentences to show similarity: also, in the same way, just as \ldots so too, likewise, similarly

\item Examples of transitions to be used between sentences to show contrast: but, however, in spite of, on the one hand \ldots on the other hand, nevertheless, nonetheless, notwithstanding, in contrast, on the contrary, still, yet

\item Examples of transitions to be used between sentences to show cause and effect: accordingly, consequently, hence, so, therefore, thus

\end{enumerate}

\subsection{SAT Worksheet: Correct the following sentences so that they each use formal language and are free of grammatical mistakes}

\begin{enumerate}
\item Sentence 1

\begin{itemize}
\item Errors: verb tense, informal language
\item Fix It: We initially thought that we would have to bring many items to the picnic, but we only brought a few things. 
\end{itemize}

\item Sentence 2

\begin{itemize}
\item Errors: verb tense- the verbs in this sentence should have the same tense
\item Fix It: Although she was at school in 1931, it is clear from context that she was unhappy. 
\end{itemize}

\item Sentence 3

\begin{itemize}
\item Errors: No major errors, perhaps not as concise as it could be
\item Fix It: This argument is important, as it suggests that we should always do our best on any assignment and conduct ourselves with integrity. 
\end{itemize}

\item Sentence 4

\begin{itemize}
\item Errors: Uses cliche
\item Fix It: It was clear that he was trying to accomplish two things at once. 
\end{itemize}

\item Sentence 5

\begin{itemize}
\item Errors: Informal language
\item Fix It: Paul wasn’t sure what he was going to see when he entered the party. 
\end{itemize}

\end{enumerate}

\subsection{SAT Essay Practice}

\begin{enumerate}
\item Sample Essay 1

Possible theses: We should judge our society based on the opportunities it provides as members, as demonstrated by the rise of Facebook and other members of the American technology sector, radio personality and best-selling author Dave Ramsey, and the Constitution. 

\bigskip
Examples from the Occupy Wall Street movement, the show ``the Wire'', and the writings of Mark Twain, illuminate the struggles of the poor and demonstrate that we should judge our society based on the conditions of those who achieve the least rather than the opportunities that it has provided for others. 

\item Sample Essay 2

Possible theses: The leadership qualities of George Washington, Martin Luther King Jr., and Jeffrey Skilling (the CEO of Enron) indicate that these are more important to the effectiveness of a group than the characteristics of the member of the group. 

\bigskip
A group—such as Wikipedia contributors, those involved with Habitat for Humanity, and a classroom of students— is needed to implement the ideas of their leaders, indicating that the characteristics of the group are more important to group effectiveness than the qualities of their leaders. 

\end{enumerate}


%%%%%%%%%%%%%%%%%%%%%%%%%%%%%
% Chapter 14
%%%%%%%%%%%%%%%%%%%%%%%%%%%%%

\section{Chapter 14}
\subsection{SAT Warm-Up}

\textbf{Sample Theses}
\begin{itemize} 
\item The examples of Frederick Douglass, Don Quixote, and Steve Job's convocation speech at Stanford, illustrate that it is better to fill our days with novel experiences rather than to slow down to appreciate the present. 

\item It is better to take time to appreciate our present moments than to fill our days with novel adventures. The former viewpoint is supported by writings from the Transcendentalists such as Ralph Waldo Emerson, remarks from the Dalai Lama, and the song ``Let Her Go'' by Passenger (Michael David Rosenberg). 

\end{itemize}

\subsection{Words to Know}

\begin{itemize}

\item Write the words and definitions on flashcards

\item Write the words in sentences that demonstrate its context

\item Write a story that incorporates the words 

\item Change the lyrics of your favorite songs to those that include vocabulary words

\item Learn the root or derivations of the words

\item Learning words grouped by definition

\end{itemize}

\subsection{SAT Worksheet: Practice with Grouping Vocabulary Words}

\begin{enumerate}

\item Words that mean ``reserved''?: reticent, taciturn, detached, introverted, reclusive, coy, unforthcoming

\item Words that mean ``careful'' or ``critical''?: discriminating, exacting, hairsplitting, particular, acute, censorious

\item Words that mean ``puzzling'' or ``mysterious''?: enigmatic, ambiguous, cryptic, obscure, eqivocal, indecipherable, perplexing, unintelligible, inscrutable

\item Words that mean ``bitter'' or ``sharp''?: caustic, acerbic, acrid, astringent, biting, vitriolic, piquant

\item Words that mean ``highly productive''?: bountiful, profusive, proliferant, fruitful, rich, profitable

\item Words that mean ``generous'' or ``noble?'': magnanimous, altruistic, selfless, beneficent, benevolent, munificent, philanthropic, humanitarian

\item Words that mean ``peaceful'' or ``committed to peace''?: amicable, harmonious, placid, tranquil, halcyon, pacifistic, placatory

\item Words that mean ``disapprove''?: decry, denoune, deplore, censure, chastise, deprecate, dismiss, remonstrate, reprove

\end{enumerate}

\subsection{SAT Worksheet: Practice Determining SAT Words using Prefixes, Suffixes, and Roots}

\begin{enumerate}

\item No because it might hurt your throat.

\item Finding one's way properly

\item Patriarchal because there has not been a female president

\item If faced with one's enemies

\item to take life, to begin

\item believe. CNN, BBC, etc.

\item To speak praise

\item Pulls them out of their current situation (e.g. kidnap)

\item Communicating across the country

\item No because they will probably lie to you

\item Puts their trust in you, tells you something important

\item By increasing the number of questions that you get correct on an SAT section

\item At a town hall or place of worship

\item sounds the same (e.g. right and write)

\item To insert into someone else to harm them

\item The class clown

\item Bill Gates

\item See if it stays in the same place or moves locations over a year

\item Someone that has recently had a baby

\item Oligarchy because the few people with wealth have more power than others

\item Cooked carrots- I can't stand to eat them!

\item To think about something as to put it in its proper place

\item strict

\item extremely sacred

\item Into 8 even slices

\item To sit (be placed) below

\end{enumerate}

\subsection{Connotations}

Should read: that means that you are looking for an answer choice with a negative connotation and can eliminate all of the words that have a positive connotation

\subsection{SAT Worksheet: Determine the connotation of unfamiliar words}

\begin{itemize}

\item negative

\item negative

\item negative

\item positive

\item neutral

\item negative

\item negative

\item negative

\item negative

\item negative

\item positive

\item positive

\item negative

\item positive

\item negative or positive

\item positive

\item negative

\item negative

\item positive

\item positive

\item negative

\item negative

\item positive

\item negative

\item positive

\item negative or positive

\item positive

\end{itemize}

\subsection{SAT Worksheet: Practice with Unfamiliar Words}

\begin{enumerate}

\item Billowing- ballooning, bouncing. Labyrinth- winding. Credible- believable. Abrogate- to cancel or abolish. Tangible- able to be touched.  

\item Reprehensible- shameful. Penguin- an animal. Fabricated- created from falsehood. Rancorous- resentful. Enigmatic- mysterious. 

\item Consecrations- devotion. Enigmas- mysteries. Fabrications- lies. Accolades- praise, awards. Amalgamations- mergers, consolidations.

\item Acquiesced- to agree to with some reluctance. Mimicked- copied. Consecrated- hallowed. Curtailed- to cut short. Plummeted- sharp decrease

\end{enumerate}

\subsection{SAT Worksheet: Sentence Completion Strategies Practice}

\begin{itemize}

\item truthful

\item heinous

\item awards

\item gave in

\item refrain

\item disgusting smelling

\item hide \ldots lot. \textbf{delete the word the}

\item naive \ldots story

\item interesting \ldots negative.\textbf{is should be replaced with it} 

\end{itemize}

\subsection{SAT Worksheet: Practice with 1-Blank Sentence Completion Questions}

\begin{enumerate}

\item (C) credible

\item (A) reprehensible

\item (D) accolades

\item (A) acquiesced

\item (C) abstain

\item (A) noxious

\end{enumerate}

\subsection{SAT Worksheet: Practice with 2-Blank Sentence Completion Questions}

\bigskip
\textbf{delete the word the}

\bigskip
(D) obfuscate \ldots plethora

\begin{enumerate}

\item (D) credulous \ldots drivel

\item (A) tentative \ldots detrimental

\end{enumerate}

\subsection{Vocabulary-in-Context Practice for Reading Comprehension}

\begin{enumerate}

\item onslaught- attack.``Nothing but a narrow stretch of the sea kept the Moslems'' from attacking

\item fire- to ignite, to inspire. We want a word for making people imagine

\item scaffold- a temporary structure to stand on outside of a building. We want a description of the place that the bishops are standing

\item \textbf{should read insolent}- Insolent- looking down upon. It is said that she ``marched in'' so we need a word that matches this tone which suggests that the girl thinks herself important. 

\item indulgent- giving helping. Her mother is sick and therefore needs help.

\end{enumerate}

\vfill

\end{enumerate}