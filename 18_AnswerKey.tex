\chapter{Answer Key}

\textbf{\ref{2.3}}

\begin{enumerate}[label=ex\arabic*)]
\item 11
\item E
\end{enumerate}

\textbf{\ref{2.4}}

\begin{enumerate}[label=\arabic*)]
\item C
\item D
\item B
\item E
\item 17
\item 15
\end{enumerate}


%%%%%%%%%% Chapter 11 %%%%%%%%%%%%%%%%

\section{Chapter 11 SAT Writing Multiple Choice Part III}


%%%%%%%%%% Chapter 11 %%%%%%%%%%%%%%%%

\subsection{SAT Worksheet 1C: Warm-up}

\begin{enumerate}
\item \textbf{B}. Explanation:  This sentence can be improved by changing the gerund `'reporting'' to its noun form, `'the report.''
\item \textbf{C}. Explanation:  Wordiness.  Improve this sentence by simplifying `'for it will encourage'' to simply ``to encourage.''
\item \textbf{B}. Explanation:  Shift in point of view.  The subject needs to stay the same in both clauses.  It is incorrect to switch from second person ``you'' to first person plural ``our.''  Changing ``you'' to ``we'' in the first clause fixes this problem.
\end{enumerate}


%%%%%%%%%% Chapter 11 %%%%%%%%%%%%%%%%

\subsection{Strategy for Paragraph improvements}

\begin{enumerate}
\item Quickly read over the paragraphs and take note of the \textbf{main idea of the essay}. \textbf{Mark} any grammatical errors that you see right away because there will probably be a question about that.
\item Then look at the \textbf{first question}.  
\item Reread the sentence in context. Read the sentence \textbf{before} and the sentence \textbf{after} the line in the question. 

If it is a sentence improvement question, look for the ``alerts" and other errors in the Writing Multiple Choice Part I and Part II. 

If it is a sentence addition or move question, ask yourself, ``does it fit where it is, or would it be better somewhere else?" Remember: The sentences need to be coherent. Frequently, the SAT tests this by making the first part of the sentence similar to the last part of the previous sentence (that is, discussing the same topic as the second half of the previous sentence). 

If it doesn't discuss the topic(s) presented in the sentences around it or otherwise seems random, it may need to be removed completely. 

If you think that it should stay where it is, you can re-visit the option of fixing it.  If it's a run-on, try shortening it or splitting it into two sentences.  This tests the same skills as Sentence Improvement questions.  Before you “improve” the sentence though, make sure it belongs where it is!  Always consider sentences in context.

\item Think of \textbf{your own answer choice}.
\item Look at the answer choices.  \textbf{Select} the one that comes closest to your answer.
\end{enumerate}


%%%%%%%%%% Chapter 11 %%%%%%%%%%%%%%%%

\subsection{SAT Worksheet 2C: Exercises for Improving Paragraphs}

\begin{enumerate}
\item \textbf{Exercise 1: Fragments}

Is there Life on Mars?

The basis of this belief is that if, as we saw, all the globes in our solar system are masses of metal that are cooling down, the smaller will have cooled down before the larger, and will be further ahead in their development. Now Mars is very much smaller than the earth, and must have cooled at its surface millions of years before the earth did. Hence, if a story of life began on Mars at all. \textbf{Sample correction: Hence, if a story of life began on Mars at all, it began long before the story of life on the earth.} We cannot guess what sort of life-forms would be evolved in a different world, but we can confidently say that they would tend toward increasing intelligence; and thus we are disposed to look for highly intelligent beings on Mars.

But this argument supposes that the conditions of life, namely air and water, are found on Mars, and it is disputed whether they are found there in sufficient quantity. The late Professor Percival Lowell, who made a lifelong study of Mars, there are hundreds of straight lines drawn across the surface of the planet, and he claimed that they are beds of vegetation marking the sites of great channels or pipes by means of which the ``Martians'' draw water from their polar ocean. \textbf{Sample correction: The late Professor Percival Lowell, who made a lifelong study of Mars, said that there are hundreds of straight lines drawn across the surface of the planes \ldots} Professor W. H. Pickering, another high authority, thinks that the lines are long, narrow marshes fed by moist winds from the poles. There are certainly white polar caps on Mars. They seem to melt in the spring, and the dark fringe round them grows broader.

Other astronomers, however, say that they find no trace of water-vapor in the atmosphere of Mars, and they think that the polar caps may be simply thin sheets of hoar-frost or frozen gas. And they point out that, as the atmosphere of Mars is certainly scanty, and the distance from the sun is so great. \textbf{Sample correction: And they point out that, as the atmosphere of Mars is certainly scanty, and the distance from the sun is so great, there probably isn't water} If one asks why our wonderful instruments cannot settle these points, one must be reminded that Mars is never nearer than 34,000,000 miles from the earth, and only approaches to this distance once in fifteen or seventeen years. The image of Mars on the photographic negative taken in a big telescope is very small. Astronomers rely to a great extent on the eye, which is more sensitive than the photographic plate. But it is easy to have differences of opinion as to what the eye sees. \textbf{Sample correction: Combine the previous two sentences.}

In August, 1924, the planet again be well placed for observation, and we may learn more about it. \textbf{Sample correction: In August, 1924, the planet will again be well placed for observation, and we may learn more about it.} Already a few of the much-disputed lines, which people wrongly call ``canals,'' have been traced on photographs. Astronomers who are skeptical about life on Mars are often not fully aware of the extraordinary adaptability of life. There was a time when the climate of the whole earth, from pole to pole, was semi-tropical for millions of years. No animal could then endure the least cold, yet now we have plenty of Arctic plants and animals. If the cold came slowly on Mars, as we have reason to suppose, the population could be gradually adapted to it. On the whole, it is possible that there is advanced life on Mars, and it is not impossible, in spite of the very great difficulties of a code of communication, that our ``elder brothers'' may yet flash across space the solution of many of our problems.

\item \textbf{Exercise 2: Transitions and Conjunctions}

The following are sample answers. Answers may vary.

As a recently permitted driver, I have become familiar with some of the laws that the Massachusetts State Legislature has enacted in an attempt to curb the number of accidents involving teen drivers. These admirable bills have saved the lives of the state's children and made the roads safer for all drivers and passengers, not to mention pedestrians. \textbf{However}, it is of grave concern that there is currently very little regulation of the age group with the second highest number of motor vehicle crashes, the elderly (age sixty-five and older). The passage of proposed Bill 1914 will mandate that ``the registrar shall require that all persons aged 85 or older who are seeking to renew their operator's licenses take a vision and road test before being reissued such license''. \textbf{As a result}, I am writing to urge your support for this precedent-setting bill.  

Those unfit to operate a vehicle are literally ``accidents waiting to happen'', jeopardizing their own lives and endangering public safety when they operate an automobile. According to the Governor's Highway Safety Bureau, older drivers were involved in 18,743 crashes in Massachusetts in 2002 \textbf{and} this age group was responsible for over 12\% of all motor vehicle fatalities. The National Highway Traffic Safety Administration reports that these statistics continue to rise steadily \textbf{as} the number of traffic-related deaths in the general population decline.  
	
Opponents of this bill may argue that since few elderly people are licensed to drive, they do not pose a serious threat on the road. This is an oversimplification \textbf{because} the number of seniors driving in Massachusetts totaled more than 858,000 in 2000. The aging of the Baby Boom generation \textbf{and} medical advancements have contributed to an increase in elderly drivers. \textbf{Consequently} these problems can only escalate, as the National Institute on Aging predicts that in 30 years, the number of drivers over the age of eighty-five will be five times greater than today. Using this conservative estimate of the number of drivers and current accident rates, it is forecasted that this increase in population will result in the tripling of traffic fatalities caused by this age group. \textbf{Moreover} legislation passed now will lower preventable deaths and will also have widespread implications for the future. 
 
Currently, the only people that are required to have their vision checked are first time license applicants and those persons renewing their licenses at a Registry branch. People of any age who renew their license online are exempt from this personal screening. This is relevant information \textbf{because} researchers at the University of Alabama concurred with a 1995 Johns Hopkins University study which found that state-mandated vision tests of elderly drivers are successful in lowering their accident rate.

\textbf{Therefore}, Bill 1914 does not seek to deprive elders of their independence by mandating that senior citizens forfeit their license at a certain age; \textbf{instead}, the bill assures that those who choose to drive are able to safely do so.   

\item \textbf{Exercise 3: Sentence Order in a Paragraph Directions:}

\begin{enumerate}
\item 4
\item 2
\item 5
\item 1
\item 3
\end{enumerate}

\textbf{The whole paragraph reads as follows:}

Images of 1920s America is one of endless parties, bootlegging, and flappers doing the Charleston all night long. Throughout the novel, The Great Gatsby, the character of Mr. Gatsby is at the center of this stereotype. Coming from humble, mid-western roots and arriving at his current living situation, complete with domestic servants and a forty-acre mansion, his story seems to be the epitome of the American dream. He learns, however, that he can not use his power to buy the sole object of his affection, Daisy Buchannan. Although money correlates to economic power in 1920's American society and allows some leverage in other arenas, the powerful Mr. Gatsby is ultimately rejected by Daisy due to socio-economic class divisions.

\item \textbf{Exercise 4: Making an essay clearer and more concise}

\textbf{The following boldface sentences or phrases can be deleted or moved:}

\textbf{College in the United States is extremely expensive.} With tuition costing as much as \$60,000 per year, many parents and students are worried about paying for college. There are many opportunities for students to get money for school from outside organizations, government agencies, and the particular university that they hope to attend. Students can get grants or scholarships that do not have to be paid back in addition to loans, which students begin to pay back after graduation. \textbf{For example, a student named Ben Kaplan was so worried about paying for Harvard that he wrote scholarship essays that earned him over \$90,000 for college. Would be better later in the essay.}

Students don't have to be the valedictorian or a star quarterback, in fact, lots of qualities, interests, and affiliations will make students eligible. \textbf{Lots of different students can get money for college!} Many organizations, including companies, non-profits, and unique heritage groups, award scholarships--there are scholarships for almost everything. Parents and students should start searching early in the college process and \textbf{not wait until the last minute. There are also national contests for high school writers, historians, and scientists, such as the Ayn Rand Essay Contests, National History Day Contest, and Intel International Science and Engineering Fair. The previous sentence would be better if it was moved to the last sentence of the paragraph.} There are many online resources such as finaid.org and fastweb.com to help students find scholarships. Students can search for books about obtaining money for college, such as How to Go to College Almost for Free, at local libraries, bookstores, and high school guidance departments.  Students can also call local businesses and associations and ask if they have any scholarships or contests with prize money. Some national groups have awards as well. 

In order to get federally backed loans from their university, students need to fill out the Free Application for Federal Student Aid (FASFA) online.\textbf{ Each college also gives out a certain number of merit-based scholarships to attract students to their schools. This sentence is better moved later in the paragraph to the part about merit-based scholarships.} After filling out the FASFA, students are sent the Student Aid Report (SAR).  Colleges use the SAT to determine student's financial aid package, that is to say colleges use this to estimate how much money a student's family can contribute to tuition and how much money a student is able to borrow. \textbf{The best way to get this type of scholarship is to have a high GPA and SAT score relative to the current freshman class. This sentence is better moved later in the paragraph after the introduction of merit-based scholarships} Strong students are awarded merit-based scholarships. Students can also ask about the number and amount of merit-based scholarships that by calling the school or asking an admissions officer during their college visit.  These sentences are better moved later in the paragraph to the part about merit-based scholarships. With scholarships and financial aid, it 'pays' to plan!

\item \textbf{Exercise 5: Dividing an essay into different paragraphs}

\textbf{Note that the bolded sentences are where a new paragraph begins. You may want to provide the hint to students that there are 4 paragraphs.}

\textbf{The poetry of Robert Frost and Thylias Moss were influenced by different events, which attributed to the contrasting attitudes toward similar subjects in their poetry.}  Frost, living in New England, was influenced by serenity and breathtaking scenes of nature. His poem, “Stopping By Woods on a Snowy Evening” epitomizes this, as the Speaker desperately desires to watch the woods fill up with snow, but regretfully acknowledges that he must continue on his journey before he can rest. With thoughts of the beauty of his home, Frost's mood is joyous. Not all poets, however, are as agreeable as he. When writing “Interpretation of A Poem By Frost”, Thylias Moss describes the filling up of woods with snow through the bleak perspective of a young African American girl. In both poems, diction and imagery work collaboratively to establish the two dissimilar tones of the authors. While Frost's attitude towards the metaphorical woods is cheery and appreciative, Moss's seems to be angry and frustrated. 

\textbf{In creating the different tones of voice, Frost and Moss utilize contrasting choices of words.} Frost utilizes passive diction, “easy wind and downy flake”, as well as end rhymes to create a pleasant atmosphere. The rhymes make the poem flow smoothly and the words sound gentle, pleasing to the ear. These combine to produce a lighter mood than in the poem by Moss. The latter author employs words such as “inter”, “emptiness”, “polarity”, “edge” (synonymous for sharp), and “defiance”. The use of these harsh words verifies her feelings toward the subject as angry. The most apparent difference in diction, however, are the words chosen to describe the animal. Frost playfully calls the creature a “little horse” who shakes his harness bells when the Speaker stops to look at the woods. Moss refers to it as a “limited audience”, and believes the girl's efforts of defiance to be wasted on something so unworthy as a horse. Frost's lively description of the horse and entire scene contrasts Moss's degrading and cruel one, representative of Frosts relaxed tones and the overwhelming frustration felt by Moss. 

\textbf{Both authors employ imagery and the connotations of words to establish the different emotions.} Frost describes the environment: woods lovely, dark, and deep, and the frozen lake. In the background, the snow falls down softly, and the only noises are peaceful, that of a light wind and downy flake, and horse bells jingling merrily. After reading this, many imagine a peaceful scene, one which could easily be set during winter holidays (which also conjure feelings of great pleasure). On the other hand, Moss paints an entirely different picture of the landscape. In her poem, the girl is surrounded by an aura of emptiness- not welcome and serene like the former poem- but eerie and frightening. She views as the snow “inters the grass”, which for the readers results in many visualizing violent images of the snow relentlessly beating on the frail grass. Moss uses these morose images to establish her attitude towards the poem, while Frost's jovial illustrations reflect his serene attitude. 

\textbf{The Speakers of both poems experience different metaphorical journeys, and the tones of the authors on current issues in the poems significantly affect their views of the future.} The work “Stopping By Woods on a Snowy Evening" ends with the lines, ``I have miles to go before I sleep, I have miles to go before I sleep”. This suggests the Speaker still has things to accomplish in his life; yet, impacted by Frost's appreciative and joyous temperament, he believes the rest of the journey- the future- will also be something to be enjoyed, as beautiful as the woods described in the beginning of the piece. While Frost's feelings of the present-day imply the future will be pleasurable, Moss's frustration continues. Moss reflects on society, believing there are, ``miles to go, more than the distance from Africa to Andover, more than the distance from black to white before she sleeps with Jim". Moss's anger stemming from prejudice influences her to predict a bleak future; she believes that racism will still be a problem, as she has observed that diverse peoples seem unwilling to cooperate and tolerate each other. The overall meaning of ``Stopping By Woods on a Snowy Evening", authored by an optimistic Robert Frost, is one should appreciate life- now and in the future; however, Thylias Moss's contrasting tone of anger and frustration expresses the challenges of a young African American, and implies unless drastic social measures are taken, intolerance will only continue.        
\end{enumerate}




%%%%%%%%%% Chapter 11 %%%%%%%%%%%%%%%%

\subsection{SAT Worksheet 3C: Paragraph Improvements Practice Questions}

\textbf{Passage 1(about privacy)}

\begin{enumerate}
\item \textbf{Answer (D)}

\begin{enumerate}[label=(\Alph*)]
\item The shift to past tense is inappropriate. The passage is in the present.
\item The beginning clause is passive.
\item The gerund in the beginning makes it seem like we're all currently``thinking''
\item Corrects the point of view error by maintaining ``we'' as the subject
\item Point of view error - unnecessarily shifts between ``you'' and ``we''
\end{enumerate}

\item \textbf{Answer (C)}

These are all transition words. Sentences 2 and 3 point out quite a few things that come up when privacy is thought about. Sentence 4 implies that although a lot of things related to privacy do come up, there are still more things that we cannot imagine. The keyword there is ``still,'' which is the transition word that should be used.

\item \textbf{Answer (B)}

\begin{enumerate}[label=(\Alph*)]
\item The problem here is pronoun reference. The word ``this'' does not refer to anything within the sentence.
\item Correctly indicates what ``this'' should refer to and transitions smoothly from the previous sentence.
\item Too abrupt and too simplistic a statement to really relate back to the previous sentence. The reader is left asking, ``so what?''
\item Does not correct the pronoun reference error mentioned in (A).
\item Too wordy and the phrase ``as an example'' is disconnected - as an example of what?
\end{enumerate}

\item \textbf{Answer (A)}

\begin{enumerate}[label=(\Alph*)]
\item Correct. Best expresses the cause and effect relationship between the two sentences.
\item Run-on. ``Therefore'' is not a conjunction.
\item Starting the sentence with ``with'' is awkward and does not establish the desired cause and effect relationship clearly.
\item The cause and effect relationship is not made clear. Protecting patient information and facing liability are presented as almost two separate things.
\item Misuse of the word ``liable.''
\end{enumerate}


\item \textbf{Answer (B)}

The second paragraph deals with hospitals and how they deal with privacy issues. The paragraph is NOT about lawyers.

\item \textbf{Answer (E)}

Sentence 11 talks about possible consequences and sentence 12 expands upon what those consequences might be. Those consequences for violating patient privacy are the ``hypothetical result'' that sentence 12 elaborates on.
\end{enumerate}

\textbf{Passage 2 (about Korean food)}

\begin{enumerate}
\item \textbf{Answer (C)}

Although other answer choices are grammatically correct, C is the most concise of these answer choices.  

\item \textbf{Answer (C)}

This sentence should go in between when ingredients were scare (and therefore, according to the sentence, they had to cook imitations) and when they were more readily available. This is sentence 4. 

\item \textbf{Answer (D)}

The original sentence contains a modifier error. Only answer choices B, C, and fix this. B contains an unclear pronoun, ``it''. C is incorrect because the clause at the beginning of the sentence does not make sense. 

\item \textbf{Answer (B)}

The original sentence contains a modifier error. Only answer choices B and C fix this. B is both grammatically correct and more concise than C. 

\item \textbf{Answer (D)}

``According to the past'' is implied, given the tense of the word ``signified''. 

\item \textbf{Answer (B)}

Correct use of connectors. ``and'' is not the most logical connector between the two clauses. 
\end{enumerate}