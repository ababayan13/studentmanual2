\section{Strategy for Eliminating Incorrect Answers}

On sentence improvements, the correct answer will be grammatically correct. Therefore, before
you see if an answer choice makes sense in the original sentence, determine if it is grammatically
correct. If not, then you can eliminate this right away.

In SAT sentence improvement problems, you should try to eliminate the 2-3 answer choices that are grammatically incorrect so that you are left with 2-3 other answer choices. The SAT sentence improvement section is looking for the "best" sentence, one that is concise and precise. In SAT world, this translates to the sentence that is not only grammatically correct AND concise. How does the SAT measure "conciseness"? By length. \\

Therefore, the answer choice that you are looking for is grammatically correct and short without changing the meaning of the original sentence. The latter part means that it can not be so short that it is missing a key part of the original sentence,
but this is not usually an issue on sentence improvement problems. \\

Find the shortest answer. Then, check if it preserves the original meaning of the sentence by reading this answer choice in place of the underlined part of the original sentence. If so, this answer choice is the correct answer, so you should mark it.

\bigskip
\textit{Directions: Eliminate the answer choices that are grammatically incorrect for the following sentences. After you have elimintated an answer choice, write why it was incorrect on the line. Then, determine the correct answer using the strategy presented above. The first one has been done for you.}

\begin{enumerate}

\begin{enumerate}
\item The Boston Common is \ul{ older than it but still just as well-maintained as Central Park}.
\begin{enumerate}[label=(\Alph*)]

\item older than it but still just as well-maintained as Central Park 

  \ul{Eliminate because of ambiguous pronoun}
  
\item older than Central Park but just as well-maintained. 

\textbf{\ul{Correct. It is grammatically correct and the most concise. }}

\item  older than Central Park; it is just as well-maintained.   

  \ul{Eliminate because of ambiguous pronoun}
  
\item older and it is just as well-maintained as Central Park. 

\textbf{\ul{Grammatically correct but not as concise as (B).}}

\item just as comfortable as Central Park and it is older than it. \ul{Eliminate because of ambiguous pronoun}

\end{enumerate}

\bigskip


\bigskip
\item While most people detest high prices \ul{for food items, but organic food sells well despite the increased cost.}

\bigskip
\begin{enumerate}[label=(\Alph*)]
\item\hrulefill
\item\hrulefill
\item\hrulefill
\item\hrulefill
\item\hrulefill
\end{enumerate}

\bigskip
\item With determination and diligence, anyone can achieve a high score on the SAT test. 

\bigskip
\begin{enumerate}[label=(\Alph*)]
\item\hrulefill
\item\hrulefill
\item\hrulefill
\item\hrulefill
\item\hrulefill
\end{enumerate}

\bigskip
\item The movie \ul{featured many well-respected actors and was winning many awards for} acting, directing, producing, and writing. 

\bigskip
\begin{enumerate}[label=(\Alph*)]
\item\hrulefill
\item\hrulefill
\item\hrulefill
\item\hrulefill
\item\hrulefill
\end{enumerate}

\bigskip
\item \ul{Many educators believe that technology of the sort that} helps monitor student progress and deliever feedback to parents could be helpful in increasing test performance. 

\bigskip
\begin{enumerate}[label=(\Alph*)]
\item\hrulefill
\item\hrulefill
\item\hrulefill
\item\hrulefill
\item\hrulefill
\end{enumerate}

\bigskip
\item After waiting an hour for her friend, the woman finally \ul{arrived in the theater donning a red dress.}

\bigskip
\begin{enumerate}[label=(\Alph*)]
\item\hrulefill
\item\hrulefill
\item\hrulefill
\item\hrulefill
\item\hrulefill
\end{enumerate}

\bigskip
\item Overjoyed that he was accepted his first choice college, \ul{Stephen is currently being slightly ridiculous.}

\bigskip
\begin{enumerate}[label=(\Alph*)]
\item\hrulefill
\item\hrulefill
\item\hrulefill
\item\hrulefill
\item\hrulefill
\end{enumerate}

\bigskip
\item \ul{Many people think that Americans take the right to vote for granted, and I think that it is the right of Americans to not exercise their right to vote.}

\bigskip
\begin{enumerate}[label=(\Alph*)]
\item\hrulefill
\item\hrulefill
\item\hrulefill
\item\hrulefill
\item\hrulefill
\end{enumerate}

\bigskip
\item The bank robbers threatened the tellers by waving their guns, one of the criminals held a teller hostage until the police arrived. 

\bigskip
\begin{enumerate}[label=(\Alph*)]
\item\hrulefill
\item\hrulefill
\item\hrulefill
\item\hrulefill
\item\hrulefill
\end{enumerate}

\bigskip
\item After a major political event such as September 11th, the president will address the nation, with his purpose being to inform and comfort the public.

\bigskip
\begin{enumerate}[label=(\Alph*)]
\item\hrulefill
\item\hrulefill
\item\hrulefill
\item\hrulefill
\item\hrulefill
\end{enumerate}

\bigskip
\item Mary's secret, the whereabouts of the items that had been missing for weeks, \ul{were more compelling than Jeff's.}

\bigskip
\begin{enumerate}[label=(\Alph*)]
\item\hrulefill
\item\hrulefill
\item\hrulefill
\item\hrulefill
\item\hrulefill
\end{enumerate}
\end{enumerate}
