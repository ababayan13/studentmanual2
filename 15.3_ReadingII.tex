\section{Determining Key Phrases in the Passage}

Some of the most difficult parts of the passage-based reading section is figuring out which words or
sentences are important for understanding the passage and answering the passage-based reading
questions correctly.

When you are reading a passage for the passage-based reading section, you should focus on identifying the main idea and important details rather than precisely what every single word in every
single line of the passage means. This strategy works for the following reasons:

\begin{enumerate}

\item \textbf{Time Limits}- You have a very limited amount of time to read a lot of text and answer questions. It is difficult to understand every word or even every sentence of a text perfectly in the short time that the SATs gives you to read the passages. 

\item \textbf{Active Reading Promotes Understanding}- Finding the main idea in each paragraph as well as for the entire passage helps you to engage with the passage. 

\item \textbf{Answering Questions Accurately}- Understanding the passage as a whole
will help you to answer the general questions in the passage-based reading (e.g. questions about the main idea of the passage) as well as eliminate
incorrect answers from the questions that are more detail-oriented.

Let's remind ourselves about the types of questions on the passage-based readings:

\item \textbf{You can (and should) refer back to the passage for detail questions}- When you are given a
question about a specific detail from the passage, you will have to go back into the passage and
read it again regardless of whether you understood it the first time or not.

\end{enumerate}

Because of the usefulness of identifying the main idea, we will practice quickly reading and identifying the a) one main point in each paragraph
and b) the main idea of the passage.


\subsection{SAT Worksheet: Practice Identifying Main Ideas and Important Points}

\textit{Directions: For each passage, underline the main point in each paragraph. After you have finished the passage, write the main idea of the passage.}

\begin{enumerate}
\item 
\begin{linenumbers}
\modulolinenumbers[5]
\indent What is Society?--Perhaps the great question which sociology seeks to answer is this question which we have put at the beginning. Just as biology seeks to answer the question ``What is life?''; zoology, ``What is an animal?''; botany, ``What is a plant?''; so sociology seeks to answer the question ``What is society?'' or perhaps better, ``What is association?'' Just as biology, zoology, and botany cannot answer their questions until those sciences have reached their full and complete development, so also sociology cannot answer the question ``What is society?'' until it reaches its final development. Nevertheless, some conception or definition of society is necessary for the beginner, for in the scientific discussion of social problems we must know first of all what we are talking about. We must understand in a general way what society is, what sociology is, what the relations are between sociology and other sciences, before we can study the social problems of to-day from a sociological point of view.

\textbf{What is the main idea of this paragraph?:} \hrulefill

\hrulefill

\indent The word ``society'' is used scientifically to designate the reciprocal relations between individuals. More exactly, and using the term in a concrete sense, a society is any group of individuals who have more or less conscious relations to each other. We say conscious relations because it is not necessary that these relations be specialized into industrial, political, or ecclesiastical relations. Society is constituted by the mental interaction of individuals and exists wherever two or three individuals have reciprocal conscious relations to each other. Dependence upon a common economic environment, or the mere contiguity in space is not sufficient to constitute a society. It is the interdependence in function on the mental side, the contact and overlapping of our inner selves, which makes possible that form of collective life which we call society. Plants and lowly types of organisms do not constitute true societies, unless it can be shown that they have some degree of mentality. On the other hand, there is no reason for withholding the term "society" from many animal groups. These animal societies, however, are very different in many respects from human society, and are of interest to us only as certain of their forms throw light upon human society.

\textbf{What is the main idea of this paragraph?:} \hrulefill

\hrulefill

\indent We may dismiss with a word certain faulty conceptions of society. In some of the older sociological writings the word society is often used as nearly synonymous with the word nation. Now, a nation is a body of people politically organized into an independent government, and it is manifest that it is only one of many forms of human society. Another conception of society, which some have advocated, is that it is synonymous with the cultural group. That is, a society is any group of people that have a common civilization, or that are bearers of a certain type of culture. In this case Christendom, for example, would constitute a single society. Cultural groups no doubt are, again, one of the forms of human society, but only one among many. Both the cultural group and the nation are very imposing forms of society and hence have attracted the attention of social thinkers very often in the past to the neglect of the more humble forms. But it is evident that all forms of association are of equal interest to the sociologist, though, of course, this is not saying that all forms are of equal practical importance.

\textbf{What is the main idea of this paragraph?:} \hrulefill

\hrulefill

\indent Any form of association, or social group, which may be studied, if studied from the point of view of origin and development, whether it be a family, a neighborhood group, a city, a state, a trade union, or a party, will serve to reveal many of the problems of sociology. The natural or genetic social groups, however, such as the family, the community, and the nation, serve best to exhibit sociological problems. In this text we shall make particular use of the family, as the simplest and, in many ways, the most typical of all the forms of human association, to illustrate concretely the laws and principles of social development. Through the study of the simple and primary forms of association the problems of sociology can be much better attacked than through the study of society at large, or association in general. From what has been said it may be inferred that society as a scientific term means scarcely more than the abstract term association, and this is correct. Association, indeed, may be regarded as the more scientific term of the two; at any rate it indicates more exactly what the sociologist deals with. A word may be said also as to the meaning of the word social. The sense in which this word will generally be used in this text is that of a collective adjective, referring to all that pertains to or relates to society in any way. The word social, then, is much broader than the words industrial, political, moral, religious, and embraces them all; that is, social phenomena are all phenomena which involve the interaction of two or more individuals. The word social, then, includes the economic, political, moral, religious, etc., and must not be thought of as something set in opposition to, for instance, the industrial or the political.

\end{linenumbers}

\textbf{What is the main idea of this paragraph?:} \hrulefill

\hrulefill

\textbf{What is the main idea of this entire passage?:} \hrulefill

\hrulefill

\hrulefill

\sloppy The passage, SOCIOLOGY AND MODERN SOCIAL PROBLEMS, is adapted from http://www.gutenberg.org/cache/epub/6568/pg6568.html. 
\end{enumerate}