\section{Properties of Exponents}

\bigskip
\textbf{General Equation:} 

\centerline{The Laws of Exponents}

\begin{spacing}{2}
\begin{tabularx}{\textwidth}{*2{@{}>{\centering\arraybackslash}X}}
$a^m\cdot a^n = a^{m+n}$ & $\frac{a^m}{a^n}=a^{m-n}$\\
$(ab)^m=a^m\cdot b^m$ & $(a^m)^n=a^{mn}$
\end{tabularx}
\end{spacing}

\vfill
\begin{enumerate}[labelindent=*,style=multiline,leftmargin=*,label=\textbf{Example \arabic*:}]
\item There are 1000 milimeters in one meter and 1000 meters in one kilometer. How many times larger is a kilometer than a millimeter?

\vfill\item A pallet contains 5 rows of 5 columns of boxes, each column 5 boxes high. How many boxes does the pallet contain?

\vfill\item The intensity of a sound is inversely proportional to the square of your distance from the source of the sound. If  you are 6 times further away from a set of speakers at a concert as your friend, what is the ratio of the intensity of you to your friend?
\end{enumerate}

\vfill
\newpage
\begin{multicols*}{2}
\begin{outline}[enumerate]
\medium

\1 A wooden block occupies a volume of $2a$. If the side length is an integer, what is the smallest possible value of $a$?

\bigskip
\textbf{Equation/Strategy:} \hrulefill

\bigskip
\textbf{Solve:}

\vfill
\2 2
\2 3
\2 4
\2 8
\2 16

\midline

\1 The sum of the squares of two numbers, $a$ and $b$, is equal to to the square of the sum. Which of the following must be true?

\begin{enumerate}[label=\Roman*.]
\item $a\cdot b=0$
\item $a=b$
\item $(a+b)^3=a^3+b^3$
\end{enumerate}

\bigskip
\textbf{Equation/Strategy:} \hrulefill

\bigskip
\textbf{Solve:}

\vfill
\2 I is true
\2 II is true
\2 III is true
\2 I and II is true
\2 I, II, and III are true

\columnbreak
\advanced

\1 The distance of an object falling from height is proportional to the square of time of the fall. If Betty drops a rock from the top of a building, the rock travels a distance of $d^3$ meters after time $t$ seconds. If both the distance and the time are whole numbers and $d\neq t$, what is the least distance traveled by the rock?

\bigskip
\textbf{Equation/Strategy:} \hrulefill

\bigskip
\textbf{Solve:}

\vfill
\2 2 m
\2 4 m
\2 8 m
\2 32 m
\2 64 m

\midline

\1 A coat goes on sale x percent off of the original sale price every month. Which of the following expressions represents the amount taken off after $m$ months?

\bigskip
\textbf{Equation/Strategy:} \hrulefill

\bigskip
\textbf{Solve:}

\vfill
\2 $n\left(\frac{x}{100}\right)^m$
\2 $n\left[1-\left(\frac{x}{100}\right)^m\right]$
\2 $1-\left(\frac{x}{100}\right)^m$
\2 $n\left[1-\left(\frac{x^m}{100}\right)\right]$
\2 $n(1-x^m)$
\end{outline}
\end{multicols*}