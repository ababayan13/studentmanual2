\section{Descriptive Statistics: Mean, Median, and Mode}

\textbf{General Equation}

\bigskip
\begin{equationbox}{Descriptive Statistics}
\begin{center}
\renewcommand{\arraystretch}{1.5}
\begin{tabular}{>{\itshape\bfseries}ll}
Mean & The sum of all terms in a list, divided by the number of terms\\
Median & The middle value in an ordered list\\
Mode & The value that occurs most often\\
Range & The difference between the highest and lowest values in a set
\end{tabular}
\end{center}
\end{equationbox}

\bigskip
\begin{enumerate}[labelindent=*,style=multiline,leftmargin=*,label=\textbf{Example \arabic*:}]
\item The average (arithmetic mean) of Sally's previous three tests is 85. If a fourth test boosts her test average up by 2 points, what grade did she receive on her the fourth test?
\vfill\item The median of a set of four numbers is 10. If a large number greater than the other numbers in the set is added to the set, the median becomes 12. What is the value of the second lowest number in the set?
\vfill\item A set of five numbers has a mode of 10. If the average of the set is 25 and the maximum is 50, how many values in the set are equal to the mode?
\end{enumerate}

\vfill
\newpage
\begin{multicols*}{2}
\begin{outline}[enumerate]
\medium

\1 The mean of a set is less than the median of the set. If $a$ is a positive integer, which of the following could be the set?

\bigskip
\textbf{Equation/Strategy:} \hrulefill

\bigskip
\textbf{Solve:}

\vfill
\2 $\{a,a,a,a,a,a\}$
\2 $\{a,a,a,a,2a,2a\}$
\2 $\{a,a,a,2a,2a,2a\}$
\2 $\{a,a,2a,2a,2a,2a\}$
\2 $\{a,2a,3a,4a,5a,6a\}$

\midline

\1 The range of a set is twice its median. If the median is $m$, what is the average (arithmetic mean) of the set in terms of $m$?

\bigskip
\textbf{Equation/Strategy:}

\bigskip
\textbf{Solve:}

\vfill
\2 0
\2 1
\2 $m$
\2 $2m$
\2 Cannot be determined from the information given

\columnbreak
\advanced

\1 Which of the following are true about the set of consecutive integers from 1 to $n$?

\begin{enumerate}[label=\Roman*]
\item The median is $\frac{n}{2}$
\item The average is $\frac{n+1}{n}$
\item The range is equal to $n-1$
\end{enumerate}

\bigskip
\textbf{Equation/Strategy:} \hrulefill

\bigskip
\textbf{Solve:}

\vfill
\2 I only
\2 II only
\2 I and II are true
\2 II and III are true
\2 I, II, and III are true

\midline

\1 Kelly's test average was an 85. On her last exam, she received a 95 and her test average increased by 2 points. How many tests has Kelly taken?

\bigskip
\textbf{Equation/Strategy:} \hrulefill

\bigskip
\textbf{Solve:}

\vfill
\2 2
\2 3
\2 4
\2 5
\2 6
\end{outline}
\end{multicols*}