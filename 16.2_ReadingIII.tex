\section{Determining the Relationships Between Parts of the Passage}

The reading passages presented in the SAT can be long and, quite frankly, boring. Therefore, it is important to stay engaged with the passage as you read. As discussed in the previous chapter, it is important to identify the main point(s) in each paragraph and the main point of the entire passage and the tone or attitudes conveyed in the passage. It can also be helpful to think about how the sentences and paragraphs connect to each other. Some sample relationships between the paragraphs may be the following:

\bigskip
\begin{itemize}
\item provide a greater explanation of a topic presented in the previous sentence or paragraph 

\bigskip
\item provide other details, an example, or a personal anecdote related to a topic presented in the previous sentence or paragraph

\bigskip
\item provide a contradiction to the statement made in the previous sentence or paragraph

\bigskip
\item provide a conjecture or hypothesis about the topic presented in the previous sentence or paragraph

\bigskip
\item discuss the implications of an idea or topic presented in the previous sentence or paragraph. 

\bigskip
\item transition to a new idea

\end{itemize}

Understanding the main points and elements of the passage can help you to understand the passage better and answer questions, particularly ones about the main idea and tone more effectively. 

\subsection{Practice}

\bigskip
\textit{Directions: Read the following passage and answer the questions that follow:}

\bigskip
\begin{linenumbers*}
\modulolinenumbers[5]
\indent A general impression prevails with the large picture-loving public that a special training is necessary to any proper appreciation of Rembrandt. He is the idol of the connoisseur because of his superb mastery of technique, his miracles of chiaroscuro, his blending of colors. Those who do not understand these matters must, it is supposed, stand quite without the \textbf{pale} of his admirers. Too many people, accepting this as a dictum, take no pains to make the acquaintance of the great Dutch master. It may be that they are repelled at the outset by Rembrandt's indifference to beauty. His pictures lack altogether those superficial qualities which to some are the first requisites of a picture. Weary of the familiar commonplaces of daily life, the popular imagination looks to art for happier scenes and fairer forms. This taste, so completely gratified by Raphael, is at first strangely disappointed by Rembrandt. While Raphael peoples his canvases with beautiful creatures of another realm, Rembrandt draws his material from the common world about us. In place of the fair women and charming children with whom Raphael delights us, he chooses his models from wrinkled old men and beggars. Rembrandt is nevertheless a poet and a visionary in his own way. ``For physical beauty he substitutes moral expression,'' says Fromentin. If in the first glance at his picture we see only a transcript of common life, a second look discovers something in this common life that we have never before seen there. We look again, and we see behind the commonplace exterior the poetry of the inner life. A vision of the ideal hovers just beyond the real. Thus we gain \textbf{refreshment,} not by being lifted out of the world, but by a revelation of the beauty which is in the world. Rembrandt becomes to us henceforth an interpreter of the secrets of humanity. As Raphael has been surnamed ``the divine,'' for the godlike beauty of his creations, so Rembrandt is ``the human,'' for his sympathetic insight into the lives of his fellow men.

\indent Even for those who are slow to catch the higher meaning of Rembrandt's work, there is still much to entertain and interest in his rare story-telling power--a gift which should in some measure compensate for his lack of superficial beauty. His story themes are almost exclusively Biblical, and his style is not less simple and direct than the narrative itself. Every detail counts for something in the development of the dramatic action. Probably no other artist has understood so well the pictorial qualities of patriarchal history. That singular union of poetry and prose, of mysticism and practical common sense, so striking in the Hebrew character, appealed powerfully to Rembrandt's imagination. It was peculiarly well represented in the scenes of angelic visitation. Jacob wrestling with the Angel affords a fine contrast between the strenuous realities of life and the pure white ideal rising majestically beyond. The homely group of Tobit's family is glorified by the light of the radiant angel soaring into heaven from the midst of them.
\end{linenumbers*}
From: https://www.gutenberg.org/files/19602/19602-h/19602-h.htm

\bigskip
\textit{Directions: Answer the questions below. Any questions that ask about evidence for your answer should reference particular parts of the text. Remember, the SATs will ask questions that are supported by text and the correct answers will have strong evidence supporting them in the text. Also, this type of exercise will help you to think of your own answer choice before looking at the multiple choice answers on the SAT.}

\begin{enumerate}
\item What is the main idea of the passage? \hrulefill

\item What is the passage's main purpose? \hrulefill

\item What is your evidence for this? \hrulefill

\item What is the relationship between the 2 paragraphs? \hrulefill

\item What is the author's attitude towards Rembrandt? \hrulefill

\item What is your evidence for this? \hrulefill

\item How does the author's attitude towards Rembrant compared to her attitude about Raphael? \hrulefill

\item What is your evidence for this? \hrulefill

\item How does the author feel about the public? \hrulefill

\item What is your evidence for this? \hrulefill

\item ``Pale'' most closely means \hrulefill

\item What is the purpose of mentioning Rembrandt's ``rare story-telling power''? \hrulefill

\item Why does the author include the description of ``Jacob wrestling with the angel''? \hrulefill

\item What is your evidence for this? \hrulefill
\end{enumerate}