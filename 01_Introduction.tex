\chapter{Introduction}

Hi

ASC's SAT Advanced course is designed to help students master the most difficult topics on the SATs. By focusing on and practicing these topics, advanced SAT students can improve their SAT scores.

\bigskip
Me making changes at 1:06 AM EST. The SAT is composed of ten sections—a 25-minute essay, six 25-minute sections, two 20-minute sections, and one 10-minute section. Total testing time is 3 hours and 45 minutes. The breakdown of each section is as follows:

\vfill
\newpage
\begin{spacing}{1.25}
\begin{tabularx}{\textwidth}{|*4{@{ }>{\raggedright\arraybackslash}X|}}\hline
\centerline{\textbf{Topic}} & \centerline{\textbf{Testing Time}} & {\textbf{Number of Questions}} & \centerline{\textbf{Skills Tested}}\\\hline
Critical Reading & Two 25 minute sections and one 20 minute section & 67 Questions in total: 19 sentence completions and 48 passage based questions & Vocabulary, sentence logic, answering questions and making inferences about a text\\\hline
Math & Two 25 minute sections and one 20 minute section & 54 Questions in total: 44 multiple choice and 10 student-produced responses & Integrating and applying mathematical concepts, including algebra, functions, geometry, probability, statistics, and data interpretation\\\hline
Writing Multiple Choice & One 25 minute section and one 10 minute section & 49 Questions in total: 25 Improving sentences, 18 identifying sentence errors, and 6 improving paragraph questions & Sentence structure and grammar, coherence and cohesion\\\hline
Writing Essay & 25 minutes & Write one essay on a given topic & Writing and analysis skills\\\hline
\end{tabularx}
\end{spacing}

\bigskip
You should also be aware that SAT test includes one 25-minute section called the “experimental section”. It can be in critical reading, math, or writing and is used by the testmakers to design and test questions for future exams. This section does not contribute to your SAT score, however, you won't know which section is the “experimental section”, so you should try your best on every part of the exam.

\bigskip
\underline{SAT Scoring}

Each section (critical reading, math, and writing) is scored by giving you a raw score and then converting that to a scaled score. The raw score is the number of questions that you got correct minus one-fourth of the questions that you got wrong. Leaving a question blank does not affect your score. This equation can be seen as:

\bigskip
\textbf{Raw Score: \underline{\hspace{1in}} correct $\boldsymbol{-}$ 0.25 (\underline{\hspace{1in}}) incorrect = \underline{\hspace{1in}}}

\bigskip
The raw score is then converted to a scaled score between 200 and 800 points. It should be noted that in the writing section, essay score is also factored into your scaled score. Additionally, in the math section, correct student-produced responses (grid ins) are worth one raw score point whereas incorrect student-produced responses (grid ins) do not affect your score.

\bigskip
This brings us to two very important questions:

\bigskip
\begin{inparaenum}[\bfseries1. ]
\item \textbf{If wrong answers lead to subtracting points, but a blank does not affect my score, should I guess?}

\bigskip
The answer is, it depends. If you are able to eliminate at least one choice, then the long run average results in the same or greater raw score than if you didn't guess. This also depends on the individual test taker's personality. Someone that tends to be more cautious might be tempted to leave a lot more blank than one should. On the other hand, a person that is more risk-inclined may have a tendency of not leaving enough blank. Therefore, if you are unsure, then you should complete a practice section leaving a few blank and guessing on the majority of questions that you don't know for sure and find your raw score using the equation above. Then, calculate what your raw score would have been had you left more of the ones that you were unsure of blank. Use whatever strategy gives you the highest score.


\bigskip
\item \textbf{What is a ``good'' SAT score?}
\end{inparaenum}

\bigskip
Although many people know that the coveted 2400 is a perfect score on the SATs, many students and parents wonder what other scores are classified as ``good''. This question does not have a simple answer because a ``good'' SAT score for one college might not be ``good'' for a more competitive school. For example, the top schools in the country tend to look for scores at least in the 700s in each of the three sections (2100 total), whereas smaller, less competitive schools will accept lower scores. While it is true that the higher a student's SAT scores are, the more opportunities will be available to a student, there are schools for students with a large range of SAT scores. Fortunately, there are tools to help students figure out their SAT score goal and what is a ``good'' score for their ability level and the colleges that they are hoping to gain acceptance from.

\bigskip
So what is a “good” score on the SAT? The answer is: it depends on what schools and, in some cases, what programs of study a student is aiming for. Therefore, first step in deciding what a “good score” is would be to decide what colleges or universities interest you and come up with a few ideas of what you might want to study. Next, just check online what scores your ideal school is looking for and make it your goal to score a bit higher just so you stand out among all the other applicants. Oftentimes, the university's website will contain the average SAT score and GPA for admitted applicants. They might also give a 25\% to 75\% percentile scores. Someone in this range might be a good match for the school, whereas it might be more difficult for someone with SAT scores than the 25\% percentile to be admitted. The College Board (the same company that makes the SATs) also has an online program called ``My College Matches'' to help students identify colleges that might be good for them sorted by individual factors such as SAT score. Identifying potential areas of study could also help to put SAT scores in context. For example, a student who scores a 2100 by getting 800s on the verbal and reading section and a 500 on math might make it into a writing program at a top university but would not be considered by a high ranking technical institution. Every school and every student's situations are different.

\newpage
\textbf{\large About the Critical Reading Section}

The critical reading section is composed of two parts, sentence completions and passage-based reading. The sentence completion focuses on vocabulary and sentence logic in order to select the word that best fits in the blank within the sentence. It is imperative that students learn to detect the types of sentence completions and the clues given in each of the sentences which will lead to the correct answer. For the passage-based reading, students will learn the types of passages and questions tested as well as strategies for detecting the correct answer and the reasons that incorrect answers are incorrect.

\bigskip
\textbf{\large About the Math Section}

The following topics are tested on the SAT math section: number and operations, algebra and functions, geometry and measurement, and data analysis, statistics, and probability questions. 

\bigskip
Below is a list from the College Board of each topic tested in more detail:

\bigskip
\begin{outline}
\0 \underline{Number and Operations ($20-25$\% of the test)}
\1 Arithmetic word problems (including percent, ratio, and proportion)
\1 Properties of integers (even, odd, prime numbers, divisibility, and so forth)
\1 Rational numbers
\1 Sets (union, intersection, elements)
\1 Counting techniques
\1 Sequences and series (including exponential growth)
\1 Elementary number theory
\0 \underline{Algebra and functions questions ($35-40$\% of the test)}
\1 Substitution and simplifying algebraic expressions
\1 Properties of exponents
\1 Algebraic word problems
\1 Solutions of linear equations and inequalities
\1 Systems of equations and inequalities
\1 Quadratic equations
\1 Rational and radical equations
\1 Equations of lines
\1 Absolute value
\1 Direct and inverse variation
\1 Concepts of algebraic functions
\1 Newly defined symbols based on commonly used operations
\0 \underline{Geometry and measurement questions ($25-30$\% of the test)}
\1 Area and perimeter of a polygon
\1 Area and circumference of a circle
\1 Volume of a box, cube, and cylinder
\1 Pythagorean theorem and special properties of isosceles, equilateral, and right triangles
\1 Properties of parallel and perpendicular lines
\1 Coordinate geometry
\1 Geometric visualization
\1 Slope
\1 Similarity
\1 Transformations
\0 \underline{Data analysis, statistics, and probability questions ($10-15$\% of the test)}
\1 Data interpretation (tables and graphs)
\1 Descriptive statistics (mean, median, and mode)
\1 Probability
\end{outline}

You will note that there is no pre-calculus or advanced trigonometry (sine, cosine, tangent, etc.), so if you haven't taken these classes, don't worry about it. However, you should be cognizant of when you took what classes and, consequently how much time that you will need to focus on each topic. For example, an 11th grader that took geometry in 9th grade may need to spend more time reviewing geometry than a 11th grader that is currently in a geometry class.

\bigskip
\textbf{\large About the writing section}

The writing section is composed of two sections, the 25 minute essay and multiple choice questions. Students will learn what the SAT graders are looking for and also practice with timing, brainstorming, and writing so that they can get a perfect score on the essay. Students will also be exposed to the three types of writing multiple choice questions-- sentence improvements, sentence errors, and paragraph improvements—as well as the grammatical or other writing concepts taught in this section.

\newpage
\centerline{\textbf{SAT Homework Agenda}}

\bigskip
\begin{spacing}{1.5}
\begin{tabularx}{\textwidth}{|*3{@{}>{\bfseries\centering\arraybackslash}X|}}\hline
Date Due & SAT Verbal & SAT Math\\\hline
& & \\[8ex]\hline
& & \\[8ex]\hline
& & \\[8ex]\hline
& & \\[8ex]\hline
& & \\[8ex]\hline
& & \\[8ex]\hline
& & \\[8ex]\hline
& & \\[8ex]\hline
& & \\[8ex]\hline
\end{tabularx}
\end{spacing}
