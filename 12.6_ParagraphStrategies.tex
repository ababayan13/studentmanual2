\section{SAT Worksheet: Paragraph Improvement Practice}

\textit{Directions: Reading the following essay and complete the questions that follow:}

\bigskip
\textbf{Passage 1:}

\bigskip
\indent \textbf{1} Animals experience stress if they are not in an environment where they can express their evolved preferences. \textbf{2} Therefore, Brian Hare, head of the Canine Cognition Center at Duke Univeristy argues that scientists should take into account the laboratory animal’s stressors when designing protocols so that they can minimize these events and therefore collect data that is more reflective of the animal’s natural responses. \textbf{3} The foundation of the preference based approach is that animals should be treated based on the preferences of their species rather than more sweeping regulations. \textbf{4} Since wild minks spend the majority of time in and around water, when a paper written by Dr. Mason in 2001 gave farm-raised minks a choice of a spacious room with preferred foods and toys or a smaller one with a water pool, the minks overwhelmingly chose the latter. \textbf{5} As a ``proof of concept'', the cortisol levels (used as a measure of stress) of the minks with access to the pool were significantly lower after swimming.

\indent \textbf{6} The Canine Cognition Center uses the preference based approach by recruiting human volunteers with their pet dogs to participate in research studies. \textbf{7} Motivated by love and interest in dogs, many are interested in bringing their dog to the Biological Sciences building to participate. \textbf{8} Because domestic dogs like to be near humans, Hare contends these results are more accurate than if the dogs were to be raised in animal storage facilities. 

\indent \textbf{9} There are, however, sources of bias in these experiments that have been acknowledged by other scientists. \textbf{10} In addition to the limitations of developing investigator-subject relationships, it can be tempting to think that because there are lots of different dogs being tested, the results can be generalized to all dogs. \textbf{11} There is a sample bias because the people that bring their dogs to be tested at the Center are more likely to take good care of their dogs and be at least somewhat interested in dog cognition. \textbf{11} The investigators can not control for the past experiences of the dogs being tested, which is a potential confounding variable in their experiments. 

\begin{enumerate}

\item Where is the best place to insert the following sentence?

\textit{For example, animal cages in laboratory facilities generally do not contain pools.}

\begin{enumerate}[label=(\Alph*)]
\item After sentence 2
\item After sentence 3
\item After sentence 4
\item After sentence 5
\item After sentence 6
\end{enumerate}

\item{Which of the following is the best revision of sentence 5 (reproduced below)?}

\textit{As a ``proof of concept'', the cortisol levels (used as a measure of stress) of the minks with access to the pool were significantly lower after swimming.}

\begin{enumerate}[label=(\Alph*)]
\item As a ``proof of concept'', the cortisol levels (used as a measure of stress) of the minks with access to the pool were significantly lower after swimming.
\item As a ``proof of concept'', the cortisol levels, which are frequently used as a measure of stress, of the minks with access to the pool were significantly lower after swimming.
\item As a ``proof of concept'', the cortisol levels (used as a measure of stress) of the minks with access to the pool were significantly lower after swimming than before.
\item The minks with access to the pool were significantly lower after swimming. 
\item As a ``proof of concept'', the mink's cortisol levels (used as a measure of stress) were significantly lower after swimming than before. 
\end{enumerate}

\item Of the following, which is the best way to phrase sentence 8 (reproduced below)?

\textit{Because domestic dogs like to be near humans, Hare contends these results are more accurate than if the dogs were to be raised in animal storage facilities.}

\begin{enumerate}[label=(\Alph*)]
\item (as it is now)
\item Because domestic dogs like to be near humans, Hare designs experiments so that domestic dogs are near humans and contends that since the dogs like to be near humans, the results collected are more accurate than if the dogs were to be raised in animal storage facilities and then tested.
\item  Because domestic dogs like to be near humans, Hare contends that these results are more accurate than if the dogs had been raised in animal storage facilities. 
\item Because domestic dogs like to be near humans, Hare contends that the results collected with this in mind are more accurate than if the dogs were to be raised in animal storage facilities and then tested.
\item Hare contends these results are more accurate than if the dogs were to be raised in animal storage facilities because domestic dogs like to be near humans.
\end{enumerate}

\item Which revision appropriately shortens sentence 9 (reproduced below)?

\textit{There are, however, sources of bias in these experiments that have been acknowledged by other scientists.}

\begin{enumerate}[label=(\Alph*)]
\item Delete ``however''
\item Delete ``sources of bias''
\item Delete ``in these experiments''
\item Delete  ``that have been acknowledged''
\item Delete ``by other scientists''
\end{enumerate}

\item Which of the following, if placed after sentence 11, would be the most effective concluding sentence for the essay?

\begin{enumerate}[label=(\Alph*)]
\item  Despite some limitations of the preference based approach, it appears to be a good alternative to animal storage and testing typically opposed to by animal advocates. 
\item These limitations mean that all results discovered at the Canine Cognition Center are invalid. 
\item The preference based approach is opposed by many scientists.
\item The Canine Cognition Center is an outstanding examples of Duke's many innovative approaches to research. 
\item Interestingly, a high percentage of people that bring their dogs to the Canine Cognition Center bring their dogs back for another study. 
\end{enumerate}

\end{enumerate} 

\bigskip
\textbf{Passage 2:}

\bigskip
\indent \textbf{1} Familial searching is when law enforcement gets a genetic sample from the criminal at the crime scene and compares the genetic sample to genetic samples that are known (usually they are stored in databases) to see if the criminal’s sample is very similar to someone’s genetic information already in a database.  \textbf{2} The goal of this is to see if the sample in the database could be the sample of a close relative such as a parent, offspring, or sibling of the person that committed the crime, which could help narrow down the number of suspects. \textbf{3} Familial searching was used in the Grim Sleeper case. \textbf{4} The Grim Sleeper was the name given to a serial criminal who was difficult to identify for many years. \textbf{5} The genetic sample found on one survivor was put through the database. \textbf{6} They did not find an exact match but they found that the sample was similar to that of a man named Christopher Franklin. \textbf{7} From this, the law enforcement decided that it was likely that the Grim Sleeper was related to Christopher Franklin. \textbf{8} They got the DNA of Franklin’s father from a discarded napkin and plate, and the sample was a match to the crime scene sample so they were able to convict his father. 

\indent \textbf{9} As of 2010, four states use familial searching. \textbf{10} Many Americans are proponents of the expression, ``my rights end where yours begin.'' \textbf{11} As a result, some people support the use of familial searching in severe crimes, including murder. \textbf{12} They cite the reason that the safety of the victim and other innocent civilians should be prioritized over the privacy of the criminal and their family. \textbf{13} In this group, there are divisions as to whether familial searching should be used crimes that are not as dangerous like misdemeanors because these types of crimes do not pose as dangerous threats to society. \textbf{14} Issues around privacy rights, particularly those surrounding familial searching, are extremely controversial. 

\bigskip
\begin{enumerate}

\item{In context, which of the following phrases is best to insert at the beginning of sentence 3?}
\begin{enumerate}[label=(\Alph*)]
\item{Likewise,}
\item{Furthermore,}
\item{Still,}
\item{Nevertheless,}
\item{For instance,}
\end{enumerate}

\bigskip
\item{Which of the following revisions is most needed in sentence 6 (reproduced below)?}

\textit{They did not find an exact match but they found that the sample was similar to that of a man named Christopher Franklin.}

\begin{enumerate}[label=(\Alph*)]
\item Insert ``In addition'' at the beginning of the sentence
\item Delete ``that of a man named''
\item Change ``They'' at the beginning of the sentence to ``Law enforcement''
\item Delete ``to that of a man named Christopher Franklin''
\item Change ``found'' to ``had found''
\end{enumerate}

\bigskip
\item{Of the following, which is the best way to revise and combine sentences 11 and 12 (reproduced below)?}

\textit{As a result, some people support the use of familial searching in severe crimes, including murder. They cite the reason that the safety of the victim and other innocent civilians should be prioritized over the privacy of the criminal and their family. .}

\begin{enumerate}[label=(\Alph*)]
\item As a result, some people support the use of familial searching in severe crimes including murder because they believe that the safety of the victim and other innocent civilians should be prioritized over the privacy of the criminal and their family.
\item As a result, some people support the use of familial searching in severe crimes, including murder, due to the safety of the victim and other innocent civilians may be prioritized over the privacy of the criminal and their family. 
\item As a result, some people support the use of familial searching in severe crimes, including murder because they do not recognize the right to privacy of the criminal and their family.
\item As a result, some people support the use of familial searching in severe crimes including murder since they believe that the safety of the victim and other innocent civilians should be the highest priority. 
\item As a result, some people support the use of familial searching in severe crimes, including murder; citing the reason that the safety of the victim and other innocent civilians should be prioritized over the privacy of the criminal and their family. 
\end{enumerate}


\item{In context, which of the following is the best version of sentence 13 (reproduced below)?}

\textit{In this group, there are divisions as to whether familial searching should be used crimes that are not as dangerous like misdemeanors because these types of crimes do not pose as dangerous threats to society.}

\begin{enumerate}[label=(\Alph*)]
\item In this group, there are divisions as to whether familial searching should be used crimes that are not as dangerous like misdemeanors because these types of crimes do not pose as dangerous threats to society
\item In this group, there are divisions as to whether familial searching should be used crimes that are not as dangerous like misdemeanors because these types of crimes do not pose as dangerous threats to society as violent crimes.
\item In this group, there are divisions as to whether familial searching should be used crimes that are not as dangerous like misdemeanors. Some argue that they should not be because these types of crimes do not pose as dangerous threats to society.
\item In this group, there are divisions as to whether familial searching should be used crimes that less dangerous. Some argue that familial searching should not be used in crimes like misdemeanors because these crimes are not as dangerous as violent crimes.
\item In this group, there are divisions as to whether familial searching should be used in less dangerous crimes like misdemeanors. Some argue that familial searching should not be used in these crimes because the criminals are not threats to society.
\end{enumerate}

\item{Which of the following sentences should be omitted to improve the unity of the second paragraph?}
\begin{enumerate}[label=(\Alph*)]
\item{Sentence 9}
\item{Sentence 10}
\item{Sentence 12}
\item{Sentence 13}
\item{Sentence 14}
\end{enumerate}

\end{enumerate}

\bigskip
\textbf{Passage 3:}

\bigskip
\indent \textbf{1} Many high school students and parents find it very confusing and stressful to create a list of colleges to visit or to apply to. \textbf{2} As a result, it can be tempting to apply to colleges with ``big names'' or famous alumni or ones that they have heard are ``good'' from friends or the media. \textbf{3} While these sources may be good to start with, there are more important factors to consider. \textbf{4} Pondering these factors can help to narrow down university choices and also better ensure that the student is successful after they enroll at the university. \textbf{5} It is important to note that these selections are personal and unique to every student and as a result the ``right fit'' college for one student may not be the best choice for another student. 

\indent \textbf{6} There are many academic factors to consider when choosing a university. \textbf{7} It is important that the college matches a student’s academic interests and goals as well as their learning style. \textbf{8} While each college has some mandatory classes, university students also have much more freedom to choose classes than those in high school. \textbf{9} As a result, it is important to find colleges that have courses in areas of study that one finds interesting. \textbf{10} To begin this process, students should consider their current academic interests by asking themselves questions such as, ``what classes am I interested in?'' \textbf{11} While a high school student may not know exactly what they want to major in during college, answering these questions can help student to find schools that match these interests, some schools are ``liberal arts'' whereas others are more focused in research or another topic, such as engineering or business.

\indent \textbf{12} Students may have preferences about non-academic aspects of college. \textbf{13} For example, some students may want a rural campus whereas others might want to attend a university in a city. \textbf{14} Students can also consider how far from home they want the college to be and narrow down colleges based on location.  \textbf{15} Parents can also obtain safety records for each college from the college admissions office. \textbf{16} Students should also ask themselves if they want a small, medium, or large college, what sorts of activities or sports are on campus, and the level of diversity on campus. \textbf{17} If a student is unsure of their preferences, then they should visit colleges of different sizes and settings that are nearby and see which environment they prefer. \textbf{18} The facilities and design of the college may be important to some students. \textbf{19} Financial facts can affect the college search. 

\begin{enumerate}

\item Which of the following is the strongest thesis for the passage?

\bigskip
\begin{enumerate}[label=(\Alph*)]
\item All students should attend college
\item People should select a college based on student's interests and preferences
\item Students may prefer colleges in different locations, such as rural or city campuses.
\item Students should select ``big name'' colleges, as they enjoy advantages such as good reputations and famous alumni.
\item There are many factors that determine a school's ranking in national publications
\end{enumerate}

\bigskip
\item In context, the underlined portion of sentence 2 (reproduced below) could best be revised in which of the following ways?

\bigskip
\textit{As a result, it can be tempting to apply to colleges with ``big names'' or famous alumni or ones that they have heard are ``good'' from friends or the media.}

\bigskip
\begin{enumerate}[label=(\Alph*)]
\item As a result, it can be tempting to apply to colleges with ``big names'' or famous alumni. Students might also look at ones that they have heard are ``good'' from friends or the media.
\item As a result, students may be tempted to apply to colleges with ``big names'' or famous alumni or ones that they have heard are ``good'' from friends or the media.
\item As a result, it can be tempting to apply to colleges with ``big names'' or famous alumni or ones that they have heard are ``good''.
\item As a result, it can be tempting to apply to colleges with ``big names'', famous alumni, or ``good'' reputations according to friends or the media.
\item As a result, it can be tempting to apply to colleges with ``big names'', famous alumni, or ones that they have heard are ``good'' from friends or the media.
\end{enumerate}

\item Of the following, which is the best way to revise sentence 11 (reproduced below)?

\bigskip
\textit{While a high school student may not know exactly what they want to major in during college, answering these questions can help student to find schools that match these interests, some schools are ``liberal arts'' whereas others are more focused in research or another topic, such as engineering or business.}

\bigskip
\begin{enumerate}[label=(\Alph*)]
\item While a high school student may not know exactly what they want to major in during college, answering these questions can help student to find schools that match these interests, some schools are ``liberal arts'' whereas others are more focused in research or another topic, such as engineering or business.
\item While a high school student may not know exactly what they want to major in during college, answering these questions can help student to find schools that match their interests. Some schools are ``liberal arts'' whereas others are more focused in research or another topic, such as engineering or business.
\item  While a high school student may not know exactly what they want to major in during college, answering these questions can help student to find schools that match these interests; some schools are ``liberal arts'' whereas others are more focused in research or another topic, such as engineering or business.
\item While a high school student may not know exactly what they want to major in during college, answering these questions can help student to find schools that match their interests. For example, some schools are ``liberal arts'' whereas others are more focused in engineering, business, or research.
\item  While a high school student may not know exactly what they want to major in during college, answering these questions can help student to find schools that match these interests. There are ``liberal arts'' whereas others are more focused in research or another topic, such as engineering or business.
\end{enumerate}

\bigskip
\item Which of the following sentences, if inserted before sentence 19, would best improve the third paragraph?

\bigskip
\begin{enumerate}[label=(\Alph*)]
\item Parents and students should also discuss if they will pay for college out-of-pocket or with merit-based scholarships, financial aid loans, or grants.
\item For example, modern facilities and interesting architecture can help students to imagine themselves as a student on the campus. 
\item These can be particularly important if the student plans to live on campus. 
\item Who would have imagined that there are so many factors to consider when choosing a college?
\item For many students, college is the first time that students will have lived away from their parents for an extended period of time. 
\end{enumerate}

\bigskip
\item Which of the folloiwng would make the most logical final sentence for the essay?

\bigskip
\begin{enumerate}[label=(\Alph*)]
\item It is important to choose a college or college program that is aligned with a student’s personal academic and social goals. 
\item Choosing a college that families can afford helps students to succeed in school.
\item Many students are satisfied with their college choice, although some do transfer after their first or second year.
\item Some high school students may decide to attend trade school or take a gap year before attending college. 
\item Academic and non-academic factors can affect the college's reputation. 
\end{enumerate}

\end{enumerate}

\textit{Note: After you have completed the paragraph improvement section, correct your answers. Then, for each answer you got incorrect, write the correct answer and why the correct answer is correct. Also, attempt understand why the incorrect answer you chose is wrong and write this down as well.}





