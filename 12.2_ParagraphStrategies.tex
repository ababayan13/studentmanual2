The SAT writing section always includes 6 questions dealing with paragraph improvement. Here's
the general strategy for approaching paragraph improvement questions:

\section{Skim the Passage}
The writing section is less about comprehension and more about grammar and sentence relation-
ships. With practice, it'll be easy to deduce a lot of answers without reading the full passage. In
fact, answering questions as you go along will often give you the enough context to answer other
ones. As you are reading the passage and notice an error, then you might want to mark that sentence. This can be helpful because one of the questions following the passage will probably ask about this error. At the same time, don't spend too much time reading or looking for errors.

\bigskip
Unlike the other parts of the writing section, these questions do not ascend in difficulty as you go. In fact, they tend to go in chronological order as they appear in the passage. Because some of the easier questions may be placed at the end, its important that you get to as many questions as possible.

When you are reading the passage, note the purpose of each paragraph. Usually paragraph one will introduce the topic and the following paragraph(s) will go into more depth about something presented in the first paragraph and then conclude the essay. Thinking about the structure of the essay and how different parts of the passage are presented can help with the general or organizational questions about the passage.

\subsection{Practice This Strategy}
\textit{Directions: Quickly skim the following passage. Mark any errors or places where the passage could be improved.}

\begin{inparaenum}[\bfseries 1]
\indent \item Animals experience stress if they are not in an environment where they can express their evolved preferences. \item Therefore, Brian Hare, head of the Canine Cognition Center at Duke University argues that scientists should take into account the laboratory animal's stressors when designing protocols so that they can minimize these events and therefore collect data that is more reflective of the animal's natural responses. \item The foundation of the preference based approach is that animals should be treated based on the preferences of their species rather than more sweeping regulations. \item Since wild minks spend the majority of time in and around water, when a paper written by Dr. Mason in 2001 gave farm-raised minks a choice of a spacious room with preferred foods and toys or a smaller one with a water pool, the minks overwhelmingly chose the latter. \item As a ``proof of concept'', the cortisol levels (used as a measure of stress) of the minks with access to the pool were significantly lower after swimming.

\indent \item The Canine Cognition Center uses the preference based approach by recruiting human volunteers with their pet dogs to participate in research studies. \item Motivated by love and interest in dogs, many are interested in bringing their dog to the Biological Sciences building to participate. \item Because domestic dogs like to be near humans, Hare contends these results are more accurate than if the dogs were to be raised in animal storage facilities. 

\indent \item There are, however, sources of bias in these experiments that have been acknowledged by other scientists. \item In addition to the limitations of developing investigator-subject relationships, it can be tempting to think that because there are lots of different dogs being tested, the results can be generalized to all dogs. \item There is a sample bias because the people that bring their dogs to be tested at the Center are more likely to take good care of their dogs and be at least somewhat interested in dog cognition. \item The investigators can not control for the past experiences of the dogs being tested, which is a potential confounding variable in their experiments.
\end{inparaenum}

\textit{Directions: Answer the questions about the passage. You should be asking yourself these types of questions after you skim the paragraph improvement passage on the real SAT test.}

\begin{spacing}{1.5}
\begin{enumerate}
\item What is the thesis or main idea of the passage? \hrulefill

\textbf{About the organization of the passage}
\item What is the purpose of paragraph 1? \hrulefill

\hrulefill

\item What is the purpose of paragraph 2? \hrulefill

\hrulefill

\item What is the purpose of paragraph 3? \hrulefill

\hrulefill
\end{enumerate}
\end{spacing}