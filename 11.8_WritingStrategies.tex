\section{Misplaced Modifiers}
A modifier is a group of word describing a noun or pronoun. In proper grammar (a.k.a. on the SATs), modifiers need to be next to what they are describing.


For example,

\bigskip
\begin{itemize}
\item{\textbf{Incorrect:} Running down the street, the trash can was in Laurens way.}
\item{The problem here is that we know Lauren is running down the street but the sentence implies that the trash can is running because trash can is what comes directly after the modifier.}
\item{\textbf{Better but still wrong:} While running down the street, Laurens way was blocked by a trash can. In this case Lauren's way appears to be running down the street.}
\item{\textbf{Correct: While running down the street, Lauren had a trash can in her way.}}
\item{Finally, we have Lauren, the subject, being modified and the modifier next to the subject.}
\end{itemize} 

\subsection{SAT Worksheet: SAT Writing Multiple Choice Practice with Modifiers}
\textit{Directions: Underline the modifier and then circle what the modifier is modifying. If the modifier is not next to what it is modifying, re-write the sentence so that the modifier is next to what it is modifying.}

\bigskip
\begin{enumerate}
\item{Since he is a gentleman, Adam is always willing to help others.}
\item{Having come down lightly throughout the morning, Sarah thought that she would be able to move her car through the snow.}
\item{Full of lights, we were impressed with the holiday tree.}

\end{enumerate}
