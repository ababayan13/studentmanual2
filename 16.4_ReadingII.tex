\section{Determining Key Phrases in the Passage-Based Reading Questions

There are six main types of questions asked on the passage-based reading section:
\begin{itemize}
\item Main idea\/primary purpose\/title
\item Details
\item Style
\item Vocabulary in context
\item Inferences\/drawing conclusions
\item Tone
\end{itemize}

Each of these carry their own set of clue words. Look through a recent practice test you have taken and write down the phrasing used with each of these questions:

\begin{itemize}
\item Main idea\/primary purpose\/title \hrulefill
\item Details \hrulefill
\item Style \hrulefill
\item Vocabulary in context \hrulefill
\item Inferences\/drawing conclusions \hrulefill
\item Tone \hrulefill
\end{itemize}

The SATs are a standardized test, which means that they have to be able to demonstrate deni-
tively why there is one and only one correct answer among the five answer choices given in a
multiple choice problem. This means that there is always strong line number evidence for the
correct answer (phrases from the passage) and many times the correct answer is re-worded from
a line in the passage, particularly for main idea and details questions.