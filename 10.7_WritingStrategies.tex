\section{Pronouns}
Some common pronoun rules tested on the SAT Writing Section are agreement, unclear pronouns,
and inconsistent point of view.

\subsection{Agreement with the Antecedent}
\begin{enumerate}
\item{What is an antecedent?} \hrulefill
\item{Circle the antecedent and box the pronoun in the following sentence: Kenna the dog plays
with her favorite chew toys daily.}

\end{enumerate} 

\subsection{Unclear Pronouns}
Pronouns need to be clear regarding who or what they are referring to.
\begin{enumerate}
\item Emily and Kate are going to her house today. Why is this sentence incorrect?

\hrulefill
\item She is going to teach them today. Why is this sentence incorrect?

\hrulefill
\item The amusement park rides are full of mice, therefore we will avoid them. Why is this sentence
incorrect?

\hrulefill
\item Write a possible correction to the following sentence: The amusement park rides are full of
mice, therefore we will avoid them.

\hrulefill
\end{enumerate}

\subsection{Consistent Point of View}
Each sentence or paragraph needs to have the same point of view. For example, the following
sentence is incorrect: 

\bigskip
If one wants to go to the store, then I recommend that you find a driver.

\begin{itemize}
\item{\textbf{Correct:} If you want to want to go to the store, then I recommend that you find a driver.}
\item{\textbf{Another correct version of this sentence is,} If one wants to go to the store, I recommend that
one finds a driver.}
\item{Circle the following choice that makes the sentence correct: You/one should never complain, even when one is given a difficult task.}
\item{Why is the answer that you circled correct?} \hrulefill
\end{itemize}

\subsection{Sample SAT Practice Questions}

\begin{enumerate}

\item \begin{inparaenum}[A]
As nervous as \tfrac{her}{\item} mom \tfrac{was}{\item}, the second grader was determined \tfrac{to walk to}{\item} the bus stop by \tfrac{herself}{\item}.\tfrac{No Error}{\item}
\end{inparaenum}

\item \begin{inparaenum}[A]
If I \tfrac{was}{\item} to invent a time machine, then \tfrac{I}{\item} would go back in time to the period around the Revolutionary War and learn \tfrac{firsthand}{\item} what it \tfrac{was}{\item} like to fight for freedom. \tfrac{No Error}{\item}
\end{inparaenum}

\end{enumerate}