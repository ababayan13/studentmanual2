\chapter{General Strategies for SAT Math}

The SAT math section contains vocabulary terms used in your typical algebra or geometry math class. A list of the most commonly used terms are below:

\bigskip

\renewcommand{\arraystretch}{2}
\begin{tabularx}{\textwidth}{p{1.6in}X}
\textbf{Constant} & A value represented by a variable that will not change\\
\textbf{Expression} & A combination of terms joined by operations\\
\textbf{Equation} & Two sets of expressions joined by an equal sign\\
\textbf{Distributed equally} & To divide evenly among\\
\textbf{Set} & A group whose members are referred to as the set's elements\\
\textbf{Integers} & The set of numbers consisting of positive and negative whole numbers, as well as 0\\
\textbf{Domain} & A function's set of all possible input values (often $x$ values)\\
\textbf{Range} & A function's set of all possible output values (often $y$ values)\\
\textbf{Average/\hspace{2em} Arithmetic mean} & The sum of all values divided by the number of values\\
\textbf{Median} & The center value of an ordered set; If the set contains an even number of elements, it is the average of the two central terms\\
\textbf{Mode} & The most frequent appearing member of a set\\
\end{tabularx}


\vfill
\newpage
\section[SAT Algebra]{General Strategies for SAT Algebra}

It can be difficult to know every concept on the SAT, which topic(s) is best used to solve the problem. Therefore, if you get to a problem that you don't know how to solve, then you can use the following strategies to help you find the right answer. They spell the acronym PUFS.

\vfill
\textbf{\sffamily\large P}lug in real numbers

\bigskip
\textbf{\sffamily\large U}se the answer choices

\bigskip
\textbf{\sffamily\large F}ormulas

\bigskip
\textbf{\sffamily\large S}ee the problem

\vfill
\dotfill

\bigskip
\textbf{\large\sffamily SAT Math Strategy 1: Plug in real numbers}

\bigskip
Many SAT problems are made unnecessarily tricky by using variables instead of numbers.

\begin{enumerate}
\item Assign each variable a unique value (for example, if there are variables $a, b$, and $c$ in the problem, make $a=2, b=3, c=5$). These numbers should be easy to work with. If you are using percentages, it is recommended to use 100\% for the base variable because percentages are out of 100.
\vfill\item Solve the problem with these values instead of variables.
\vfill\item If necessary, translate the numbers back to decimals, percentages, fractions, etc
\end{enumerate}

\vfill
\textbf{Try It:} If $a$ is 1/3 of $b$ and $b$ is 2/5 of $z$ and $z>0$, then $a$ is what percentage of $z$?

\vfill
\newpage
\textbf{\large\sffamily SAT Math Strategy 2: Use the answer choices}

If the question is multiple choice, then you may be able to plug in the answer choices into the question to see which answer choice gives you the correct answer.

\bigskip
Hint: It is suggested in SAT literature that if you try this approach, that you should start with answer choice C and work outwards or answer choice E and go backwards from E to A because the SAT will rarely make problems that can be solved with this strategy to have answer choice A. 

\bigskip
\textbf{Try It:} What is the smallest of 5 consecutive even integers if the sum of these integers equals 300?

\begin{enumerate}[label=(\Alph*)]
\item 50
\item 52
\item 54
\item 56
\item 58
\end{enumerate}

\vfill
\dotfill

\bigskip
\textbf{\large\sffamily SAT Math Strategy \#3: Quickly recall and write down general formulas}

\bigskip
Look at the formulas that are given on the SAT and memorize the ones that aren't. Fill in the formulas below that are not given on the SAT test:

\bigskip
Formula for directly proportional:

\vfill
Formula for inversely proportional:

\vfill
Formula for average:

\vfill
Formula for slope:

\vfill
Formula for equation of a line:

\vfill
\newpage
**When you get to a problem that calls for a topic with a formula associated with it, write the general formula in words on your test booklet. For example, if there was a question that has to do with averages we would write down its equation. Then, rewrite the formula with the numbers or variables from the problems filled in.**

\vfill
\textbf{Try it:} 30\% of the students in Ms. Lee's class had an average test score of 78 points. The rest of the class had an average test score of 84 points. What is the average test score for all students in Ms. Law's class?

\begin{enumerate}[label=(\Alph*)]
\item 79.5
\item 81.0
\item 82.2
\item 83.0
\item 83.1
\end{enumerate}

\vfill
\dotfill

\bigskip
\textbf{\large\sffamily SAT Math Strategy 4: See and solve the problem using visuals like diagrams, charts, or tables}

\bigskip
Drawing a diagram, table, or chart can help you to visualize the problem, particularly for geometric and word problems. They don't need to be detailed or drawn perfectly-just enough to for you to see the information that you currently have and what you need to solve for.

\vfill
\textbf{Try It:} 5 students are going to be lined up against a wall. In how many different ways can the 5 students be arranged in a row?

\begin{enumerate}[label=(\Alph*)]
\item 5
\item 24
\item 25
\item 100
\item 120
\end{enumerate}