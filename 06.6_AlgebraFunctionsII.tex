\section{Newly Defined Symbols Based on Commonly Used Operations}
\bigskip
\textbf{General Equation:} Use the notation as you would with variables

\vfill
\textbf{Example 1.} If $f\otimes g$ is defined as $f\cdot g-(f+g)$, what must be true of $f$ and $g$ so that $f\otimes g=0$?

\vfill
\textbf{Example 2.} If $f\%g$ is defined as the remainder of $f$ when divided by $g$, what is $(x-4x+4)\%(x-2)$?

\vfill
\textbf{Example 3.} $S$ is a set with elements $s_1, s_2,\ldots s_n$. Let $S\bullet S$ be defined as $s_1\cdot s_1+s_2\cdot s_2+\ldots s_n\cdot s_n$. If $S\bullet S=0$ what must be true of the elements of $S$?

\vfill
\newpage
\begin{multicols*}{2}
\begin{outline}[enumerate]
\centerline{\large MEDIUM}

\1 If $x^2-64=36$ and $x-8=9$, what is the value of $x+8$?

\bigskip
\textbf{Equation/Strategy:} \hrulefill

\bigskip
\textbf{Solve:}

\vfill
\2 4
\2 18
\2 25
\2 64
\2 100

\bigskip
\centerline{\rule{0.4\textwidth}{1pt}}

\bigskip
\1 $i\#a$ is defined as $i^a$. If $i\#2=-1$ and $i\#3=-i$, what is the value of $i\#115$?

\bigskip
\textbf{Equation/Strategy:} \hrulefill

\bigskip
\textbf{Solve:}

\vfill
\2 1
\2 -1
\2 $i$
\2 $-i$
\2 0

\columnbreak
\centerline{\large ADVANCED}

\1 $\lfloor x\rfloor$ is defined as the greatest integer less than or equal to $x$ whereas $\lceil x\rceil$ is defined as the least integer greater than or equal to $x$. What is the value of $\lfloor\lceil x\rceil\rfloor$

\bigskip
\textbf{Equation/Strategy:} \hrulefill

\bigskip
\textbf{Solve:}

\vfill
\2 1
\2 0
\2 $x$
\2 $x^2$
\2 Cannot be determined

\bigskip
\centerline{\rule{0.4\textwidth}{1pt}}
\1 The ternary operation $a@b@c$ is defined as $a=b$ when $a\geq0$, and $a=-a$ when $a<0$. What is the value of $a^2@-\left|a^2\right|@\sqrt{(-a^2)^2}$?

\bigskip
\textbf{Equation/Strategy:} \hrulefill

\bigskip
\textbf{Solve:}

\vfill
\2 $-a^2$
\2 $a^2$
\2 $\left|-a^2\right|$
\2 $\sqrt{a^4}$
\2 $\sqrt{-a^4}$
\end{outline}
\end{multicols*}