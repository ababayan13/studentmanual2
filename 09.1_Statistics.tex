\section{Data Interpretation with Tables}
\textbf{General Equation}

\bigskip
\begin{equationbox}{Data and Tables}
\textbf{Definition:} \textbf{\textit{Data}} is a collection of facts and statistics for reference and analysis. The information from data is gathered and presented in a \textbf{\textit{table}} which arranges categories by columns and rows.
\end{equationbox}

\bigskip
\headertitle{Prices of Conventional vs Organic Produce in 2012\footnote{``USDA ERS - Organic Prices." USDA ERS - Organic Prices. N.p., n.d. Web. 06 Apr. 2015.}}
\begin{figure}[h]
\centering
\renewcommand{\arraystretch}{1.5}
\begin{tabular}{cccccccc}
\rowcolor[gray]{.90}
\textbf{\begin{tabular}[c]{@{}c@{}}
Conv/Org\end{tabular}} & \textbf{City} & \textbf{Jan} & \textbf{Feb} & \textbf{Mar} & \textbf{Apr} & \textbf{May} & \textbf{June} \\
Conv & Atlanta & 12.50 & 12.25 & 11.50 & 11.81 & 11.08 & 12.68 \\ \hline
Org & Atlanta & 24.97 & 23.93 & 30.63 & 28.39 & 31.39 & 31.25 \\ \hline
Conv & San Fran & 6.60 & 7.28 & 6.87 & 6.71 & 8.03 & 7.84 \\ \hline
Org & San Fran & 22.00 & 22.00 & 22.00 & 22.00 & 22.00 & 21.96
\end{tabular}
\caption{Prices by city of 1 unit of conventional vs organic produce in 2012}
\end{figure}

\begin{enumerate}[labelindent=*,style=multiline,leftmargin=*,label=\textbf{Example \arabic*:}]
\item According to the table above, which city has the higher average cost of organic produce per unit from January to June?
\vfill\item Approximately what percentage of the average cost of conventional produce of Atlanta is the average cost of organic produce of San Fransisco?
\vfill\item What ratio of months in either city was the cost of 3 units of conventional produce lower than the cost of 1 unit of organic produce?
\end{enumerate}

\vfill
\newpage
\begin{center}
\headertitle{Births in Massachusetts in 2012\footnote{\textit{Massachusetts Births 2011 and 2012}. Boston, MA: Office of Data Management and Outcomes Assessment, Massachusetts Department of Public Health. May 2014.}}

\begin{table}[h]
\small\centering
\begin{tabular}{>{\bfseries}{c}|*{12}{c}}
Month & Jan & Feb & Mar & Apr & May & Jun & Jul & Aug & Sept & Oct & Nov & Dec\\\hline
Births & 5,600 & 5,266 & 5,957 & 5,872 & 6,398 & 6,179 & 6,464 & 6,413 & 6,211 & 6,270 & 5,954 & 5,855\\
\end{tabular}
\caption{\small Births in 2012 by month}
\end{table}
\begin{table}[h]
\centering\small
\begin{tabular}{>{\bfseries}{c}|*8{c}}
Day & Sun & Mon & Tues & Wed & Thurs & Fri & Sat\\\hline
Births & 648 & 870 & 1,001 & 836 & 807 & 817 & 621
\end{tabular}
\caption{\small Births in January 2012 by day}
\end{table}
\end{center}

\begin{multicols*}{2}
\begin{outline}[enumerate]
\medium

\1 If there were 72,439 births in Massachusetts in 2012, what ratio of months were the number of births above the average?

\bigskip
\textbf{Equation/Strategy:} \hrulefill

\bigskip
\textbf{Solve:}

\vfill
\2 $1/6$
\2 $1/4$
\2 $1/3$
\2 $1/2$
\2 $2/3$
\vfill\phantom{}

\columnbreak
\1 Approximately what percentage of babies were born on a weekday in January?

\bigskip
\textbf{Equation/Strategy:} \hrulefill

\bigskip
\textbf{Solve:}

\vfill
\2 $74\%$
\2 $77\%$
\2 $78\%$
\2 $79\%$
\2 $82\%$
\vfill\phantom{}

\pagebreak
\advanced

\1 Approximately what percentage of the month with the lowest number of births does the difference between that month and the month with the highest number of births represent?

\bigskip
\textbf{Equation/Strategy:} \hrulefill

\bigskip
\textbf{Solve:} 

\vfill
\2 $19\%$
\2 $20\%$
\2 $21\%$
\2 $22\%$
\2 $23\%$

\midline

\1 If there were 5 Sundays, Mondays, and Tuesdays in January of 2012, and only 4 of every other day, which day had the highest average number of births for the month?

\bigskip
\textbf{Equation/Strategy:}

\bigskip
\textbf{Solve:}

\vfill
\2 Monday
\2 Tuesday
\2 Wednesday
\2 Thursday
\2 Friday
\end{outline}
\end{multicols*}