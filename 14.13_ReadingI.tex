\section{Vocabulary-in-Context Practice for Reading Comprehension}
In addition to sentence completion questions, the reading comprehension sections on the SATs also have passage-based reading questions. Understanding vocabulary in context can help with these passage-based reading sections for several reasons. 

\begin{enumerate}
\item \large{\textbf{It will help you understand the passages better}}- Besides Sentence Completions, the Critical Reading section of the SATs is composed of passage-based reading questions.  The passages can range from about 100 to 850 words.  They are drawn from a wide range of sources, including natural sciences, literary fiction, and social studies.  Critical Reading questions test your understanding of the written word and your ability to read carefully and analytically.  They also test your vocabulary.  Some questions are based on a single passage, while other questions ask you to compare and contrast two related questions, usually based around the same topic or theme.

\item \large{\textbf{It will help you to solve passage-based reading questions more accurately}}

The types of reading questions are as follows:

\begin{itemize}
\item Main idea\/primary purpose\/title
\item Details
\item Style
\item Vocabulary in context
\item Inferences\/drawing conclusions
\item Tone
\end{itemize}
 
In this lesson, we will focus on Vocabulary in context questions. 

\item \large{\textbf{Discern complex answer choices}}- Answer choices will sometimes contain difficult vocabulary words or complex structures 

\end{enumerate} 

\subsection{SAT Worksheet: Vocabulary-in-Context Practice for Reading Comprehension}

\textit{Use the strategies practiced above to determine the definition of the bolded word. Show your work
on the line underneath the passage. Remember to look at the sentence in context of the passage and define the word \textbf{in the context of the passage}
as this will often utilize a word's secondary definition.}

\begin{enumerate}
\item  The following is a passage from ``The Story of the Crusades'' by E. M. Wilmot\-Buxton.

\indent While Southern Europe was thus being stirred to enthusiasm by being brought into personal contact with one who had seen for himself the woes of the Holy Land, Pope Urban had already called a council to consider the matter in a practical form. At this Council of Placentia, however, the chief part of the attention of those present was drawn to the representations of the Greek Emperor, on whose behalf ambassadors pleaded the cause of the city of Constantinople. If that city fell before the threatened \textbf{onslaught} of the Turks, they said, Christianity must perish for ever in the East, and nothing but a narrow stretch of sea kept the Moslems from the gates of the capital city of the Eastern Empire.

\indent At these words the deepest sympathy was expressed, but it was suggested that the best way of succouring the threatened city was to draw off the attention of the Turks by an attack upon Palestine itself. This was just what Urban desired. A definite march upon Jerusalem would \textbf{fire} the imaginations of men of all ranks far more than an attempt to defend Constantinople before it was actually besieged. The old jealousy between the Eastern and Western Empire had to be reckoned with; and the Emperor Alexios was no heroic figure to stand for the Cause of Christ. The whole question, was, therefore, deferred until the autumn of 1095, when a Council was summoned at Clermont in France.

\indent That dull November day witnessed a most striking scene. The vast open square in front of the Cathedral was crammed with people of all classes drawn from all quarters by the rumour that the subject of a Crusade would be discussed. From the great western door, immediately after High Mass, emerged the figure of the Pope, and a number of bishops and cardinals, dressed in vestments glowing with colour, followed him upon the high \textbf{scaffold} covered with red cloth.

\textit{Definition of ``onslaught''}: \hrulefill
Context clues: \hrulefill

\textit{Definition of ``fire''}: \hrulefill

\hrulefill

\textit{Definition of ``scaffold''}: \hrulefill

\hrulefill

From http://www.gutenberg.org/ebooks/47780. 

\item The following scene takes place at a private school.

The door of Mrs. Boyd’s room stood partly open. Louie Howe gave a light tap and marched in with an air that was rather \textbf{insolent.}

``Oh, Mrs. Boyd, I’ve given my walking dress such an awful tear! Mrs. Barrington said she was quite sure you could mend it. You see I’m going to a sort of musicale in about an hour and I couldn’t take it to the tailors. It’s my best suit, too, and—it must be done very neatly.''

Mrs. Boyd examined it. ``Yes, it’s pretty bad, I’ve done worse though, and part of it will be under the plait. Let me see if I have the right color.''

She opened a box of spools and took up several colors to match.

``Oh, yes, here is one,'' and she gave a smile of gratification.

Louie dropped into a chair. Was she going to wait? Lilian wondered.

``What a pleasant room this is, Mrs. Boyd! But all the rooms are just cozy and nice. Of course Mrs. Barrington can afford to keep it in a lovely fashion for her prices are high and she doesn’t care to take any scholars only from the best families. I do wonder how that Nevins girl slipped in? Her father is a first-class banker, I have understood. They have a big house in New York and a summer house at Elberon, and their New York house is rented out for seven thousand dollars; but isn’t she a terror? How do you stand her, Miss Boyd?''

``She has had very little training. Her mother has been ill and seems very \textbf{indulgent},'' answered Lilian quietly. ``Yet she may make a very fair scholar.''

``It’s funny to hear her talk. Bragging, we call it. Do you suppose the stories are true?''

``Mrs. Barrington would know,'' was the cautious reply.

``Well, I suppose she must be satisfactory or she wouldn’t be here. But there’s common blood back of her somewhere. Money doesn’t give you the prestige of good birth. That always shows—don’t you think so?'' with a confident upward glance.

``I have not had experience enough with the world to judge,'' answered Lilian. ``We lived in a factory town \ldots `'

\textit{Definition of ``insolvent''}: \hrulefill

\hrulefill

\textit{Definition of ``indulgent''}: \hrulefill

\hrulefill

From: The Girls at Mount Morris, by Amanda Minnie Douglas. From \sloppy https://www.gutenberg.org/files/24070/24070-h/24070-h.htm.
\end{enumerate}











