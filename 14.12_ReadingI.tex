\section{SAT Worksheet: Practice with 1-Blank Sentence Completion Questions}

\textit{Directions: Combine your knowledge from this lesson to complete the sentence completions below. After you finish the question, briefly describe the strategies and processes that you used to complete the problem.}

\begin{enumerate}


\item John’s story about the alien abduction was not seen as by \underline{\hspace{2in}} his friends; everyone thought he was lying.

\begin{enumerate}[label=(\Alph*)]
\item billowing  
\item labyrinth 
\item credible
\item abrogate 
\item tangible 
\end{enumerate}

\large{\textbf{Strategies used:}} \hrulefill
\large{\textbf{How I solved the problem:}} \hrulefill

\item The \underline{\hspace{2in}} act was denounced by everyone who heard about it.

\begin{enumerate}[label=(\Alph*)] 
\item reprehensible
\item penguin  
\item fabricated
\item rancorous 
\item enigmatic   
\end{enumerate}

\large{\textbf{Strategies used:}} \hrulefill
\large{\textbf{How I solved the problem:}} \hrulefill

\item Emma received \underline{\hspace{2in}} for her heroic act.
 
\begin{enumerate} [label=(\Alph*)]
\item consecrations 
\item enigmas 
\item fabrications 
\item accolades 
\item amalgamations  
\end{enumerate}

\large{\textbf{Strategies used:}} \hrulefill
\large{\textbf{How I solved the problem:}} \hrulefill

\item Rebecca \underline{\hspace{2in}} to her boss’s demands, as it was easier to comply than to argue.

\begin{enumerate} [label=(\Alph*)]
\item acquiesced l
\item mimicked  
\item consecrated 
\item curtailed
\item plummeted
\end{enumerate}

\large{\textbf{Strategies used:}} \hrulefill
\large{\textbf{How I solved the problem:}} \hrulefill

\item As a result of his disdain for the political climate, Rich decided to \underline{\hspace{2in}} from voting in the presidential election.

\begin{enumerate} [label=(\Alph*)]
\item recite
\item obfuscate
\item abstain
\item destroy
\item initiate
\end{enumerate}

\large{\textbf{Strategies used:}} \hrulefill
\large{\textbf{How I solved the problem:}} \hrulefill

\item The \underline{\hspace{2in}} gasses forced the building to be evacuated.

\begin{enumerate} [label=(\Alph*)]
\item noxious
\item rancorous
\item vicarious
\item enigmatic
\item indiscriminate
\end{enumerate}

\large{\textbf{Strategies used:}} \hrulefill
\large{\textbf{How I solved the problem:}} \hrulefill

\end{enumerate}

\subsection{SAT Worksheet: Practice with 2-Blank Sentence Completion Questions}
 
\textbf{Strategy for 2-Blank Sentence Completions}
When you see a sentence completion question with two blank spaces, you will be asked to idenitify the combination of words in the answer choice that is appropriate for both blanks. While this usually looks more intimidating than the 1-blank questions, they are frequently easier than the 1-blank questions because incorrect answer choices are easier to eliminate.  

Ff you are looking through the answer choices but the other doesn't, then this is \textbf{not} the correct answer choice and you should \textbf{cross out the entire answer choice}. In this manner, you can often eliminate incorrect answer choices by eliminating the first word or the second word. If you don't know what the first word in the answer choice, see if the second word makes sense or vice versa. 

\textit{Directions: Complete the following example as a class. Cross out the answer choices in which one or both of the words do not make sense in the sentence.}

%Make some of the 1st anwer correct and some with the 2nd answer are incorrect.
Do not \underline{\hspace{2in}} the your meaning with a \underline{\hspace{2in}} of fancy words. Rather, speak clearly and simply. 

\begin{enumerate} [label=(\Alph*)]
\item intimidate \ldots debacle
\item synthesize \ldots colossus
\item temper \ldots harangue
\item obfuscate \ldots plethora
\item abrogate \ldots laceration
\end{enumerate} 

\large{\textbf{How I solved the problem:}} \hrulefill


\textit{Directions: Complete the following 2-blank sentence completions individually or with a partner. Cross out the answer choices in which one or both of the words do not make sense in the sentence.}

\begin{enumerate} 
\item The \underline{\hspace{2in}} child believed even the  most absurd \underline{\hspace{2in}}.

\begin{enumerate} [label=(\Alph*)]
\item persnickety \ldots perfidy
\item virulent \ldots quandary
\item complacent \ldots quarry
\item credulous \ldots drivel
\item tawdry \ldots tedium
\end{enumerate}

\large{\textbf{How I solved the problem:}} \hrulefill

\hrulefill

\item The \underline{\hspace{2in}} proposal would not be adopted until is could be shown to not have \underline{\hspace{2in}} side effects.

\begin{enumerate} [label=(\Alph*)]
\item tentative \ldots detrimental
\item assiduous \ldots undetermined
\item palliative \ldots unctuous
\item caucus \ldots analgesic
\item sanguine \ldots inadvertent
\end{enumerate} 

\large{\textbf{How I solved the problem:}} \hrulefill

\hrulefill
\end{enumerate}

