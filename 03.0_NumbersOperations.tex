\chapter{Numbers and Operations}

The SAT Math section relies heavily on you knowledge of the real numbers and their properties. The real numbers can be broken up into two categories: rational numbers and irrational numbers. \textit{Rational numbers} are all numbers that can be expressed as any whole number divided by any other non-zero whole number. Some examples are $-1, 0.75, \frac{2}{3},$ and $-1.\overline{125}$. \textit{Irrational numbers} are all numbers that cannot be expressed as a fraction of whole numbers. They are non-repeating, never ending decimals. For example, $\sqrt{2}, 1.2345\ldots$ are all irrational numbers. The properties of the real numbers to bare in mind are:

\vspace{2em}
\headertitle{Properties of the Real Numbers}

\bigskip
For all real numbers $a,\ b$, and $c$,

\begin{center}
\renewcommand{\arraystretch}{1.25}
\begin{tabular}{|>{\centering}p{2in}|c|}\hline
$a+b=b+a$ & \multirow{2}{*}{Commutative Property}\\
$ab=ba$ & \\\hline
$a+(b+c)=(a+b)+c$ & \multirow{2}{*}{Associative Property}\\
$a(bc)=(ab)c$&\\\hline
\multirow{2}{*}{$a(b+c)=ab+ac$} & \multirow{2}{*}{Distributive Property}\\
&\\\hline
\multirow{2}{*}{$a\cdot1=a$} & \multirow{2}{*}{Multiplicative Identity}\\
&\\\hline
\multirow{2}{*}{$a+0=a$} & \multirow{2}{*}{Additive Identity}\\
&\\\hline
\multirow{2}{*}{$a+(-a)=0$} & \multirow{2}{*}{Additive Inverse}\\
&\\\hline
\multirow{2}{*}{$a\cdot\frac{1}{a}=1$} & \multirow{2}{*}{Multiplicative Inverse}\\
&\\\hline
\end{tabular}
\end{center}