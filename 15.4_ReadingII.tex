\section{Determining Key Phrases in the Passage-Based Reading Questions}

There are six main types of questions asked on the passage-based reading section:

\begin{itemize}
\item Main idea/primary purpose/title
\item Details
\item Style
\item Vocabulary in context
\item Inferences/drawing conclusions
\item Tone
\end{itemize}

Each of these carry their own set of clue words. Look through a recent practice test you have taken and write down the phrasing used with each of these questions:

\begin{itemize}
\item Main idea/primary purpose/title \hrulefill
\item Details \hrulefill
\item Style \hrulefill
\item Vocabulary in context \hrulefill
\item Inferences/drawing conclusions \hrulefill
\item Tone \hrulefill
\end{itemize}

The SATs are a standardized test, which means that they have to be able to demonstrate definitively why there is one and only one correct answer among the five answer choices given in a multiple choice problem. 

\large{\textbf{This means that there is always strong line number evidence for the correct answer (phrases from the passage) and many times the correct answer is re-worded from a line in the passage, particularly for main idea and details questions.}} 

What does this mean that you should do after reading a question?

\large{\textbf{If a line number is given in the question, go back and read a few lines before and a few lines after line(s) the question is referrring to.}}

For example, given the following passage:

\begin{linenumbers*}
\modulolinenumbers[5]
The word ``society'' is used scientifically to designate the reciprocal relations between individuals. More exactly, and using the term in a concrete sense, a society is any group of individuals who have more or less conscious relations to each other. We say conscious relations because it is not necessary that these relations be specialized into industrial, political, or ecclesiastical relations. Society is constituted by the mental interaction of individuals and exists wherever two or three individuals have reciprocal conscious relations to each other. Dependence upon a common economic environment, or the mere contiguity in space is not sufficient to constitute a society. It is the interdependence in function on the mental side, the contact and overlapping of our inner selves, which makes possible that form of collective life which we call society. Plants and lowly types of organisms do not constitute true societies, unless it can be shown that they have some degree of mentality. On the other hand, there is no reason for withholding the term "society" from many animal groups. These animal societies, however, are very different in many respects from human society, and are of interest to us only as certain of their forms throw light upon human society.
\end{linenumbers*}
\textit{The passage, SOCIOLOGY AND MODERN SOCIAL PROBLEMS, is adapted from http://www.gutenberg.org/cache/epub/6568/pg6568.html.}

\bigskip
If a question reads, ``What does the author mean by blah blah blah (line 4)?'' which lines should you go back to the passage and read? \hrulefill

Highlight these lines in the passage now. 

Why is going back to read this important (rather than trying to answer the question based on what you remember reading in the passage)? \hrulefill

section{Determining Why An Author Does X in The Passage}

Many questions (details, style, inferences/drawing conclusions, etc.) are variations one the question of why the author chooses something in the passage or writes a certain way. For example, a question might ask why the author mentions a particular detail in a certain part of the passage. We are looking for how this contributes to the message of the passage. Some sample reasons that a line or phrase may be included in the passage are the following:

\bigskip
\begin{itemize}
\bigskip
\item provide other details, an example, or a personal ancedote related to a topic presented in the previous sentences

\bigskip
\item present the author's own attitude towards the character or topic describe (this can also provide clues to tone)

\bigskip
\item to reveal something about the character or topic described

\bigskip
\item provide a contradiction to the statement made in the previous sentence

\end{itemize}


For this type of question, it is important to re-read a few lines above and below the line cited in the question, looking for how the line cited contributes to this section and the paragraph as a whole. You should create your own answers

Look at the following passage and answer the questions that follow. 

\begin{linenumbers*}
\modulolinenumbers[5]
The word ``society'' is used scientifically to designate the reciprocal relations between individuals. More exactly, and using the term in a concrete sense, a society is any group of individuals who have more or less conscious relations to each other. We say conscious relations because it is not necessary that these relations be specialized into industrial, political, or ecclesiastical relations. Society is constituted by the mental interaction of individuals and exists wherever two or three individuals have reciprocal conscious relations to each other. Dependence upon a common economic environment, or the mere contiguity in space is not sufficient to constitute a society. It is the interdependence in function on the mental side, the contact and overlapping of our inner selves, which makes possible that form of collective life which we call society. Plants and lowly types of organisms do not constitute true societies, unless it can be shown that they have some degree of mentality. On the other hand, there is no reason for withholding the term "society" from many animal groups. These animal societies, however, are very different in many respects from human society, and are of interest to us only as certain of their forms throw light upon human society.
\end{linenumbers*}

\textit{The passage, SOCIOLOGY AND MODERN SOCIAL PROBLEMS, is adapted from http://www.gutenberg.org/cache/epub/6568/pg6568.html.}

\bigskip
\begin{enumerate}

\item Why does the author define ``conscious relations''? \hrulefill

\item Why does the author say that plants and lowly types of organisms do not constitute true societies? \hrulefill

\item What is the tone of the passage? \hrulefill

\item How can the narrator's role in this passage best be described? (Biased? Dispassionate?, etc.) \hrulefill

\end{enumerate}
