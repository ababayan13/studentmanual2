\section{SAT Worksheet: Warm-Up}

Are you having trouble remembering what types of errors are tested in the SAT Writing section?
Try this mnemonic device:

The most commonly tested and missed grammar points can be seen below. When you are answers sentence error or improvement questions, BE A CYCLOPS and always be keep one eye open for these most commonly missed grammar points. If you have already heard the BE A CYCLOPS mnemonic from another section, close your eyes and identify the grammar point that each letter refers to. Then, complete the exercise on the next page. 

\bigskip
\textbf{B is for ``being":} The word ``being" is commonly heard in speech but does not usually make for the best sentences.

\bigskip
\textbf{E is for agrEEmEnt:} Identify the subject and the verb that is associated with the subject. The verb needs to match the subject in number and gender. This means that the subject and the main verb need to be both singular or both plural. 

\bigskip

\bigskip
\textbf{A is for awful verb tense:} Check when the action is happening and then if the given verb tense can be used to describe the time period that the action is happening. 

\bigskip

\bigskip
\textbf{C is for clause (aka commas towards the beginning of the sentence):} Clauses at the beginning of sentences have a description, then a comma, then more words. The description must be describing the first word after the comma. 

\bigskip
\textbf{Y is for you, me, and other pronouns:} If ``you" is not in the underlined section, then it must be paired with ``you" in the underlined section. If ``one" or ``someone" is not in the underlined section, then it must be paired with you in the underlined section. Also, make sure that pronouns like ``it" or ``they" clearly refer to the subject of the sentence. 

\bigskip
\textbf{C if for contrasts and other conjunction/connectors:} Words like ``and" are used to add another idea, however, words like ``but" are used to show differences between things. 

\bigskip
\textbf{L is for list:} If there is a list, all of the words must be the same part of speech and the same verb tense. 

\bigskip
\textbf{O is for ``of" and commas that might separate the subject and the verb:} The verb ending is dependent on the singularity or plurality of the subject.

\bigskip
\textbf{P is for preposition:} Make sure the preposition matches the word before it. To combat this, learn your idioms!

\bigskip
\textbf{S is for short:} Is the sentence as short as it can be without changing the meaning? 

\vfill

\newpage
\textit{Directions: Write 5 sentence error questions from any five different categories in the list above. At least one should have no error. Then, switch with someone in the class so that they can solve your questions.}

\begin{enumerate}
\item
\vfill\item
\vfill\item
\vfill\item
\vfill\item\vfill
\end{enumerate}
\vfill