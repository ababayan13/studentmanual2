\section[Linear Equations]{Solutions of Linear Equations and Inequalities}

\textbf{General Equation}

\bigskip
\begin{multicols}{2}
\setlength{\columnseprule}{0pt}
\begin{equationbox}{$\boldsymbol{y-}$intercept}
The \textit{\bfseries $\boldsymbol{y-}$intercept} of a line is the point where the graph of the line intersects the $y-$axis. Its coordinate is given by the point $(0, b)$ where $b$ is determined by the equation of the line $y=mx+b$.

\phantom{}
\end{equationbox}

\begin{equationbox}{$\boldsymbol{x-}$intercept}
The \textit{\bfseries $\boldsymbol{x-}$intercept} of a line is the point where the graph of the line intersects the $x-$axis. Its coordinate is given by the point $\left(-\frac{m}{b},0\right)$ where $m$ and $b$ are determined by the equation of the line $y=mx+b$.
\end{equationbox}
\end{multicols}

\bigskip
\begin{enumerate}[labelindent=*,style=multiline,leftmargin=*,label=\textbf{Example \arabic*:}]
\item If the cost of a cab has a base fare of \$2.50 and \$0.40 per mile, how much does a 10 mile ride cost?

\vfill\item The cost of tablet devices has dropped an average of 3\% of the original price every quarter of a year. After how long will the cost of a tablet be less than half of the original cost?

\vfill\item The population of Summerville increases at a constant annual rate. The population was recorded in January 2008 as 362,000 and again in July 2010 as 384,000. What is the annual rate of growth of the population?
\end{enumerate}

\vfill
\newpage
\begin{multicols*}{2}
\begin{outline}[enumerate]
\medium

\1 The speed of a minute hand moves at a constant speed of $1/60$ rpm (revolutions per minute). If the current time is 4:00 pm and the minute hand has made 1.75 revolutions, what time was it first recorded at?

\bigskip
\textbf{Equation/Strategy:} \hrulefill

\bigskip
\textbf{Solve:}

\vfill
\2 2:15 pm
\2 2:25 pm
\2 2:45 pm
\2 3:15 pm
\2 3:25 pm

\midline

\1 Both Terri and Sam are shorter than Sissy, and Sissy and Hector is shorter than Roy. Which of the following must be true?

\begin{enumerate}[label=\Roman*.]
\item Terri is shorter than Sam
\item Sissy is shorter than Hector
\item Sam is shorter than Roy
\end{enumerate}

\bigskip
\textbf{Equation/Strategy:} \hrulefill

\bigskip
\textbf{Solve:}

\vfill
\2 I only
\2 II only
\2 III only
\2 I and II
\2 II and III

\columnbreak
\advanced

\1 A chemical reaction results in the release the constant release of 70 joules of energy over a 14 minute period. If the total initial amount of energy in the system was 370 joules, how long will it take for the system to release all of its energy?

\bigskip
\textbf{Equation/Strategy:} \hrulefill

\bigskip
\textbf{Solve:}

\vfill
\2 1 hour 10 minutes
\2 1 hour 14 minutes
\2 1 hour 17 minutes
\2 1 hour 24 minutes
\2 5 hours 17 minutes

\midline

\1 The number of students at Mainsville School with cellphones increases at a constant rate of $n$ students per year. If the number of students with cellphones in January 2010 is 200 and the population of the student body is $p$, which expression represents the proportion of students with cellphones after $m$ months?

\bigskip
\textbf{Equation/Strategy:} \hrulefill

\textbf{Solve:}

\vfill
\begin{spacing}{2.75}
\begin{tabularx}{0.4\textwidth}{*2{@{}X}}
(A) $\frac{mn}{p+200}$ & (B) $\frac{mn}{p-200}$\\
(C) $\frac{mn+200}{p}$ & (D) $\frac{mn+2400}{p}$\\
(E) $\frac{mn+2400}{12p}$
\end{tabularx}
\end{spacing}
\end{outline}
\end{multicols*}