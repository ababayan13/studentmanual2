\section{Practice}

\textit{Directions: Read the following passages and answer the questions that follow.}

\bigskip
\textbf{Questions 1-6 refer to the following passages.}

\bigskip
\textit{The following are two passages on different aspects of freedom and liberty.}

\bigskip
\textbf{Passage 1}
\begin{linenumbers*}
\modulolinenumbers[5]
\indent The struggle between Liberty and Authority is the most conspicuous feature in the portions of history with which we are earliest familiar, particularly in that of Greece, Rome, and England. But in old times this contest was between subjects, or some classes of subjects, and the government. By liberty, was meant protection against the tyranny of the political rulers. The rulers were conceived (except in some of the popular governments of Greece) as in a necessarily antagonistic position to the people whom they ruled. They consisted of a governing One, or a governing tribe or caste, who derived their authority from inheritance or conquest, who, at all events, did not hold it at the pleasure of the governed, and whose supremacy men did not venture, perhaps did not desire, to contest, whatever precautions might be taken against its oppressive exercise. Their power was regarded as necessary, but also as highly dangerous; as a weapon which they would attempt to use against their subjects, no less than against external enemies. To prevent the weaker members of the community from being preyed upon by innumerable vultures, it was needful that there should be an animal of prey stronger than the rest, commissioned to keep them down. But as the king of the vultures would be no less bent upon preying on the flock than any of the minor harpies, it was indispensable to be in a perpetual attitude of defense against his beak and claws. The aim, therefore, of patriots, was to set limits to the power which the ruler should be suffered to exercise over the community; and this limitation was what they meant by liberty. It was attempted in two ways. First, by obtaining a recognition of certain immunities, called political liberties or rights, which it was to be regarded as a breach of duty in the ruler to infringe, and which if he did infringe, specific resistance, or general rebellion, was held to be justifiable. A second, and generally a later expedient, was the establishment of constitutional checks; by which the consent of the community, or of a body of some sort, supposed to represent its interests, was made a necessary condition to some of the more important acts of the governing power. To the first of these modes of limitation, the ruling power, in most European countries, was compelled, more or less, to submit. It was not so with the second; and to attain this, or when already in some degree possessed, to attain it more completely, became everywhere the principal object of the lovers of liberty. And so long as mankind were content to combat one enemy by another, and to be ruled by a master, on condition of being guaranteed more or less efficaciously against his tyranny, they did not carry their aspirations beyond this point.

\indent A time, however, came, in the progress of human affairs, when men ceased to think it a necessity of nature that their governors should be an independent power, opposed in interest to themselves. It appeared to them much better that the various magistrates of the State should be their tenants or delegates, revocable at their pleasure. In that way alone, it seemed, could they have complete security that the powers of government would never be abused to their disadvantage. By degrees, this new demand for elective and temporary rulers became the prominent object of the exertions of the popular party, wherever any such party existed; and superseded, to a considerable extent, the previous efforts to limit the power of rulers. As the struggle proceeded for making the ruling power emanate from the periodical choice of the ruled, some persons began to think that too much importance had been attached to the limitation of the power itself. That (it might seem) was a resource against rulers whose interests were habitually opposed to those of the people. What was now wanted was, that the rulers should be identified with the people; that their interest and will should be the interest and will of the nation. The nation did not need to be protected against its own will. There was no fear of its tyrannising over itself. Let the rulers be effectually responsible to it, promptly removable by it, and it could afford to trust them with power of which it could itself dictate the use to be made. Their power was but the nation's own power, concentrated, and in a form convenient for exercise. This mode of thought, or rather perhaps of feeling, was common among the last generation of European liberalism, in the Continental section of which it still apparently predominates. Those who admit any limit to what a government may do, except in the case of such governments as they think ought not to exist, stand out as brilliant exceptions among the political thinkers of the Continent. A similar tone of sentiment might by this time have been prevalent in our own country, if the circumstances which for a time encouraged it, had continued unaltered.

\bigskip
\textbf{Passage 2}

\indent Whenever those states which have been acquired as stated have been accustomed to live under their own laws and in freedom, there are three courses for those who wish to hold them: the first is to ruin them, the next is to reside there in person, the third is to permit them to live under their own laws, drawing a tribute, and establishing within it an oligarchy which will keep it friendly to you. Because such a government, being created by the prince, knows that it cannot stand without his friendship and interest, and does it utmost to support him; and therefore he who would keep a city accustomed to freedom will hold it more easily by the means of its own citizens than in any other way.

\indent There are, for example, the Spartans and the Romans. The Spartans held Athens and Thebes, establishing there an oligarchy, nevertheless they lost them. The Romans, in order to hold Capua, Carthage, and Numantia, dismantled them, and did not lose them. They wished to hold Greece as the Spartans held it, making it free and permitting its laws, and did not succeed. So to hold it they were compelled to dismantle many cities in the country, for in truth there is no safe way to retain them otherwise than by ruining them. And he who becomes master of a city accustomed to freedom and does not destroy it, may expect to be destroyed by it, for in rebellion it has always the watchword of liberty and its ancient privileges as a rallying point, which neither time nor benefits will ever cause it to forget. And whatever you may do or provide against, they never forget that name or their privileges unless they are disunited or dispersed, but at every chance they immediately rally to them, as Pisa after the hundred years she had been held in bondage by the Florentines.

\indent But when cities or countries are accustomed to live under a prince, and his family is exterminated, they, being on the one hand accustomed to obey and on the other hand not having the old prince, cannot agree in making one from amongst themselves, and they do not know how to govern themselves. For this reason they are very slow to take up arms, and a prince can gain them to himself and secure them much more easily. But in republics there is more vitality, greater hatred, and more desire for vengeance, which will never permit them to allow the memory of their former liberty to rest; so that the safest way is to destroy them or to reside there.
\end{linenumbers*}

\bigskip
\textit{From: \url{https://www.gutenberg.org/files/1232/1232-h/1232-h.htm} and \url{https://www.gutenberg.org/files/34901/34901-h/34901-h.htm}}

\bigskip
\begin{enumerate}

\item Which of the following best describes the relationship between the two passages?

\bigskip
\begin{enumerate}[label=(\Alph*)]
\item Passage 1 and Passage 2 are both describe primarily focused with the use of force to gain or maintain power or overthrow the government. 
\item Passage 1 describes a newer form of government, whereas Passage 2 focuses on a form that is older but less commonly implemented today. 
\item Passage 1 wants the head of every government to be representative of the people, whereas Passage 2 specifically states that it is more effective to have a monarchy
\item Passage 1 is somewhat conservative in its approach to government, whereas the suggestions made in Passage 2 are novel. 
\item Passage 1 suggests that heads of governments should look act favorably for the people they represent and be removed if they do not, whereas Passage 2 discusses how rulers should act to maintain power. 
\end{enumerate}

\bigskip
\textbf{Evidence:} \hrulefill

\bigskip
\item In passage one, the use of imagery with animals suggests that 

\bigskip
\begin{enumerate}[label=(\Alph*)]
\item other forms of government (e.g. oligarchy) should be associated with inhuman behavior
\item we should look to nature to see how animal groups keep order
\item observations of other animals can be exactly the same as those of humans
\item we should exploit the weaker members of society because they are not as useful as othe rmembers
\item governments that expand liberty will be more effective than those that do not
\end{enumerate}

\bigskip
\textbf{Evidence:} \hrulefill

\bigskip
\item The tone of passage two can best be described as

\bigskip
\begin{enumerate}[label=(\Alph*)]
\item somewhat apathetic
\item dramatically reverent
\item unapologetically frank
\item abundantly enthusiastic
\item blatantly impractical
\end{enumerate}

\bigskip
\textbf{Evidence:} \hrulefill


\bigskip
\item What is the role of the examples of the Spartans and the Romans in passage 2?

\bigskip
\begin{enumerate}[label=(\Alph*)]
\item to highlight the brutality of the Romans
\item to enumerate the courses of actions and outcomes after taking power over a previous republic
\item to suggest that rulers compromise with  members of the previous republic 
\item to demonstrate that the Spartans were less brutal and more effective than the Romans
\item to illustrate that military force is sometimes justified
\end{enumerate}

\bigskip
\textbf{Evidence:} \hrulefill

\bigskip
\item How might the author of passage two respond to the idea from passage one that ``The rulers were conceived (except in some of the popular governments of Greece) as in a necessarily antagonistic position to the people whom they ruled.''

\bigskip
\begin{enumerate}[label=(\Alph*)]
\item He would disagree with the use of the title ``rulers.''
\item He would agree that an antagonistic position was necessary to maintain power
\item He would suspect that these rulers would not take his ideas seriously
\item He would advocate for these rulers to be more aligned with the viewpoints of the people not in government
\item He would urge current rulers to not take extreme action to maintain power for fear of a rebellion
\end{enumerate}

\bigskip
\textbf{Evidence:} \hrulefill

\bigskip
\item How would the author of passage 1 react to the author of passage two who asserts, `` But in republics there is more vitality, greater hatred, and more desire for vengeance, which will never permit them to allow the memory of their former liberty to rest; so that the safest way is to destroy them or to reside there.''?

\bigskip
\begin{enumerate}[label=(\Alph*)]
\item He would be ambivalent towards the entire statement. 
\item He would look favorably upon the first part of the sentence and would agree that the ruler should ``reside there.''
\item He would disagree with the entire statement.
\item He would defend this statement passionately. 
\item He would agree with the first part of the statement (before the semi-colon) but disagree with the latter part. 
\end{enumerate}

\bigskip
\textbf{Evidence:} \hrulefill

\end{enumerate}

\bigskip
\textit{The following are two arguments made during the women's suffrage movement. Remember, these were written about 100 years ago and may not express opinions that we think are appropriate in today's society.}

\bigskip
\textbf{Passage 1}
\begin{linenumbers*}
\modulolinenumbers[5]
\indent

\indent I have seldom felt so proud of being a representative of the people as now, when it gives me an opportunity to advocate a cause which can not be represented or defended in this chamber by those directly and particularly affected by it, owing to the level of prejudice that the beliefs and ideas of the past have left in the mind of modern man. The cause of female suffrage is one sure to strike a sympathetic chord in every unprejudiced man, because it represents the cause of the weak who, deprived of the means to defend themselves, are compelled to throw themselves upon the mercy of the strong. But it is not on this account alone that this cause has my sympathy and appeals to me. It has, besides, the irresistible attraction of truth and justice, which no open and liberal mind can deny. If our action as legislators must be inspired by the eternal sources of right, if the laws passed here must comply with the divine precept to give everybody his due, then we can not deny woman the right to vote, because to do otherwise would be to prove false to all the precepts and achievements of democracy and liberty which have made this century what may be properly called the century of vindication.

\indent Female suffrage is a reform demanded by the social conditions of our times, by the high culture of woman, and by the aspiration of all classes of society to organize and work for the interests they have in common. We can not detain the celestial bodies in their course; neither can we check any of those moral movements that gravitate with irresistible force towards their center of attraction: Justice. The moral world is governed by the same laws as the physical world, and all the power of man being impotent to suppress a single molecule of the spaces required for the gravitation of the universe, it is still less able to prevent the generation of the ideas that take shape in the mind and strive to attain to fruition in the field of life and reality.

\indent I remember very well that in the past, not so very long ago, the apprehension and fears were felt with regard to higher education for our women. How ridiculous—the same people argued—is it for woman to study history, mathematics, philosophy, and chemistry, which are not only superior to the assimilating power of her deficient brain, but will make her presumptuous and arrogant and convert her into a hybrid being without grace or strength, intolerable and fatuous, with a beautiful, but empty head and a big, but dry heart! However, we admitted the women to our high schools and universities and made it possible for them to attain to the degree of bachelor of arts and graduate in law, medicine, and other professions. Can it be said that those women have perverted the homes of their parents or that, when they married, they were a source of disgrace or scandal to their husbands? We are now able to observe the results, and if these results are found to be detrimental to the social and political welfare of the country, it is our duty to undo what we have done and to return to where we were before.

\indent Education has not atrophied or impaired any of the fundamental faculties of woman; on the contrary, it has enhanced and enriched them. Far from being a constant charge to the family, the educated woman has often been its sustain and support in times of great need. The educated woman has not become a blue-stocking, that fatuous creature imagined by certain elements, nor has she lost any of her feminine charms by being able to argue and discuss on every subject with the men. On the contrary, it seems to lend her an additional grace and charm, because she understands us better and can make herself better understood. Thank God, people are no longer ready to cast ridicule upon what some used to consider the foolish presumption of women to know as much as the men, and this is doubtless due to the fact that the disastrous results predicted by the calamity howlers, the terrible prophets of failure, have not materialized.

\indent Very well; if you allow the instruction and education of woman in all the branches of science, you must allow woman to take on her place not only in domestic life, but also in social and public life. Instruction and education have a twofold purpose; individually, they redeem the human intellect from the perils of ignorance, and socially they prepare man and woman for the proper performance of their duties of citizenship. A person is not educated exclusively for his or her own good, but principally to be useful and of service to the others. Nothing is more dangerous to society than the educated man who thinks only of himself, because his education enables him to do more harm and to sacrifice everybody else to his convenience or personal ambition. The real object of education is public service, that is, to utilize the knowledge one has acquired for the benefit and improvement of the society in which one is living.
In societies, therefore, where woman is admitted to all the professions and where no source of knowledge is barred to her, woman must necessarily and logically be allowed to take a part in the public life, otherwise, her education would be incomplete or society would commit an injustice towards her, giving her the means to educate herself and then depriving her of the necessary power to use that education for the benefit of society and collective progress.

\indent I can not resist this conclusion. If woman is given equal opportunities with man for educating herself; if she is encouraged to learn and study the knowledge of the world and of life, it is but just that the doors of public life should be thrown open to her in order to allow her to play in it the part to which she is entitled. In backward societies, woman is taught only such knowledge as she requires for the home; that is, she is unconsciously prepared for that gentle, that charming slavery so pleasing to the masculine sex. The question now before us is what system we shall adopt for our women: whether slavery and ignorance, or liberty and education.

\indent Female suffrage is the consequence of the education of woman; it is also the consequence of her liberty of conscience. The vote is the expression of political faith, just as worship is the expression of religious faith. There is no more reason for keeping woman from the ballot box than there is for preventing her from going to church. There is no reason why suffrage should be a privilege of sex, considering that the duties of citizenship rest as heavily upon woman as upon man. Is woman under less obligation to strive for the welfare and future of her country because she is a woman? To attempt to \textbf{curtail} the activity of woman in public life is tantamount to declaring that a woman must not love her country and must not dedicate any of her time to her duties of citizenship; that she must not feel the affection and devotion which the idea of native land and community awaken in every well-born creature.


\indent All social classes are entitled to representation in the legislative houses and are thus enabled to work for legislation favoring their interests: the merchants, the laborers, the manufacturers, all can choose one of their own number; but the women, who are not merely one group or class, but a collection of groups or classes, who represent one-half of the country and have interests of their own to defend, not only with relation to their sex, but also with relation to their position in the family, are not allowed to vote and are therefore not permitted to have representatives to promote and defend laws and measures necessary for their protection and betterment. Is this just? Is this even moral? Female labor can be exploited in shop and factory; feminine virtue can be made the object of commerce, and yet woman is not allowed to defend directly the interests of her sex, owing to one of those aberrations of the moral sense that spring from the crass egoism and brutal tyranny of man. If woman were at least exempt from complying with the laws! But no; the law binds the woman as well as the man; the Penal Code menaces man and woman alike with the sword of justice, and the burden of taxation rests upon both the masculine and the feminine wealth. Consequently, before the law, their duties are the same, but their rights are not.

\indent Is it not strange that our laws should contain so much social injustice towards woman, so much exasperating discrimination, all based upon the theory of the servile dependency of woman upon man, resulting from her congenital mental and physical inferiority? Moebius is incarnated in our Codes, governs our policy, and influences all the customs and usages of our social and political life, to such a point that we ought to be ashamed that in the midst of this era of vindication, when all classes have secured their right to liberty and equality, woman has been kept indefinitely upon the same level as in the centuries of subjection.

\textit True democracy can not exist with one-half of the people free and the other half in a stage of slavery, with one-half of the people with representation in the public affairs and the other half without it. The people does not consist of men alone, but of women as well, and conditions being equal, woman should have the same political rights as man. She should, at least, have those fundamental rights the exercise of which, like that of the right to vote, requires nothing but intelligence and capacity, in order that she may have some voice in the decision of her own destiny and may herself fight the battles for her honor, her liberty, and other rights neglected or ignored by man on account of the undisputed monopoly exercised by him over the public affairs.

\bigskip
\textbf{Passage 2}

\indent The natural position of woman is clearly, to a limited degree, a subordinate one. Such it has always been throughout the world, in all ages, and in many widely different conditions of society. There are three conclusive reasons why we should expect it to continue so for the future.

\indent Woman in natural physical strength is so greatly inferior to man that she is entirely in his power, quite incapable of self-defense, trusting to his generosity for protection. In savage life this great superiority of physical strength makes man the absolute master, woman the abject slave. And, although every successive step in civilisation lessens the distance between the sexes, and renders the situation of woman safer and easier, still, in no state of society, however highly cultivated, has perfect equality yet existed. This difference in physical strength must, in itself, always prevent such perfect equality, since woman is compelled every day of her life to appeal to man for protection, and for support.

\indent Woman is also, though in a very much less degree, inferior to man in intellect. The difference in this particular may very probably be only a consequence of greater physical strength, giving greater power of endurance and increase of force to the intellectual faculty connected with it. In many cases, as between the best individual minds of both sexes, the difference is no doubt very slight. There have been women of a very high order of genius; there have been very many women of great talent; and, as regards what is commonly called cleverness, a general quickness and clearness of mind within limited bounds, the number of clever women may possibly have been even larger than that of clever men. But, taking the one infallible rule for our guide, judging of the tree by its fruits, we are met by the fact that the greatest achievements of the race in every field of intellectual culture have been the work of man. It is true that the advantages of intellectual education have been, until recently, very generally on the side of man; had those advantages been always equal, women would no doubt have had much more of success to record. But this same fact of inferiority of education becomes in itself one proof of the existence of a certain degree of mental inequality. What has been the cause of this inferiority of education? Why has not woman educated herself in past ages, as man has done? Is it the opposition of man, and the power which physical strength gives him, which have been the impediments? Had these been the only obstacles, and had that general and entire equality of intellect existed between the sexes, which we find proclaimed to-day by some writers, and by many talkers, the genius of women would have opened a road through these and all other difficulties much more frequently than it has yet done. At this very hour, instead of defending the intellect of women, just half our writing and talking would be required to defend the intellect of men. But, so long as woman, as a sex, has not provided for herself the same advanced intellectual education to the same extent as men, and so long as inferiority of intellect in man has never yet in thousands of years been gravely discussed, while the inferiority of intellect in woman has been during the same period generally admitted, we are compelled to believe there is some foundation for this last opinion. The extent of this difference, the interval that exists between the sexes, the precise degree of inferiority on the part of women, will probably never be satisfactorily proved.

\indent Believing then in the greater physical powers of man, and in his superiority, to a limited extent, in intellect also, as two sufficient reasons for the natural subordination of woman as a sex, we have yet a third reason for this subordination. Christianity has raised woman from slavery and made her the thoughtful companion of man; and it places her by his side, his truest friend, his most faithful counselor, his helpmeet in every worthy and honorable task. It protects her far more effectually than any other system. It cultivates, strengthens, elevates, purifies all her highest endowments, and holds out to her aspirations the most sublime for that future state of existence, where precious rewards are promised to every faithful discharge of duty, even the most humble.

\indent It is true that the world has often seen individual women called by the manifest will of Providence to positions of the highest authority, to the thrones of rulers and sovereigns. And many of these women have discharged those duties with great intellectual ability and great success. It is rather the fashion now among literary men to depreciate Queen Elizabeth and her government. But it is clear that, whatever may have been her errors—and no doubt they were grave—she still appears in the roll of history as one of the best sovereigns not only of her own house, but of all the dynasties of England. Certainly she was in every way a better and a more successful ruler than her own father or her own brother-in-law, and better also than the Stuarts who filled her throne at a later day. Catherine of Russia, though most unworthy as a woman, had a force of intellectual ability quite beyond dispute, and which made itself felt in every department of her government. Isabella I. of Spain gave proof of legislative and executive ability of the very highest order; she was not only one of the purest and noblest, but also, considering the age to which she belonged, and the obstacles in her way, one of the most skillful sovereigns the world has ever seen. Her nature was full of clear intelligence, with the highest moral and physical courage. She was in every way a better ruler than her own husband, to whom she proved nevertheless an admirable wife, acting independently only where clear principle was at stake. The two greet errors of her reign, the introduction of the Inquisition and the banishment of the Jews, must be charged to the confessor rather than to the Queen, and these were errors in which her husband was as closely involved as herself. On the other hand, some of the best reforms of her reign originated in her own mind, and were practically carried out under her own close personal supervision. Many other skillful female rulers might be named.

\indent We have arrived at the days foretold by the Prophet, when "knowledge shall be increased, and many shall run to and fro." The intellectual progress of the race during the last half century has indeed been great. But admiration is not the only feeling of the thoughtful mind when observing this striking advance in intellectual acquirement. We see that man has not yet fully mastered the knowledge he has acquired. He runs to and fro. He rushes from one extreme to the other. How many chapters of modern history, both political and religious, are full of the records of this mental vacillation of our race, of this illogical and absurd tendency to pass from one extreme to the point farthest from it!

\indent An adventurous party among us, weary of the old paths, is now eagerly proclaiming theories and doctrines entirely novel on this important subject. The EMANCIPATION OF WOMAN is the name chosen by its advocates for this movement. They reject the idea of all subordination, even in the mildest form, with utter scorn. They claim for woman absolute social and political equality with man. And they seek to secure these points by conferring on the whole sex the right of the elective franchise, female suffrage being the first step in the unwieldy revolutions they aim at bringing about. These views are no longer confined to a small sect. They challenge our attention at every turn. We meet them in society; we read them in the public prints; we hear of them in grave legislative assemblies, in the Congress of the Republic, in the Imperial Parliament of Great Britain. The time has come when it is necessary that all sensible and conscientious men and women should make up their minds clearly on a subject bearing upon the future condition of the entire race.
\end{linenumbers*}

\bigskip
\textit{From: \url{http://www.gutenberg.org/files/26699/26699-h/26699-h.htm} and \url{http://www.gutenberg.org/files/2157/2157-h/2157-h.htm}}

\bigskip
\textbf{Questions 1-6 refer to the following passages.}

\begin{linenumbers*}
\modulolinenumbers[5]
\indent
\end{linenumbers*}


\bigskip
\begin{enumerate}

\item Passage 1 does NOT include which of the following arguments concerning women's right to vote?

\bigskip
\begin{enumerate}[label=(\Alph*)]
\item that women need to follow government rules but are not represented in government
\item the parallels with higher education for women 
\item that denying the vote is against religious teachings from the Bible
\item granting the vote would not change women's feminine qualities
\item that women have the capacity to vote
\end{enumerate}

\bigskip
\textbf{Evidence:} \hrulefill

\bigskip
\item The word curtail (in bold) most closely means

\bigskip
\begin{enumerate}[label=(\Alph*)]
\item to curb
\item to free
\item to permit
\item to condemn 
\item to increase
\end{enumerate}

\bigskip
\textbf{Evidence:} \hrulefill

\bigskip
\item 

\bigskip Which of the following is a major difference between passages 1 and 2?
\begin{enumerate}[label=(\Alph*)]
\item Passage 2 addresses female's ``physical inferiority'' whereas Passage 1 does not
\item Passage 2 argues that the current status of women is subordinate to males, whereas Passage 1 does not
\item Passage 1 argues that there have been great female leaders whereas Passage 2 does not
\item Passage 2 advocates for women's rights beyond voting, whereas Passage 1 does not
\item Passage 1 utilizes a first person viewpoint, whereas Passage 2 does not
\end{enumerate}

\bigskip
\textbf{Evidence:} \hrulefill


\bigskip
\item How might the author of passage 1 respond the claim of the author of passage 2 that, ``Woman is also, though in a very much less degree, inferior to man in intellect.''

\bigskip
\begin{enumerate}[label=(\Alph*)]
\item She would disagree that this statement is a widespread belief in society
\item She would discuss her husband's opinion on this matter. 
\item She would cite that given the number of women in higher education, this attitude is changing
\item She would advocate for more formal research in this area
\item She would disagree that this point needed to be addressed in other speeches by suffragists
\end{enumerate}

\bigskip
\textbf{Evidence:} \hrulefill

\bigskip Both passages use which of the following to convey their support for women's suffrage?
\item 

\bigskip
\begin{enumerate}[label=(\Alph*)]
\item specific examples of successful female leaders
\item discussion of some aspect of the church and its implications for voting rights
\item the support of their husbands 
\item the effect of the suffrage movement on future generations
\item direct quotations from the Bible
\end{enumerate}

\bigskip
\textbf{Evidence:} \hrulefill

\bigskip
\item How might the author of passage 1 respond to the idea presented in passage 2 that ``They reject the idea of all subordination, even in the mildest form, with utter scorn. They claim for woman absolute social and political equality with man.''

\bigskip
\begin{enumerate}[label=(\Alph*)]
\item She would want to revise speeches similiar to Passage 1 to include more of the rhetoric found in this statement
\item She would urge this group to focus on gaining the vote and then potentially move on to more drastic measures
\item She would immediately dismiss this as idealistic and unable to be accomplished
\item She would wholeheartedly agree with the statement
\item She would want to discuss this statement with other suffragists before stating an opinion
\end{enumerate}

\bigskip
\textbf{Evidence:} \hrulefill
\end{enumerate}