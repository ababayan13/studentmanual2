\section{Parallelism}

Clauses within a sentence must have the same phrasing (parts of speech or verb tenses). This
frequently happens to items in a list. Parallelism can also apply to the paragraph improvement
section where consecutive sentences have similar structure.

\bigskip
For example,
\begin{enumerate}

\item Emily likes soccer, hockey, and going to parties. 

What is wrong with this sentence? 

\hrulefill

Write a correct version of the sentence:

\hrulefill
\item When at college, Emily likes to go to soccer games, football games, and play frisbee. 

What
is wrong with this sentence?

\hrulefill

Write a correct version of the sentence:

\hrulefill
\item Kelly likes to go to the mall, but riding on the mall's elevators scares her. 

What is incorrect about this sentence?

\hrulefill

Write a correct version of the sentence:

\hrulefill
\end{enumerate}

\subsection{Practice SAT Questions}

\begin{enumerate}
\item \ul{The boy spoke three languages fluently: French, Spanish, and Russian.}

\begin{enumerate}[label=(\Alph*)]
\item The boy spoke three languages fluently: French, Spanish, and Russian. 
\item The fluent boy spoke three languages: French, Spanish, and Russian.
\item The boy spoke French, Spanish, and Russian fluently.
\item The boy spoke three languages fluently French, Spanish, and Russian. 
\item The boy was able to speak three languages fluently: French, Spanish, and Russian. 
\end{enumerate}

\begin{spacing}{1.5}
\item \begin{inparaenum}[A]
Mary \tfrac{was}{\item} walking \tfrac{towards the bus}{\item} when she \tfrac{decides}{\item} that she would \tfrac{prefer to}{\item} take a cab instead. \tfrac{No Error}{\item}
\end{inparaenum}
\end{spacing}
\end{enumerate}