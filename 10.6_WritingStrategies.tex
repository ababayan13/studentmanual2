\section{SAT Worksheet: SAT Writing Multiple Choice Practice with Subject-Verb Agreement}

\textit{Directions: In the following sentences, box the subject and circle the verb that agrees with the subject.}

\begin{enumerate}
\item{I never \textbf{ have or has} long fingernails because I bit them.}


\item{Finding happiness from multiple sources \textbf{is or are} important.}


\item{The dolls in the storage closet \textbf{sits or sit} on the shelf.}


\item{Plastic engineering, a booming field in many countries, \textbf{have or has} many important applications.}


\item{Reading papers \textbf{is or are} the best part of my day.}


\item{The trails, which I climb daily, \textbf{widen or widen}s towards the end.}

\end{enumerate}

\subsection{Using the Correct Verb Tense}

Knowing when to use different verb tenses are important for SATs. Sometimes, you can look for
clues based on other part of the sentence. For example, ``Based on his previous experiences, Jared
decides/decided to pursue a career in education.''

\bigskip
It can be difficult to determine if something should be simple present, past, or future versus present
perfect, past perfect, and future perfect. For example, when should you use ``waited'' versus ``had
waited''?

\begin{itemize}
\item{We use the simple tenses when there is one year or date. Select the correct verb tense in the following sentence: World War II
began/had begun in 1939.}
\item{We use past perfect when an action has started and it is interrupted by another action (past
tense). Select the correct verb tense in the following sentence: World War II occurred/had occurred for two years before the United
States entered it in 1941.}
\item{We use present perfect for an action that began in the past and continues to the present.
Explain the difference between the following sentences:}

\begin{enumerate}
\item{Scott had lived in New York for five years before he decided to move.}
\item{Scott has lived in New York for five years, although he is currently considering moving.}
<<<<<<< HEAD
=======
\item{``The conditional (would) is used for hypothetical situations. The basic formula is \/If . . .were
. . .would." If I \textit{was\/were} to win the lottery, then I \textit{would} travel around the world.}
>>>>>>> 70cbfedac09602262caef6f421f9691f9f5a5fc2
\end{enumerate}

The first sentence \hrulefill whereas the second \hrulefill.

\item A note about SAT grammar: ``The conditional (would) is used for hypothetical situations. The basic formula is ``\textit{If \ldots were \ldots would''.} Select the correct verb tense in the following sentence: If I was/were to win the lottery, then I will/would travel around the world.

\end{itemize}

\subsection{Sample SAT Practice Questions}

\begin{enumerate}
\item \begin{inparaenum}[A]
The janitors \tfrac{are}{\item} \tfrac{requiring}{\item} to mop the floors, \tfrac{wipe}{\item} the windows, and clean the chalkboard \tfrac{daily}{\item}. \tfrac{No Error}{\item}
\end{inparaenum}

\item \begin{inparaenum}[A]
The teachers, \tfrac{iinspired by}{\item} the novel \tfrac{pedagogical}{\item} techniques they learned at the conference, \tfrac{pledge to}{\item} utilize these methods to improve their \tfrac{teaching}{\item}. \tfrac{No Error}{\item} 
\end{inparaenum}

\end{enumerate}
